\usepackage{bigstrut}
\usepackage{makecell}
\usepackage{tikz-dependency}
\usepackage{subcaption}
\usepackage{wrapfig}
\usepackage{xcolor}
\usepackage{calc}
\usepackage{graphicx}
\usetikzlibrary{decorations.markings, arrows.meta}
\usetikzlibrary{decorations.pathreplacing,calligraphy}
\usetikzlibrary{automata, arrows, calc, matrix, positioning, math}
\usetikzlibrary{intersections}
\pgfdeclarelayer{background}
\pgfsetlayers{background,main}
\tikzset{>=stealth}
\newcommand{\catstyle}[2]{
	\tikzset{#1/.style={color=#2}}
}
\tikzset{dot/.style={circle,draw=black,fill=black,minimum size=1mm,inner sep=0mm}}


\def\cont{\Gamma\vdash}
\def\poulpe{\qquad}

\DeclareMathOperator{\Var}{Var}

\def\ppl{\mathbin{+\mkern-12mu+}}

\definecolor{vulm}{HTML}{7d1dd3}
\definecolor{yulm}{HTML}{ffe500}

\usepackage{amsmath, amsfonts, amssymb, amsthm}
\usepackage{mathrsfs}
\usepackage{dsfont}
\usepackage{stmaryrd}
\usepackage{mathtools}
\usepackage{tikz-qtree}
\usepackage{tipa}
\usepackage{tikz-cd}
\usepackage{nicematrix}
\usepackage{contour}
\usepackage{multicol}
\contourlength{0.005em}
\def\backbox#1{\contour{black}{#1}}

\def\ty#1{\ensuremath{\texttt{\color{yulm!70!black}#1}}}
\def\f#1{\ensuremath{\texttt{\color{vulm}#1}}}
\def\w#1{\ensuremath{\mathbf{#1}}\,}

\def\e{\ty{e}}
\def\t{\ty{t}}
\def\r{\ty{r}}
\def\ta{\ty{a}}
\def\tb{\ty{b}}

% Symbol Shorthands
\newcommand{\Id}{\mathrm{Id}}
\newcommand{\N}{\mathbb{N}}
\newcommand{\Z}{\mathbb{Z}}
\newcommand{\Q}{\mathbb{Q}}
\newcommand{\R}{\mathbb{R}}
\newcommand{\C}{\mathbb{C}}
\newcommand{\K}{\mathbb{K}}
\newcommand{\E}{\mathbb{E}}
\newcommand{\T}{\mathbb{T}}
\newcommand{\U}{\mathbb{U}}
\renewcommand{\P}{\mathbb{P}}
\newcommand{\HH}{\mathbb{H}}
\renewcommand{\S}{\mathbb{S}}
\newcommand{\term}{\mathds{1}}
\newcommand{\I}{\mathbb{I}}

\renewcommand{\O}{\mathcal{O}}
\newcommand{\A}{\mathcal{A}}
\newcommand{\mL}{\mathcal{L}}
\newcommand{\M}{\mathcal{M}}
\newcommand{\mC}{\mathcal{C}}
\newcommand{\B}{\mathcal{B}}
\newcommand{\ml}{\mathcal{l}}
\newcommand{\mP}{\mathcal{P}}
\newcommand{\mT}{\mathcal{T}}
\newcommand{\mD}{\mathcal{D}}
\newcommand{\mG}{\mathcal{G}}
\newcommand{\mE}{\mathcal{E}}
\newcommand{\mS}{\mathcal{S}}
\newcommand{\mF}{\mathcal{F}}
\newcommand{\mU}{\mathcal{U}}

\renewcommand{\phi}{\varphi}
\renewcommand{\epsilon}{\varepsilon}
\renewcommand{\emptyset}{\varnothing}
\renewcommand\qedsymbol{$\blacksquare$}
\newcommand{\id}{\mathrm{id}}
\newcommand{\walrus}{:=}
\newcommand{\suchthat}{\,\middle|\,}

% Operators
\DeclareMathOperator{\Card}{Card}
\DeclareMathOperator{\Hom}{Hom}
\DeclareMathOperator{\Obj}{Obj}

\newcommand{\abs}[1]{\left|#1\right|}
\newcommand{\norm}[1]{\left\lVert #1 \right\rVert}
\newcommand{\scalar}[1]{\left\langle #1 \right\rangle}

% Grammar in math mode
\newenvironment{mgrammar}%
{
	\setlength\tabcolsep{4pt}
	\begin{tabular}{>{$}l<{$}>{$}r<{$}>{$}l<{$} r}
		}
		{
	\end{tabular}
}

% Grammar in text mode
\newenvironment{grammar}%
{
	\setlength\tabcolsep{4pt}
	\begin{tabular}{l r l r}
		}
		{
	\end{tabular}
}

% Command for grammar lines
\newcommand*{\firstrule}[3]{#1 &::= & #2& \quad \ifstrempty{#3}{}{\textit{(#3)}}\\}
\newcommand*{\lfrule}[3]{#1 &::= & #2& \quad \ifstrempty{#3}{}{\textit{(#3)}}}
\newcommand*{\grule}[2]{&| &#1& \quad \ifstrempty{#2}{}{\textit{(#2)}}\\}
\newcommand*{\gskip}{&&\\}

\newcommand\fracpush{\hfill\mbox{}}
\newcommand\fracnotate[1]{\fracpush\rlap{#1}}


\newcolumntype{C}{>{$}c<{$}}
	\newcolumntype{L}{>{$}l<{$}}
\newcolumntype{R}{>{$}r<{$}}
\def\fmap{\texttt{fmap}}

\newcommand{\suppfrac}[2]{%
	\sbox0{$\genfrac{}{}{0pt}{0}{\ensuremath{#1}}{\ensuremath{#2}}$}%
	\ooalign{%
		\hidewidth
		$\vcenter{\moveright\nulldelimiterspace
				\hbox to\wd0{%
					\xleaders\hbox{\kern.5pt\vrule height 0.4pt width 1.5pt\kern.5pt}\hfill
					\kern-1.5pt
				}%
			}$
		\hidewidth\cr
		$\genfrac{}{}{0pt}{0}{\raisebox{5pt}{\ensuremath{#1}}}{\ensuremath{#2}}$\cr}%
}


\newcommand{\prooffrac}[2]{\genfrac{}{}{}{0}{\ensuremath{#1}}{\ensuremath{#2}}}%
\DeclareMathOperator{\mFunc}{\mathcal{F}}

\def\blank{\phantom{.}}
\def\PT#1#2{\suppfrac{#1}{#2}\fracnotate{$\top$}\phantom{\top\quad}}

\def\Japp#1#2#3#4{%
	\prooffrac{%
		\PT{\blank}{\cont #1: #2 \to #3} \poulpe \PT{\blank}{\cont #4: #2}
	}{%
		\cont #1 #4: #3
	}\fracnotate{App}\phantom{\quad App}
}

\def\Jconj#1#2#3{%
	\prooffrac{%
		\PT{\blank}{\cont #1: #3 \to \t} \poulpe
		\PT{\blank}{\cont #2: #3 \to \t}
	}{%
		\cont #1 \land #2: #3 \to \t
	}\fracnotate{$\land$}\phantom{\quad \land}
}


\def\Jfmap#1#2#3#4#5{%
	\prooffrac{%
		\PT{\blank}{\cont #1: \mC \Rightarrow \mC} \poulpe \PT{\blank}{\cont #2 = #1#2': #1#3} \poulpe \PT{\PT{\blank}{\cont #2':#3} \poulpe \PT{\blank}{\cont #4:#3 \to #5}}{\cont #4#2': #5}
	}{%
		\cont #2#4: #1#5
	}\fracnotate{\texttt{fmap}}\phantom{\quad\tt fmap}
}

\def\Junit#1#2#3#4#5{%
	\prooffrac{%
		\PT{\blank}{\cont #1: \mC \Rightarrow \mC} \poulpe
		\PT{\blank}{\cont #2 = #1#2': #1\left(#3 \to #5\right)} \poulpe
		\PT{%
			\PT{\blank}{\cont #2':#3 \to #5} \poulpe
			\PT{\blank}{\cont #4:#3}
		}{\cont #2'#4: #5}
	}{%
		\cont #2#4: #1#5
	}\fracnotate{\tt pure/return}\phantom{\quad \tt pure/return}
}

\def\sobj{\mathit{Sem}}
\def\combMR{\text{MR}}
\def\combML{\text{ML}}
\def\combUR{\text{UR}}
\def\combUL{\text{UL}}
\def\combC{\text{C}}
\def\combJ{\text{J}}
\def\combA{\text{A}}
\def\combER{\text{ER}}
\def\combEL{\text{EL}}
\def\combDN{\text{DN}}

\makeatletter
\newcommand{\@word}[4][]{%
	#2 & #3 & #4\\
	\ifx&#1&%
	%
	\else
	&\multicolumn{2}{l}{Generalizes to \textbf{#1}}\\%
	\fi%
}
\def\word#1#2#3#4{\@word[#4]{#1}{#2}{#3}}
\makeatother

\usepackage{calc}

\makeatletter
\def\textSq#1{%
	\begingroup% make boxes and lengths local
	\setlength{\fboxsep}{0.4ex}% SET ANY DESIRED PADDING HERE
	\setbox1=\hbox{#1}% save the contents
	\setlength{\@tempdima}{\maxof{\wd1}{\ht1+\dp1}}% size of the box
	\setlength{\@tempdimb}{(\@tempdima-\ht1+\dp1)/2}% vertical raise
	\raise-\@tempdimb\hbox{\fbox{\vbox to \@tempdima{%
				\vfil\hbox to \@tempdima{\hfil\copy1\hfil}\vfil}}}%
	\endgroup%
}
\def\Sq#1{\textSq{\ensuremath{#1}}}%

\def\c@lsep{2.3}
\def\r@wsep{.5}
\def\wordsep{.2}

\tikzset{
	uptree/.style={
			draw=vulm!80!black,
			thick,
		},
	typenode/.style={
			align=center,
			text width=24mm,
		},
	treenode/.style={
			align=center,
			text width=24mm,
		},
	wordnode/.style={
			inner sep=0pt,
			align=center,
		},
	downtree/.style={
			draw=yulm!80!black,
			thick,
		},
}

\newcommand{\wnode}[3]{%
	\node (#2) at (#1*\c@lsep, 0) [wordnode] {#2};
	\node[anchor=north] (#2-) at ($(#1*\c@lsep, 0) + (0, -.142)$) [typenode] {\ensuremath{#3}};
}
\newcommand{\utnode}[3]{%
	\path let \p1 = (#2.north), \p2 = (#3.north) in coordinate (Q1) at (\x1, {max(\y1, \y2)});
	\path let \p1 = (#2.north), \p2 = (#3.north) in coordinate (Q2) at (\x2, {max(\y1, \y2)});
	\node (#2#3) at ($($(Q1)!0.5!(Q2)$) + (0, \r@wsep)$) [treenode] {\ensuremath{#1}};
	\draw[uptree] ($(#2) + (0, \wordsep)$) -- ($(#2#3) + (0, -\wordsep)$);
	\draw[uptree] ($(#3) + (0, \wordsep)$) -- ($(#2#3) + (0, -\wordsep)$);
}
\newcommand{\dtnode}[4][0.5]{%
	\path let \p1 = (#3.south), \p2 = (#4.south) in coordinate (Q1) at (\x1, {min(\y1, \y2)});
	\path let \p1 = (#3.south), \p2 = (#4.south) in coordinate (Q2) at (\x2, {min(\y1, \y2)});
	\node (#3#4) at ($($(Q1)!#1!(Q2)$) + (0, -\r@wsep)$) [treenode] {\ensuremath{#2}};
	\draw[downtree] ($(#3) + (0, -\wordsep)$) -- ($(#3#4) + (0, +\wordsep)$);
	\draw[downtree] ($(#4) + (0, -\wordsep)$) -- ($(#3#4) + (0, +\wordsep)$);
}

\makeatother

\catstyle{catone}{gray!50}
\catstyle{catmc}{vulm!10!yulm}
\catstyle{catmca}{vulm!20!yulm}
\catstyle{catmcb}{vulm!30!yulm}
\catstyle{catmcc}{vulm!40!yulm}
\catstyle{catmcd}{vulm!50!yulm}
\catstyle{catmce}{vulm!60!yulm}
\catstyle{catmcf}{vulm!70!yulm}
\catstyle{catmcg}{vulm!80!yulm}
\catstyle{catmch}{vulm!90!yulm}

\def\din#1{#1\mathrm{.S}}
\def\dnb#1{#1\mathrm{.N}}
\def\dlb#1#2{#1\mathrm{.L}\left(#2\right)}
\def\dl#1{#1\mathrm{.L}}
\def\dnlg#1{#1\mathrm{.h}}
\def\dnin#1{#1\mathrm{.in}}
\def\dnout#1{#1\mathrm{.out}}

\tikzset{
	on each segment/.style={
			decorate,
			decoration={
					show path construction,
					moveto code={},
					lineto code={
							\path [#1]
							(\tikzinputsegmentfirst) -- (\tikzinputsegmentlast);
						},
					curveto code={
							\path [#1] (\tikzinputsegmentfirst)
							.. controls
							(\tikzinputsegmentsupporta) and (\tikzinputsegmentsupportb)
							..
							(\tikzinputsegmentlast);
						},
					closepath code={
							\path [#1]
							(\tikzinputsegmentfirst) -- (\tikzinputsegmentlast);
						},
				},
		},
	% style to add an arrow in the middle of a path
	mid arrow/.style={postaction={decorate,decoration={
							markings,
							mark=at position .5 with {\arrow[#1]{stealth}}
						}}},
}

\usepackage{adjustbox}
\def\columneqs#1{\adjustbox{width=\columnwidth}{\ensuremath{#1}}}
\def\columneq#1{\adjustbox{width=.8\columnwidth}{\ensuremath{#1}}}
\tikzcdset{scale cd/.style={every label/.append style={scale=#1},
			cells={nodes={scale=#1}}}}
