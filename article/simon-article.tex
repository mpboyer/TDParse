\documentclass[a4paper,UKenglish,cleveref, autoref, thm-restate]{lipics-v2021}
%This is a template for producing LIPIcs articles. 
%See lipics-v2021-authors-guidelines.pdf for further information.
%for A4 paper format use option "a4paper", for US-letter use option "letterpaper"
%for british hyphenation rules use option "UKenglish", for american hyphenation rules use option "USenglish"
%for section-numbered lemmas etc., use "numberwithinsect"
%for enabling cleveref support, use "cleveref"
%for enabling autoref support, use "autoref"
%for anonymousing the authors (e.g. for double-blind review), add "anonymous"
%for enabling thm-restate support, use "thm-restate"
%for enabling a two-column layout for the author/affilation part (only applicable for > 6 authors), use "authorcolumns"
%for producing a PDF according the PDF/A standard, add "pdfa"

%\pdfoutput=1 %uncomment to ensure pdflatex processing (mandatatory e.g. to submit to arXiv)
%\hideLIPIcs  %uncomment to remove references to LIPIcs series (logo, DOI, ...), e.g. when preparing a pre-final version to be uploaded to arXiv or another public repository

%\graphicspath{{./graphics/}}%helpful if your graphic files are in another directory

\bibliographystyle{plainurl}% the mandatory bibstyle

\title{A Diagrammatic Calculus for a Functional Model of Natural Language Semantics} %TODO Please add

%\titlerunning{Dummy short title} %TODO optional, please use if title is longer than one line

\author{Matthieu Pierre Boyer}{DI ENS, Paris, France \and Department of Linguistics, Yale University, USA \and \url{http://www.eleves.ens.fr/home/mpboyer}}{matthieu.boyer@ens.fr}{https://orcid.org/0000-0002-1825-0097}{}
\author{Simon Charlow}{Department of Linguistics, Yale University, USA}{simon.charlow@yale.edu}{https://orcid.org/0000-0002-1825-0097}{}

\authorrunning{M. P. Boyer and S. Charlow}

\Copyright{Matthieu P. Boyer and Simon Charlow}

\ccsdesc[100]{Models of computation. Document management and text processing.} %TODO mandatory: Please choose ACM 2012 classifications from https://dl.acm.org/ccs/ccs_flat.cfm 

\keywords{Typing, Natural Language Semantics, Parsing, Side Effects, String Diagrams} %TODO mandatory; please add comma-separated list of keywords

\category{Student Paper} %optional, e.g. invited paper

\relatedversion{} %optional, e.g. full version hosted on arXiv, HAL, or other respository/website
%\relatedversiondetails[linktext={opt. text shown instead of the URL}, cite=DBLP:books/mk/GrayR93]{Classification (e.g. Full Version, Extended Version, Previous Version}{URL to related version} %linktext and cite are optional

%\supplement{Link to a code for the parser will %optional, e.g. related research data, source code, ... hosted on a repository like zenodo, figshare, GitHub, ...
%\supplementdetails[linktext={opt. text shown instead of the URL}, cite=DBLP:books/mk/GrayR93, subcategory={Description, Subcategory}, swhid={Software Heritage Identifier}]{General Classification (e.g. Software, Dataset, Model, ...)}{URL to related version} %linktext, cite, and subcategory are optional

\acknowledgements{I want to thank my mother for the knitting and putting up
  with me peeling potatos to try knitting with toothpicks; Antoine Groudiev
  for his precious insights on how to label equations; Paul-André Melliès for
  his insights; Bob Frank and Bella Senturia for the help with the minimalistic
  merge syntactic theories; and of course, Simon Charlow for his advising
  during my time at Yale, and his help around the linguistics questions that
definitely arose.}

\nolinenumbers %uncomment to disable line numbering

%Editor-only macros:: begin (do not touch as author)%%%%%%%%%%%%%%%%%%%%%%%%%%%%%%%%%%
\EventEditors{John Q. Open and Joan R. Access}
\EventNoEds{2}
\EventLongTitle{42nd Conference on Very Important Topics (CVIT 2016)}
\EventShortTitle{CVIT 2016}
\EventAcronym{CVIT}
\EventYear{2016}
\EventDate{December 24--27, 2016}
\EventLocation{Little Whinging, United Kingdom}
\EventLogo{}
\SeriesVolume{42}
\ArticleNo{23}
%%%%%%%%%%%%%%%%%%%%%%%%%%%%%%%%%%%%%%%%%%%%%%%%%%%%%%

% Document defined Commands
\usepackage{bigstrut}
\usepackage{makecell}
\usepackage{emoji}
\usepackage{tikz-dependency}
\usepackage{diag}

\def\cont{\Gamma\vdash}
\def\poulpe{\qquad}

\setlength\belowcaptionskip{0pt}
\setlength\abovecaptionskip{\baselineskip}

\DeclareMathOperator{\Var}{Var}

\def\ppl{\mathbin{+\mkern-12mu+}}

\makeatletter
\renewenvironment{thebibliography}[1]
     {\section{\bibname}
      \@mkboth{\MakeUppercase\bibname}{\MakeUppercase\bibname}%
      \list{\@biblabel{\@arabic\c@enumiv}}%
           {\settowidth\labelwidth{\@biblabel{#1}}%
            \leftmargin\labelwidth
            \advance\leftmargin\labelsep
            \@openbib@code
            \usecounter{enumiv}%
            \let\p@enumiv\@empty
            \renewcommand\theenumiv{\@arabic\c@enumiv}}%
      \sloppy
      \clubpenalty4000
      \@clubpenalty \clubpenalty
      \widowpenalty4000%
      \sfcode`\.\@m}
     {\def\@noitemerr
       {\@latex@warning{Empty `thebibliography' environment}}%
      \endlist}

\def\black@or@white#1#2{%
  \@tempdima#2 pt
  \ifdim\@tempdima>0.5 pt
    \definecolor{temp@c}{gray}{0}%
  \else
    \definecolor{temp@c}{gray}{1}%
  \fi}
\def\letterbox#1#{\protect\letterb@x{#1}}
\def\letterb@x#1#2#3{%
  \colorlet{temp@c}[gray]{#2}%
  \extractcolorspec{temp@c}{\color@spec}%
  \expandafter\black@or@white\color@spec
  {\color#1{temp@c}\tallcbox#1{#2}{#3}}}
\def\tallcbox#1#{\protect\color@box{#1}}
\def\color@box#1#2{\color@b@x\relax{\color#1{#2}}}
\long\def\color@b@x#1#2#3%
 {\leavevmode
  \setbox\z@\hbox{{\set@color#3}}%
  \ht\z@\ht\strutbox
  \dp\z@\dp\strutbox
  {#1{#2\color@block{\wd\z@}{\ht\z@}{\dp\z@}\box\z@}}}
\makeatother

\contourlength{0.005em}
\def\backbox#1{\letterbox{Lavender!40}{\contour{black}{#1}}}

\def\ty#1{\backbox{\tt\color{yulm!90!black}#1}}
\def\f#1{\backbox{\tt\color{vulm}#1}}
\def\w#1{\mathbf{#1}\,}

\def\e{\ty{e}}
\def\t{\ty{t}}
\def\r{\ty{r}}

\newcolumntype{C}{>{$}c<{$}}
\newcolumntype{L}{>{$}l<{$}}
\newcolumntype{R}{>{$}r<{$}}
\def\fmap{\texttt{fmap}}


\makeatletter
\newcommand{\@word}[4][]{%
	#2 & #3 & #4\\
\ifx&#1&%
	%
\else
	&\multicolumn{2}{l}{Generalizes to \textbf{#1}}\\%
\fi%
}
\def\word#1#2#3#4{\@word[#4]{#1}{#2}{#3}}
\makeatother

\usepackage{calc}

\makeatletter
\def\textSq#1{%
\begingroup% make boxes and lengths local
\setlength{\fboxsep}{0.4ex}% SET ANY DESIRED PADDING HERE
\setbox1=\hbox{#1}% save the contents
\setlength{\@tempdima}{\maxof{\wd1}{\ht1+\dp1}}% size of the box
\setlength{\@tempdimb}{(\@tempdima-\ht1+\dp1)/2}% vertical raise
\raise-\@tempdimb\hbox{\fbox{\vbox to \@tempdima{%
  \vfil\hbox to \@tempdima{\hfil\copy1\hfil}\vfil}}}%
\endgroup%
}
\def\Sq#1{\textSq{\ensuremath{#1}}}%

\def\c@lsep{2.3}
\def\r@wsep{.8}

\tikzset{
	uptree/.style={
			draw=green!80!black,
			thick,
		},
	typenode/.style={
			align=center,
			text width=24mm,
			%font={\large},
		},
	treenode/.style={
			align=center,
			text width=24mm,
		},
	wordnode/.style={
			inner sep=0pt,
			align=center,
			font={\large},
		},
	downtree/.style={
			draw=red!80!black,
			thick,
		},
}

\newcommand{\wnode}[3]{%
	\node (#2) at (#1*\c@lsep, 0) [wordnode] {#2};
	\node[anchor=north] (#2-) at ($(#1*\c@lsep, 0) + (0, -.142)$) [typenode] {\ensuremath{#3}};
}
\newcommand{\utnode}[3]{%
	\path let \p1 = (#2.north), \p2 = (#3.north) in coordinate (Q1) at (\x1, {max(\y1, \y2)});
	\path let \p1 = (#2.north), \p2 = (#3.north) in coordinate (Q2) at (\x2, {max(\y1, \y2)});
	\node (#2#3) at ($($(Q1)!0.5!(Q2)$) + (0, 1)$) [treenode] {\ensuremath{#1}};
	\draw[uptree] ($(#2.north) + (0, .142)$) -- (#2#3.south);
	\draw[uptree] ($(#3.north) + (0, .142)$) -- (#2#3.south);
}
\newcommand{\dtnode}[4][0.5]{%
	\path let \p1 = (#3.south), \p2 = (#4.south) in coordinate (Q1) at (\x1, {min(\y1, \y2)});
	\path let \p1 = (#3.south), \p2 = (#4.south) in coordinate (Q2) at (\x2, {min(\y1, \y2)});
	\node (#3#4) at ($($(Q1)!#1!(Q2)$) + (0, -1)$) [treenode] {\ensuremath{#2}};
	\draw[downtree] ($(#3.south) + (0, -.142)$) -- (#3#4.north);
	\draw[downtree] ($(#4.south) + (0, -.142)$) -- (#3#4.north);
}

\def\inputtikz#1{
	\ifnum\tikzimp@rt=1
		\input{figures/#1}
	\else
		\ensuremath{\text{\Huge\color{vulm}A TikZ PICTURE GOES HERE.}}
	\fi
}
\makeatother

\catstyle{catone}{gray!50}
\catstyle{catmc}{vulm!10!yulm}
\catstyle{catmca}{vulm!20!yulm}
\catstyle{catmcb}{vulm!30!yulm}
\catstyle{catmcc}{vulm!40!yulm}
\catstyle{catmcd}{vulm!50!yulm}
\catstyle{catmce}{vulm!60!yulm}
\catstyle{catmcf}{vulm!70!yulm}
\catstyle{catmcg}{vulm!80!yulm}
\catstyle{catmch}{vulm!90!yulm}

\def\din#1{#1\mathrm{.S}}
\def\dnb#1{#1\mathrm{.N}}
\def\dlb#1#2{#1\mathrm{.L}\left(#2\right)}
\def\dl#1{#1\mathrm{.L}}
\def\dnlg#1{#1\mathrm{.h}}
\def\dnin#1{#1\mathrm{.in}}
\def\dnout#1{#1\mathrm{.out}}

\newcounter{lingexcnt}
\newcounter{tmplingexcnt}
\renewcommand*{\thelingexcnt}{(\arabic{lingexcnt})}
\newenvironment{sentence}[1][]{
     \begin{list}{\thelingexcnt}{\refstepcounter{lingexcnt}}\item
     \ifnum\pdfstrcmp{#1}{}=0\else\label{#1}\fi
}{\end{list}}

\newenvironment{nsentence}{%
     \setcounter{tmplingexcnt}{\value{lingexcnt}}
     \addtocounter{tmplingexcnt}{-1}
     \begin{list}{\thelingexcnt}{
         \usecounter{lingexcnt}
         \setcounter{lingexcnt}{\value{tmplingexcnt}}
         \refstepcounter{lingexcnt}
     }
}{\end{list}}

\newcommand*{\oneSentence}[2][]{\begin{sentence}[#1]#2\end{sentence}}


\begin{document}

\maketitle

\begin{abstract}
	In this paper, we study a functional programming approach to natural language
	semantics, allowing us to increase the expressivity of a more traditional
	denotation style.
	We will formalize a category based type and effect system, and construct a
	diagrammatic calculus to model parsing and handling of effects, and use it to
	efficiently compute the denotations for sentences.
\end{abstract}

\section{Introduction}
What is \emph{a chair}? How do I know that \emph{Jupiter, a planet}, is
\emph{a planet}?
To answer those questions, \cite{bumfordEffectdrivenInterpretationFunctors2025}
provide a \textsc{Haskell} based view on the notion of typing in natural
language semantics.
Their main idea is to include a layer of effects which allows for improvements
in the expressiveness of the denotations used.
This allows to model complex concepts such as anaphoras, or non-determinism in
an easy way, independent of the actual way the words are represented.
Indeed, when considering the usual denotations of words as typed lambda-terms,
this allows us to solve the issue of meaning getting lost through impossible
typing, while still being able to compose meanings properly.
When two expressions have the same syntactic distribution, they must also have
the same type, which forces quantificational noun phrases to have the same type
as proper nouns: the entity type $\e$.
However, there is no singular entity that is the referent of \emph{every
	planet}, and so, the type system gets in the way of meaning, instead of being
a tool at its service.

\smallskip

Our formalism is inscribed in the contemporary natural language semantic
theories which are based on three main elements: a \emph{lexicon}, a
\emph{syntactic description} of the language, and a theory of
\emph{composition}.
More specifically, we explain how to extend the domain of the lexicon and the
theory of composition to account for the phenomena described above.
We will not be discussing most of the linguistic foundations for the usage of
the formalism, nor its usefulness.
We refer the reader to \cite{bumfordEffectdrivenInterpretationFunctors2025} to
get an overview of the linguistic considerations that are the base of the
theory.

\smallskip

In this paper, we will provide a formal definition of an enhanced type and
effect system for natural language semantics, based on categorical tools.
This will increase the complexity (both in terms of algorithmic operations and
in comprehension of the model) of the parsing algorithms, but through the use
of string diagrams to model the effect of composition on potential effectful
denotations (or more generally computations), we will provide efficient
algorithms for computing the set of meanings of a sentence, from the meaning of
its components.

\section{Related Work}
This is not the first time a categorical representation of compositional
semantics of natural language is proposed,
\cite{coeckeMathematicalFoundationsCompositional2010} already suggested an
approach based on monoidal categories using an external model of meaning.
What our approach gives more, is additional latitude for the definition of
denotations in the lexicon, and a visual explanation of the difference between
multiple possible parsing trees.
We will go back later on the differences between their Lambek inspired grammars
and our more abstract way of looking at the semantic parsing of a sentence.

\smallskip

On a completely different approach,
\cite{marcollimatildeetchomskynoametberwickrobertc.MathematicalStructureSyntactic}
provide a categorical structure based on Hopf algebra and coloured operads
to explain their model of syntax, leading to results at the interface of syntax
and morphology presented in \cite{senturiaAlgebraicStructureMorphosyntax2025}.
Similarly, \cite{melliesCategoricalContoursChomskySchutzenberger2025} provides
a modeling of CFGs using coloured operads.
Our approach is based on the suggestion that merge in syntax can be done using
labels, independent on how it is mathematically modelled.

\section{Categorical Semantics of Effects: A Typing System}
\label{sec:typingsystem}
In this section, we will formalize a type system underlying the theory proposed
in \cite{bumfordEffectdrivenInterpretationFunctors2025}.
To do so, we will designate by $\mL$ our language, as a set of words
(with their associated meaning/denotation) and syntactic rules underlying
the semantic combination.
The absence of syntactic rules is allowed, although it partly defeats the
purpose of this work.
This might be useful when proposing compositional models of learned
representations.

Let $\O\left( \mL \right)$ be the set of words in the language whose semantic
representation is a low-order function and $\mF\left( \mL \right)$ the set of
words whose semantic representation is a functor or high-order function.
Our goals here are to describe more formally, using a categorical vocabulary,
the environment in which the typing system for our language will exist, and how
we connect words and other linguistic objects to the categorical formulation.

\subsection{Typing Category}
\subsubsection{Types}
Let $\mC$ be a closed cartesian category, which is used to represent the
domain of types for our domain of uneffectful denotations.
This represents our main typing system, consisting of types for words $\O(\mL)$
that can be expressed without effects (see Figure \ref{fig:lexicon} for an
example).
Remember that $\mC$ contains a terminal object $\bot$ representing the empty
type or the lack thereof.
We consider as our typing category $\bar{\mC}$ the categorical closure for
exponentials and products of $\mF\left( \mL \right)^{*}\left(\mC\right)$,
which consists of all the different type constructors (ergo, functors) that
could be formed in the language.
In that setting our types are those that can be attained in a finite number
from a finite number of functorial applications from an object of $\mC$.

Formally, we consider for our types the quotient set
$\star = \mathrm{Obj}\left( \bar{\mC} \right)/\mF\left( \mL \right)$.
Since $\mF\left( \mL \right)$ does not induce an equivalence relation on
$\Obj\left( \bar{\mC} \right)$ but a preorder, we consider the chains obtained
by the closure of the relation $x\succeq y \Leftrightarrow \exists F, y = F(x)$
(which should be seen as a subtyping relation as proposed in
\cite{melliesFunctorsAreType2015}).
We also define $\star_{0}$ to be the set obtained when considering types which
have not yet been \emph{affected}, that is $\Obj(\mC)$.
In contexts of polymorphism, we identify $\star_{0}$ to the adequate subset of
$\star$.
In this paradigm, constant objects (or results of fully done computations) are
functions with type $\bot \to \tau$ which we will denote directly by
$\tau \in \star_{0}$.
This will be useful when defining base combinators in Section \ref{sec:parsing}.

\subsubsection{Functors, Applicatives and Monads}
Our point of view has us consider \emph{language functors}\footnote{The
	elements of our language, not the categorical construct.} as polymorphic
functions: for a (possibly restrained) set of base types $S$, a functor is a
function:
\begin{equation*}
	x: \tau\in S\subseteq \star \mapsto F x: F\tau
\end{equation*}
This means that if a functor can be applied to a type, it can also be applied
to all \emph{affected} versions of that type, i.e.
$\mF\left( L \right)(\tau\in \star)$.
This gives us two typing judgements for the functor $F$:
\begin{equation*}
	\frac{\Gamma\vdash x: \tau \in \star_{0}}{\Gamma\vdash F x: F\tau \notin
		\star_{0}} \hspace{2cm} \frac{\Gamma\vdash x:
		\tau}{\Gamma\vdash Fx : F\tau\preceq \tau}
\end{equation*}
We use the same notation for the \emph{language functor} and the
\emph{type functor} in the examples, but it is important to note those are two
different objects, although connected.
More precisely, the \emph{language functor} is to be seen as a function whose
computation yields an effect, while the \emph{type functor} is the endofunctor
of $\bar{\mC}$ (so a functor from $\mC$) that represents the effect in our
typing category.
Examples are provided in Figure \ref{fig:functors}.

\smallskip

In this regard, applicatives and monads only provide with more flexibility on
the ways to combine functions:
they provide intermediate judgements to help with the combination of trees.
For example, the multiplication of the monad provides a new
\emph{type conversion} judgement:
\begin{equation*}
	\frac{\Gamma\vdash x: MM\tau}{\Gamma\vdash x: M\tau \succeq MM\tau}
\end{equation*}
This is actually a special case of the natural transformation rule that we
define below, which means that, in a way, types $MM\star$ and $M\star$ are
equivalent, as there is a canonical way to go from one type to another.
Remember however that $M\star$ is still a proper subtype of $MM\star$ and that
the objects are not actually equal: they are simply equivalent.

\subsubsection{Natural Transformations}
\label{subsubsec:transnat}
We could add judgements directly for adjunctions, but we instead add judgements
for natural transformations, as adjunctions are natural transformations which
arise from \emph{natural} settings.
While in general we do not want to find natural transformations, we want to be
able to express these in three situations:
\begin{enumerate}
	\item Adjunctions.
	\item To deal with the resolution of effects as explained in Section
	      \ref{sec:nondet}
	\item To create \emph{higher-order} constructs which transform words from our
	      language into other words, while keeping the functorial aspect.
	      This idea is developed in Section \ref{par:higherorder}.
\end{enumerate}
To see why we want this rule, which is a larger version of the monad
multiplication and the monad/applicative unit, it suffices to see that the
diagram defining the properties of a natural transformation provides a way
to construct the \emph{correct function} on the \emph{correct functor} side of
types.

\smallskip

In the Haskell programming language, any polymorphic function is
a natural transformation from the first type constructor to the second type
constructor, as proved in \cite{wadlerTheoremsFree1989}.
This will guarantee for us that given a \emph{Haskell} construction for a
polymorphic function, we will get the associated natural transformation.

\paragraph{Handlers}
As introduced by \cite{marsikAlgebraicEffectsHandlers}, the use of handlers
as annotations to the semantic tree of the sentence is an appropriate
formalism.
As considered by \cite{wuEffectHandlersScope2014}, handlers are to be seen
as natural transformations describing the free monad on an algebraic effect.
Considering handlers as so, allows us to directly handle our computations
inside our typing system, by ``transporting'' our functors one order higher up
without loss of information or generality since all our functors undergo the
same transformation.
Using the framework proposed in \cite{vandenbergFrameworkHigherorderEffects2024}
we simply need to create handlers for our effects/functors and we will then
have in our language the result needed.
The only thing we will require from an algebraic handler $h$ is that for any
applicative functor of unit $\eta$, $h\circ \eta = \id$.

\smallskip

Note that the choice of the handler being part of the lexicon or the parser
over the other is a philosophical question more than a semantical one, as both
options will result in semantically equivalent models, the only difference will
be in the way we consider the resolution of effects.
This choice does not arise in the case of the adjunction-induced
handlers.
Indeed here, the choice is caused by the non-uniqueness of the choices for
the handlers as two different speakers may have different ways to resolve the
non-determinism effect that arises from the phrase \textsl{A chair}.
This is the difference with the adjunctions: adjunctions are intrinsic
properties of the coexistence of the effects, while the handlers
are user-defined.

\paragraph{Higher-Order Constructs}
\label{par:higherorder}
Functors may also be used to add plurals, superlatives, tenses, aspects and
other similar constructs which act as function modifiers.
For each of these, we give a functor $\Pi$ corresponding to a new class of
types along with natural transformations for all other functors $F$ which
allows to propagate down the high-order effect.
This transformation will need to be from $\Pi \circ F$ to
$\Pi \circ F \circ \Pi$ or simply $\Pi \circ F \Rightarrow F \circ \Pi$
depending on the situation.
This allows us to add complexity not in the compositional aspects but
in the lexicon aspects, by simply saying that those constructs are predicate
modifiers passed down (with or without side effects) to the arguments of
predicates:
\begin{equation*}
	\begin{aligned}
		\mathbf{future\left( be \right)\left( arg_{1}, arg_{2} \right)}
		 & \xrightarrow{\eta} \mathbf{future\left( be \right)\left( arg_{2} \right)\left( future\left( arg_{1} \right) \right)}                           \\
		 & \xrightarrow{\eta} \mathbf{future \left( be \right) \left( future \left( arg_{2} \right) \right) \left( future \left( arg_{1} \right) \right)}
	\end{aligned}
\end{equation*}

Among other higher-order constructs that might be represented using effects are
scope islands, which could be modelled by a functor that cannot be
passed as argument to words that would otherwise need a closure to be applied
first.
See Figure \ref{fig:tree-rain} for an example, based on theory presented in
\cite{bumfordEffectdrivenInterpretationFunctors2025}, Section 5.4.

\paragraph{Monad Transformers}
In \cite{bumfordEffectdrivenInterpretationFunctors2025}, the authors present
constructions which they call monad transformers or \emph{higher-order
	constructors} and which take a monad as input and return a monad as output.
One way to type those easily would be to simply create, for each such
construct, a monad (the result of the application to any other monad) and a
natural transformation which mimics the application and can be seen as the
constructor.

\subsection{Typing Judgements}\label{subsec:judgements}
To complete this section, Figure \ref{tab:judgements} gives a simple list of different typing composition judgements through which we also re-derzive the subtyping judgement to allow for its implementation.
\begin{figure}
	\begin{align*}
	\forall F \in \mF\left( \mL \right),\poulpe
	\frac{\cont x: \tau \poulpe \cont F: S\subseteq \star \poulpe \overbrace{\tau \in S}^{\exists \tau'\in S, \tau \preceq \tau'}}{\cont Fx: F\tau \preceq \tau}\fracnotate{Cons} \\[.25cm]
	\frac{\cont x: \tau \poulpe \tau \in \star_{0}}{\cont Fx: F\tau \notin \star_{0}}\fracnotate{$FT_{0}$}                                                                        \\[.25cm]
	\frac{\cont x: F\tau_{1} \poulpe \cont \phi: \tau_{1} \to \tau_{2}}{\cont \phi x: F\tau_{2} }\fracnotate{\texttt{fmap}}                                                       \\[.25cm]
	\frac{\cont x: \tau_{1} \poulpe \cont \phi: \tau_{1} \to \tau_{2}}{\cont \phi x: \tau_{2}}\fracnotate{App}                                                                    \\[.5cm]
	\frac{\cont x: A\tau_{1} \poulpe \cont \phi: A\left( \tau_{1} \to \tau_{2} \right)}{\cont \phi x: A\tau_{2}}\fracnotate{\texttt{<*>}}                                         \\[.25cm]
	\frac{\cont x: \tau}{\cont x: A\tau}\fracnotate{\texttt{pure/return}}                                                                                                         \\[.5cm]
	\frac{\cont x: MM\tau}{\cont x: M\tau}\fracnotate{\texttt{>>=}}                                                                                                               \\[.25cm]
	\forall F \overset{\theta}{\Longrightarrow} G,\poulpe \frac{\cont x: F\tau \poulpe \cont G: S' \subseteq \star \poulpe \tau \in S'}{\cont x : G\tau}\fracnotate{\texttt{nat}}
\end{align*}

	\caption{Typing and subtyping judgements for implementation of effects in the
		type system.}
	\label{tab:judgements}
\end{figure}
Note that here, the syntax is not taken into account: a function is always written left of its arguments, whether or not they are actually in that order in the sentence.

\smallskip

Using these typing rules for our semantic parsing steps, it is important to
see that our grammar will still bear ambiguity.
The next sections will explain how to reduce this ambiguity in short enough
time.

Moreover, our current typing system is not decidable, because of the
\texttt{nat/pure/return} rules which may allow for unbounded derivations.
This is not actually an issue because of considerations on handling, as
semantically void units will get removed at that time.
This leads to derivations of sentences to be of bounded height, linear in the
length of the sentence.


\section{Handling Ambiguity}
\label{sec:nondet}
The typing judgements proposed in Section \ref{subsec:judgements} lead to
ambiguity.
In this section we propose ways to get our derivations to a certain normal
form, by deriving an equivalence relation on our derivation and parsing trees,
based on string diagrams.

\subsection{String Diagram Modelisation of Sentences}
\label{subsec:sd}
String diagrams are the Poincaré duals of the usual categorical diagrams when
considered in the $2$-category of categories and functors.
This means that we represent categories as regions of the plane, functors as
lines separating regions and natural transformations as the intersection points
between two lines.

We will always consider application as applying to the right of the line so
that composition is written in the same way as in equations.
This gives us a new graphical formalism to represent our effects using a few
equality rules between diagrams.
The commutative aspect of functional diagrams is now replaced by an equality of
string diagrams, which will be detailed in the following section.

We get a way to visually see the meaning get reduced from effectful composition
to propositional values, without the need to specify what the handler does.
This delimits our usage of string diagrams as ways to look at computations and
a tool to provide equality rules to reduce ambiguity.

\begin{wrapfigure}{r}{.45\textwidth}
	\centering
		\begin{tikzpicture}
		\path coordinate[dot, label=right:$\w{the}$] (the) + (0, 1) coordinate[dot, label=left:$\w{sleeps}$] (sleeps) + (0, 2) coordinate[label=above:$\t$] (bool)
		++(-2, 1) coordinate (ctlthe) + (0, 1) coordinate[label=above:$\f{M}$] (effthe)
		++(2, -2) coordinate[dot, label=left:$\w{cat}$] (cat) + (0, -2) coordinate[label=below:$\bot$] (bot);
		\draw (cat) -- (the) -- (sleeps) -- (bool);
		\draw[name path=effect] (the) to[out=180, in=-90] (ctlthe) -- (effthe);
		\draw[dashed] (bot) -- (cat);
		\begin{pgfonlayer}{background}
			\fill[catone] (bot) rectangle ($(bool) + (1, 0)$);
			\fill[catmca] (bot) rectangle ($(effthe) + (-1, 0)$);
			\fill[catmc] (the) to [out=180, in=-90] (ctlthe) -- (effthe) -- (bool) -- (the);
		\end{pgfonlayer}
	\end{tikzpicture}

\end{wrapfigure}
Let us define the category $\mathds{1}$ with exactly one object and one arrow:
the identity on that object. It will be shown in grey in the string diagrams
below.
A functor of type $\mathds{1} \to \mC$ is equivalent to choosing an object in
$\mC$, and a natural transformation between two such functors $\tau_{1},
	\tau_{2}$ is exactly an arrow in $\mC$ of type $\tau_{1} \to \tau_{2}$.
Knowing that allows us to represent the type resulting from a sequence of
computations as a sequence of strings whose farthest right represents an object
in $\mC$, that is, a base type.

The question of providing rules to compose the string diagrams for parts of the
sentences will be discussed in the next section, as it is related to parsing.

\smallskip

In the end, we will have the need to go from a certain set of strings (the effects that applied) to a single one, through a sequence of handlers, monadic and comonadic rules and so on.
Notice that we never reference the zero-cells and that in particular their colors are purely an artistical touch.

\subsection{Achieving Normal Forms}
\label{subsec:normalforms}
We will now provide a set of rewriting rules on string diagrams (written as
equations) which define the set of different possible reductions.

First, Theorem \ref{thm:isotopy} reminds the main result by \cite{joyalGeometryTensorCalculus1991} about string diagrams which shows that our artistic representation of diagrams is correct and does not modify the equation or the rule we are presenting.
\begin{theorem}[Theorem 1.2 \cite{joyalGeometryTensorCalculus1991}]
	\label{thm:isotopy}
	A well-formed equation between morphism terms in the language of monoidal categories follows from the axioms of monoidal categories if and only if it holds, up to planar isotopy, in the graphical language.
\end{theorem}

Secondly, let us now look at a few of the equations that arise from the
commutation of certain class of diagrams:
\begin{description}
	\item[The \emph{Snake} Equations] are a rewriting of the categorical diagrams
	      which are the defining properties of an adjunction.
	\item[The \emph{(co-)Monadic} Equations] are the string diagrammatic
	      translation of the properties of unitality and associativity of the
	      monad.
	      Similarly, there are co-monadic equations which are the categorical dual
	      of the previous equations.
\end{description}
This set of equations, when added to our reduction rules from the Section
\ref{sec:parsing} explain all the different reductions that can be made to
limit non-determinism in our parsing strategies.
Indeed, considering the equivalence relation $\mathcal{R}$ freely generated
from the equations defined above and the equivalence relationship
$\mathcal{R}'$ of planar isotopy from Theorem \ref{thm:isotopy}, we get a set
of normal forms $\mathcal{N}$ from the set of all well-formed parsing diagrams
$\mD$ defined by:
$\mathcal{N} = \left( \mD / \mathcal{R} \right) / \mathcal{R}'$


\subsection{Computing Normal Forms}
Now that we have a set of rules telling us what we can and cannot do in our model while preserving the equality of the diagrams, we provide a combinatorial description of our diagrams to help compute the possible equalities between multiple reductions of a sentence meaning.

\cite{delpeuchNormalizationPlanarString2022} proposed a combinatorial description to check
in linear time for equality under Theorem \ref{thm:isotopy}.
However, this model does not suffice to account for all of our equations, especially as
labelling will influence the equations for monads, comonads and adjunctions.
To provide with more flexibility (in exchange for a bit of time complexity) we use the
description provided and change the description of inputs and outputs of each $2$-cell by
adding types and enforcing types.
In this section we formally describe the data structure we propose, as well as algorithms for
validity of diagrams and a system of rewriting that allows us to compute the normal forms
for our system of effects.

\subsubsection{Representing String Diagrams}
We follow \cite{delpeuchNormalizationPlanarString2022} in their combinatorial description
of string diagrams. We describe a diagram by an ordered set of its $2$-cells (the natural
transformations) along with the number of input strings, for each $2$-cell the following
information:
\begin{itemize}
	\item Its horizontal position: the number of strings that are right of it.
	\item Its type: an array of effects that are the inputs to the natural
	      transformation and an array of effects that are the outputs to the
	      natural transformation.
\end{itemize}
We will then write a diagram $D$ as a tuple of $3$ elements:
$\left( D.N, D.S, D.L \right)$ where $D.N$ is a positive integers representing the
height (or number of nodes) of $D$, $D.S$ is an array for the input strings of $D$ and
where $D.L$ is a function which takes a natural number smaller than $D.N - 1$ and
returns its type as a tuple of arrays
$nat = \left( \dnlg{nat}, \dnin{nat}, \dnout{nat} \right)$.
From this, we can derive a naive algorithm to check if a string diagram is
valid or not.

\smallskip

Since our representation contains strictly more information (without slowing
access by a non-constant factor) than the one it is based on, our
datastructure supports the linear and polynomial time algorithms proposed with
the structure by \cite{delpeuchNormalizationPlanarString2022}.
In particular our structure can be normalized in time
$\O\left( n \times \sup_{i} \abs{\dnin{\dlb{D}{i}}} + \abs{\dnout{\dlb{D}{i}}}
	\right)$, which depends on our lexicon but most of the times will be linear
time.

\subsubsection{Equational Reductions}
We are faced a problem when computing reductions using the equations for our diagrams
which is that by definition, an equation is symmetric.
To solve this issue, we only use equations from left to right to reduce as much as
possible our result instead.
This also means that trivially our reduction system computes normal forms: it suffices
to re-apply the algorithm for recumbent equivalence after the rest of equational
reduction is done.
Moreover, note that all our reductions are either incompatible or commutative, which
leads to a confluent reduction system, and the well definition of our normal forms:
\begin{theorem}[Confluence]\label{thm:confluence}
	Our reduction system is confluent and therefore defines normal forms:
	\begin{enumerate}
		\item Right reductions are confluent and therefore define \emph{right} normal forms for
		      diagrams under the equivalence relation induced by exchange.
		\item Equational reductions are confluent and therefore define \emph{equational}
		      normal forms for diagrams under the equivalence relation induced by exchange.
	\end{enumerate}
\end{theorem}

Before proving the theorem, let us first provide the reductions for the different
equations for our description of string diagrams.
\begin{description}
	\item[The Snake Equations]
	      First, let's see when we can apply the equation for $\id_{L}$ to a
	      diagram $D$ which is in \emph{right} normal form, meaning it's been
	      right reduced as much as possible.
	      Suppose we have an adjunction $L \dashv R$.
	      Then we can reduce $D$ along the equation at $i$ if, and only if:
	      \begin{itemize}
		      \item $\dnlg{\dlb{D}{i}} = \dnlg{\dlb{D}{i + 1}} - 1$
		      \item $\dlb{D}{i} = \eta_{L, R}$
		      \item $\dlb{D}{i + 1} = \epsilon_{L, R}$
	      \end{itemize}
	      This comes from the fact that we can't send either $\epsilon$
	      above $\eta$ using right reductions and
	      that there cannot be any natural transformations between the two.
	      Obviously the equation for $\id_{R}$ works the same.
	      Then, the reduction is easy: we simply delete both strings,
	      removing $i$ and $i + 1$ from $D$ and reindexing the other nodes.
	\item[The Monadic Equations] For the monadic equations, we only use
	      the unitality equation as a way to reduce the number of natural
	      transformations, since the goal here is to attain normal forms
	      and not find all possible reductions.
	      We ask that associativity is always used in the direct
	      sense $\mu\left( \mu\left( TT \right),T \right) \to \mu\left(
		      T\mu\left( TT \right) \right)$ so that the algorithm terminates.
	      We use the same convention for the comonadic equations.
	      The validity conditions are as easy to define for the monadic
	      equations as for the \emph{snake} equations when considering
	      diagrams in \emph{right} normal forms.
	      Then, for unitality we simply delete the nodes
	      and for associativity we switch the horizontal
	      positions for $i$ and $i + 1$.
\end{description}

\begin{proof}[Proof of the Confluence Theorem]
	The first point of this theorem is exactly Theorem 4.2
	in \cite{delpeuchNormalizationPlanarString2022}.
	To prove the second part, note that the reduction process terminates as
	we strictly decrease the number of $2$-cells with each reduction.
	Moreover, our claim that the reduction process is confluent is obvious
	from the associativity equation and the fact the other
	equations delete nodes.
	Since right reductions do not modify the equational reductions, and thus
	right reducing an equational normal form yields an equational normal form,
	combining the two systems is done without issue, completing our proof of
	Theorem \ref{thm:confluence}.
\end{proof}

\begin{theorem}[Normalization Complexity]
	\label{thm:normalize}
	Reducing a diagram to its normal form is done in polynomial time in
	the number of natural transformations in it.
\end{theorem}
\begin{proof}
	Let's now give an upper bound on the number of reductions.
	Since each reductions either reduces the number of $2$-cells or applies the
	associativity of a monad, we know the number of reductions is linear in the
	number of natural transformations.
	Moreover, since checking if a reduction is possible at height $i$ is done in
	constant time, checking if a reduction is possible at a step is done in
	linear time, rendering the reduction algorithm quadratic in the number of
	natural transformations.
	Since we need to \emph{right} normalize before and after this method, and
	that this is done in linear time, our theorem is proved.
\end{proof}



\section{Efficient Semantic Parsing}
\label{sec:parsing}
In this section we explain our algorithms and heuristics for efficient semantic
parsing with as little non-determinism as possible, and reducing time complexity
of our parsing strategies.

\subsection{Naïve Semantic Parsing on Syntactic Parsing}
In this section we suppose that we have a set of syntax trees (or parsing trees) corresponding to a certain
sentence.
We will now focus on how to derive proofs of typing from a syntax tree.
First, note that \cite{bumfordEffectdrivenInterpretationFunctors2025} provides a way to do so by constructing
semantic trees labelled by sequence of \emph{combinators}.
In our formalism, this amounts to constructing proof trees by mapping combination modes to their equivalent proof
trees, inverting if needed the order of the presuppositions to account for the order of the words.
Computing one tree is easily done in linear time in the number of nodes in the parsing tree (which is linear in the input size,
more on that in the next section), multiplied by a constant whose size depends on the size of the inference
structure.
The main idea is that to each node of the tree, both nodes have a type and there is only a finite set of rules
that can be applied, provided by the following rules, which are a rewriting of Figure \ref{tab:judgements}.
In Figure \ref{tab:proof-trees} we provide \emph{matching-like} rules for different possibilities on the types to
combine and the associated possible proof tree(s).
Note that there is no condition on what the types \emph{look like}, they can be effectful.
If multiple cases are possible, all different possible proof trees should be added to the set of derivations: the
set of proof trees for the union of cases is the union of set of proof trees for each case, naturally:
\begin{equation*}
	PT\left( \cup C \right) = \cup PT(C)
\end{equation*}
Remember again that the order of presuppositions for an inference structure does not matter, so always we have the
other order of arguments available.

\begin{figure}
	\centering
	\resizebox{\linewidth}{!}{%
  \def\arraystretch{2}
	\begin{NiceTabular}{CCC>{$}c<{$}}[hvlines]
		                             & S_{1}                                  & S_{2}                & PT(S_{1}, S_{2})              \\
		\Block{2-1}{\rm Base Types}  & l: \ta \to \tb                         & r: \ta               & \Japp{l}{\ta}{\tb}{r}         \\
		                             & l: \ta \to \t                          & r: \ta \to \t        & \Jconj{l}{r}{\ta}             \\
		\Block{1-1}{\rm Functors}    & \cont l: \f{F}\ta                      & \cont r: \ta \to \tb & \Jfmap{\f{F}}{l}{\ta}{r}{\tb} \\
		\Block{1-1}{\rm Applicative} & \cont l: \f{F}\left(\ta \to \tb\right) & \cont r: \ta         & \Junit{\f{F}}{l}{\ta}{r}{\tb}
	\end{NiceTabular}
}

	\caption{List of possible combinations for different presuppositions for inputs, as a definition of a function
		$PT$ from proof trees to proof trees.}
	\label{tab:proof-trees}
\end{figure}

It is important to note that here the proof trees are done ``in the wrong direction'', as they are written
top-down instead of bottom-up in the sense that we start by the axioms which are the typing judgements of
constants in the lexicon (this could actually be said to be a part of the sentence, but non-determinism in the
meanings is fine as long as the syntactic category is provided in the parsing tree).
This leads to proof trees being a bit weird to the type theorist and a bit weird to the linguist as they are
not written as usual trees.
Moreover, while a proof tree only provides a type, the tree also provides the sequence of function applications
that is needed to compute the actual denotation.

This leads the induced algorithm (for computing the set of denotations) to be exponential in the input in the
worst case, when using the recursive scheme that naturally arises from the definition of $PT$.
Indeed, and while this may seem weird, in the case of a function $\f{F}\ta \to \tb$ where $\f{F}$ is applicative
and the other combinator is of type $\f{F}\ta$, one might apply the function directly or apply it under the $\f{F}$
of the argument, leading to two different proof trees:
\begin{center}
	\begin{align*}
		\resizebox{\textwidth}{!}{$\prooffrac{\cont \f{F}: \mC \Rightarrow \mC \poulpe \Junit{\f{F}}{l}{\ta}{r}{\tb}}{\cont \f{F}lr: \f{F}\tb}$} \\[1em]
		\Japp{l}{\f{F}\ta}{\tb}{r}
	\end{align*}
\end{center}
Those two different proof trees have different semantic interpretations that may be useful, as discussed by
\newcite{bumfordEffectdrivenInterpretationFunctors2025}, especially in their analysis of closure properties, which
is modified in the following section.

\subsubsection{Islands}
\newcite{bumfordEffectdrivenInterpretationFunctors2025} provide a formal analysis of islands\footnote{Syntactic
	structures which prevent some notion of moving outside of the island.} based on a islands being a different type
of branching nodes in the syntactic tree of a sentence, which asks to resolve all $\f{C}$ effects\footnote{Those
	represent continuations.} before that node or being resolved at that node.
The main example they propose is the one of existential closure inside conditioning.
To reconcile this inside our type system, we propose the following change to their formalism: once the syntactic
information of an island existing is added to the tree (or at semantic parsing time, this does not change the
time complexity), we \emph{mark} each node inside the island by adding an ``void effect'' to it, in the same
way as we did for our model of plural.
This translates into a functor which just maintains the \emph{island marker} on \texttt{fmap} and add to the
words which create an island node\footnote{Or \emph{post-compose} the proof trees with a rule on handling the
	$\f{C}$ effect.} a function which handles the $\f{C}$ effects, which could be seen as adding a node in the
semantics parse tree from the syntactic parse tree.
A way to do this would be the following: we pass all functors until finding a $\f{C}$, handle it inside the other
functors and keep going, on both sides, where $\mathrm{PT}$ is defined in Figure \ref{tab:proof-trees}:
\begin{equation*}
	\mathrm{PT}\left(
	\suppfrac{\raisebox{-5pt}{$\cont x_{1}: \f{F}\f{C}\f{F'}\tau$}}{\cont x_{1}: \f{F}\f{F'}\tau}\suchthat
	\suppfrac{\raisebox{-5pt}{$\cont x_{2}: \f{F}\f{C}\f{F'}\tau$}}{\cont x_{2}: \f{F}\f{F'}\tau}
	\right)
\end{equation*}
Note that this preserves the linear size of the parse tree in the number of input words, as we at most double the
number of nodes, and note that this would be preserved with additional similar constructs.

\medskip

This idea amounts to seeing islands, whatever their type, as a form of grammatical/syntactic effect, which is
still part of the language but not of the lexicon, a bit like future, modality or plurals, without the semantic
component.
The idea of seeing everything as effects, while semantically void, allows us to translate into our theory of
type-driven semantics outer information from syntax, morphology or phonology that influences the semantics of the
sentence.
Other examples of this can be found in the modelisation of plural (for morphology, see Sections \ref{par:higherorder}
and \ref{subsec:modality}) and the emphasis of the words by the $\f{F}$ effect (for phonology), and show the
empirical well-foundedness of this point of view.
While we do not aim to provide a theory of everything linguistics through categories\footnote{``\emph{A theory is as
		large as its most well understood examples}''.}, the idea of expressing everything in our effect-driven type-driven
theory of semantics allows us to prepare for any theoretical or empirical observation that has an impact on the
semantics of a word/sentence, the allowed combinators and even the addition of rules.

\subsection{Syntactic-Semantic Parsing}
\label{subsec:ssparsing}
\subsubsection{The Improved Method}
If using a naïve strategy on the trees yields an exponential algorithm,
the best way to improve the algorithm, is to directly leave the naïve strategy.
To do so, we could simply extend the grammar system used to do the syntactic
part of the parsing.
In this section, we will take the example of a CFG since it suffices to create
our typing combinators,
For example, we can increase the alphabet to be the free monoid on the product
$\Sigma$ of the usual base alphabet and $\bar{\mC}$ our typing category.
Since the usual syntactic structure and our combination modes can both be
expressed by CFGs (or almost, cf Figure \ref{fig:combination-cfg}), our product
system can be expressed by (almost) a CFG.
Here, we change our notations from the proof trees of Figure
\ref{tab:proof-trees} to have a more explicit grammar of combination modes
provided below is a rewriting of the proof trees provided earlier, and
highlights the combination modes in a simpler way, based on
\cite{bumfordEffectdrivenInterpretationFunctors2025} as it simplifies the
rewriting of our typing judgements in a CFG\footnote{That, in a sense, was
	already implicitly provided in Figure \ref{tab:judgements}.}.
The grammars provided  in Figures \ref{fig:english-cfg} and \ref{fig:combination-cfg}
are actually used from right to left, as we actually do combinations over the
types and syntactic categories of the words and try to reduce the sentence, not
derive it from nothing.
Now, all the improvements discussed in Section \ref{subsec:rewrite}
can still be applied here, just a bit differently, as this amounts to reducing
ambiguity in the grammar.

\begin{figure}
	\centering
	{\centering
\setlength{\columnsep}{-1.5cm}
\begin{multicols}{2}
	\def\arraystretch{1.2}
	\begin{mgrammar}
		\gskip
		\firstrule{>, b}{\left(a\to b\right), a}{}
		\firstrule{<, b}{a, \left(a \to b\right)}{}
		\firstrule{\wedge, a \to \t}{\left(a \to \t\right), \left(a \to \t\right)}{}
		\firstrule{\vee, a \to \t}{\left(a \to \t\right), \left(a \to \t\right)}{}
		\gskip
		\firstrule{\combJ_{\f{F}}\  \f{F}\tau}{\f{F}\f{F}\tau}{}
		\lfrule{\combDN_{\f{C}}\  \tau}{\f{C}_{\tau}\tau}{}
	\end{mgrammar}

	\def\arraystretch{1.3}
	\begin{mgrammar}
		\firstrule{\combML_{\f{F}} \left(\alpha, \beta\right)}{\f{F}\alpha, \beta}{}
		\firstrule{\combMR_{\f{F}} \left(\alpha, \beta\right)}{\alpha, \f{F}\beta}{}
		\firstrule{\combA_{\f{F}} \left(\alpha, \beta\right)}{\f{F}\alpha, \f{F}\beta}{}
		\firstrule{\combUR_{\f{F}} \left(\alpha \to \alpha', \beta\right)}{\f{F}\alpha\to \alpha', \beta}{}
		\firstrule{\combUL_{\f{F}} \left(\alpha, \beta\to \beta'\right)}{\alpha, \f{F}\beta \to \beta'}{}
		\firstrule{\combC_{\f{L}\f{R}} \left(\f{L} \alpha, \f{R}\beta\right)}{\left(\alpha, \beta\right)}{}
		\firstrule{\combER_{\f{R}} \left(\f{R}\left(\alpha \to \alpha'\right), \beta\right)}{\alpha\to \f{R}\alpha', \beta}{}
		\lfrule{\combEL_{\f{R}} \left(\alpha, \f{R}\left(\beta \to \beta'\right)\right)}{\alpha, \beta\to \f{R}\beta'}{}
	\end{mgrammar}
\end{multicols}
}

	\caption{Possible Type Combinations in the form of a near CFG}
	\label{fig:combination-cfg}
\end{figure}

This grammar works in five major sections:
\begin{enumerate}
	\item We reintroduce the grammar defining the type and effect system.
	\item We introduce a structure for the semantic parse trees and their labels,
	      the combination modes from
	      \newcite{bumfordEffectdrivenInterpretationFunctors2025}.
	\item We introduce rules for basic type combinations.
	\item We introduce rules for higher-order unary type combinators.
	\item We introduce rules for higher-order binary type combinators.
\end{enumerate}
The idea of the \emph{grammatical} reduction is that from the flat sentence,
we create a full parse tree as a sequence of types $\tau$.
We then reduce it using the binary effect combinators, before choosing the
appropriate binary type combinator.
It is at this point in the reduction we actually do the composition
of functions.
We close up the reduction\footnote{For the combination of two words, which is
	then repeated along the syntactic structure.} by possibly using unary effect
combinators.

\medskip

We do not prove here that these typing rules cannot be written in a simpler
syntactic structure\footnote{It is important to note that for a typing system,
	we only have syntactic-like rules that allow, or not, to combine types.}, but
it is easy to see why a regular expression would not work, and since we need
trees, to express our full system, the best we can do would be to disambiguate
the grammar.
A thing to note is that this grammar is not complete but explains how types can
be combined in a compact form.
For example, the rules of the form $\combML_{\f{F}} M, \tau' \gets M, \tau$ are
rules that provide ways to combine effects from the two inputs in the order we
want to: we can combine $\f{R}\f{S}\e$ and $\f{C}\f{W}(\e \to \t)$ into
$\f{R}\f{C}\f{W}\f{S}\t$ with the mode $\combML\combMR\combMR\combML <$ (see
\cite{bumfordEffectdrivenInterpretationFunctors2025} Example 5.14).

\begin{figure}
	\centering
	\def\arraystretch{1.5}
% f · x = (\fmap_{\f{F}} f) (x)
% \eta = pure 
\begin{multicols}{2}
	$	>                    = \lambda \phi. \lambda x. \phi x $\\[1.5ex]
	$ <                    = \lambda x. \lambda \phi. \phi x $\\[1.5ex]
	$ \combML_{\f{F}}      = \lambda M. \lambda x. \lambda y. (\fmap_{\f{F}} \lambda a. M(a, y)) x $\\[1.5ex]
	$ \combMR_{\f{F}}      = \lambda M. \lambda x. \lambda y. (\fmap_{\f{F}} \lambda b. M(x, b)) y $\\[1.5ex]
	$	\combA_{\f{F}}       = \lambda M. \lambda x. \lambda y. (\fmap_{\f{F}}\lambda a. \lambda b. M(a, b))(x) \texttt{<*>} y $\\[1.5ex]
	$	\combUL_{\f{F}}      = \lambda M. \lambda x. \lambda \phi. M(x, \lambda b. \phi(\eta_{\f{F}} b))$\\[1.5ex]
	$	\combUR_{\f{F}}      = \lambda M. \lambda \phi. \lambda y. M(\lambda a. \phi(\eta_{\f{F}} a),y) $\\[1.5ex]
	$ \combJ_{\f{F}}       = \lambda M. \lambda x. \lambda y. \mu_{\f{F}} M(x, y) $\\[1.5ex]
	$	\combC_{\f{L}\f{R}}  = \lambda M. \lambda x. \lambda y. \epsilon_{\f{L}\f{R}}(\fmap_{\f{L}}(\lambda l. \fmap_{\f{R}}(\lambda r. M(l, r))(y)) (x)) $\\[1.5ex]
	$	\combEL_{\f{R}}      = \lambda M. \lambda \phi. \lambda y. M(\Upsilon_{\f{R}} \phi, y)$\\[1.5ex]
	$	\combER_{\f{R}}      = \lambda M. \lambda x. \lambda \phi. M(x, \Upsilon_{\f{R}} \phi)$\\[1.5ex]
	$	\combDN_{\Downarrow} = \lambda M. \lambda x. \lambda y. \Downarrow M(x, y)$
\end{multicols}

	\caption{Denotations describing the effect of the combinators used in the
		grammar describing our combination modes presented in
		Figure \ref{fig:combination-cfg}}
	\label{fig:combinator-denotations}
\end{figure}

Each of these combinators can be, up to order, associated with a inference
rule, and, as such, with a higher-order denotation, which explains the actual
effect\footnote{Pun intended} of the combinator, and are described in
Figure \ref{fig:combinator-denotations}.
The main reason we need to get derivations associated to combinators, is to
properly define equivalence and reduce the number of rules in the grammar.
Explanation on how that is done will come in Section \ref{subsec:rewrite}.
The important thing on those derivation is that they're a direct translation
of the rules defining the notions of functors, applicatives, monads and thus
are not specific to any denotation system, even though we will use
lambda-calculus styled denotations to describe them.
The same structure for combinators apply when describing the combinators
than when describing their rules of existence.

\medskip

Moreover, it is important to note that while we talk about the structure in
Figure \ref{fig:combination-cfg} as a grammar, it is more of a structure that
computes if it is possible or not to combine two types and how.
In particular, our grammar is not finite, since there are infinitely many types
that can be playing the part of $\alpha$ and $\beta$, and infinitely many
functors that can play the roles for $\f{F}$ and so on, but that does not pose
a problem, as recognizing the general form of a type can be done in constant
time for the forms that we want to check, given a proper memory representation.
Furthermore, it can easily be rendered into something that looks like a
Chomsky Normal Form for a CFG and allows us to integrate this in the CYK
algorithm, and even get a correct time complexity.
Since our grammar is not actually finite, the modified version of CYK that
parses only the types (not the full system) will have a complexity in the
size of $\mFunc\left(\mL\right)$ that is linear, albeit with a not so
small constant factor, which comes from the fact that our grammar's size is
linear in $\abs{\mFunc\left(\mL\right)}$.

\begin{thm}
	\label{thm:ptime-parse}
	Semantic parsing of a sentence is polynomial in the length of the	sentence
	and the size of the type system and syntax system.
\end{thm}
\begin{proof}
	Suppose we are given a syntactic generating structure $G_{s}$ along with our
	type combination grammar $G_{\tau}$.
	The system $G$ that can be constructed from the (tensor) product of $G_{s}$ and
	$G_{\tau}$ has size $\abs{G_{s}}\times \abs{G_{\tau}}$.
	Indeed, we think of its rules as a rule for the syntactic categories and a rule
	for the type combinations. Its terminals and non terminals are also in the
	cartesian products of the adequate sets in the original grammars.
	What this in turn means, is that if we can syntactically parse a sentence in
	polynomial time in the size of the input and in $\abs{G_{s}}$, then we can
	syntactico-semantically parse it in polynomial time in the size of the input,
	$\abs{G_{s}}$ and $\abs{G_{\tau}}$.
\end{proof}

While we have gone with the hypothesis that we have a CFG for our language,
any type of polynomial-time structure could work\footnote{Simply, the algorithm
	will need to be	adapted to at least process the CFG for the typing rules.},
as long as it is more expressive than a CFG.
We will do the following analysis using a CFG since it allows to cover enough of
the language for our example and for simplicity of describing the process of
adding the typing CFG, even though some think that English is not a context
free language \cite{higginbothamEnglishNotContextFree1984}.
Since the CYK algorithm provides us with an algorithm for CFG based parsing
cubic in the input and linear in the grammar size, we end up with an algorithm
for semantically parsing a sentence that is cubic in the length of the sentence
and linear in the size of the grammar modeling the language and the type system.

\begin{thm}
	\label{thm:ptime-denot}
	Retrieving a pure denotation for a sentence can be done in polynomial time in
	the length of the sentence, given a polynomial time syntactic parsing
	algorithm and polynomial time combinators.
\end{thm}
\begin{proof}
	We have proved in Theorem \ref{thm:ptime-parse} that we can retrieve a
	semantic parse tree from a	sentence in polynomial time in the input.
	Since we also have shown that the length of a semantic parse tree is quadratic
	in the length of the sentence it represents, being linear in the length of a
	syntactic parse tree linear in the length of the sentence.
	We have already seen that given a denotation, handling all effects and
	reducing effect handling to normal forms can be done in polynomial time.
	The superposition of these steps yields a polynomial-time algorithm in the
	length of the input sentence.
\end{proof}

The \emph{polynomial time combinators} assumption is not a complex assumption,
this is for example true for our denotations in Section \ref{sec:language},
with function application being linear in the number of uses of variables in
the function term, which in turn is linear in the number of terms used to
construct the function term and thus of words, and \fmap being in constant
time\footnote{Depending on the functor but still bounded by a constant.} for
the same reason.
Of course, function application could be said to be in constant time too, but
we consider here the duration of the $\beta$-reduction:
\begin{equation*}
	\left(\lambda x. M\right)N \xrightarrow{\beta} M\left[x / N\right]
\end{equation*}

\subsubsection{Diagrammatical Parsing}
\label{subsubsec:diagram-parsing}
When considering \newcite{coeckeMathematicalFoundationsCompositional2010}
way of using string diagrams for syntactic parsing/reductions, we can see them
as (yet) another way of writing our parsing rules.
In our typed category, we can make see our combinators as natural
transformations ($2$-cells): then we can see the different sets of combinators
as different arity natural transformations.
$>$, $\combML_{\f{F}}$ and $\combJ_{\f{F}}$ are represented in
Figure \ref{fig:combinator-sd}, up to the coloring of the regions, because that
is purely for an artistic rendition\footnote{Colours make this less boring,
	trust me}.

\begin{figure}
	\centering
	\inputtikz{combinators-sd}
	\caption{String Diagramatic Representation of Combinator Modes}
	\label{fig:combinator-sd}
\end{figure}

Now, a way to see this would be to think of this as an orthogonal set of
diagrams to the ones of Section \ref{sec:nondet}: we could use the syntactic
version of the diagrams to model our parsing, according to the rules in
Figure \ref{fig:combination-cfg}, and then combine the diagrams as in Equation
\ref{eq:CSD}, as shown in Figure \ref{fig:parsing-diagram}, which highlights
why we can consider the diagrams orthogonals.
The obvious reaction\footnote{That I'm sure everyone have had seeing this.} is:
"\emph{Why did we start by saying we could just paste?}".
This is a perfectly valid point, and pasting diagrams is simply a notation
abuse, that is justified by the fact there is a possible reduction.
In this figure we exactly see the sequence of reductions play out on the types
of the words, and thus we also see what exact \emph{quasi-braiding} would be
needed to construct the effect diagram.
Here we talk about \emph{quasi-braiding} because, in a sense, we use $2$-cells
to do braiding-like\footnote{Permutations on the order of the strings}
operations on the strings, and don't actually allow for braiding inside the
diagrammatic computation.

\begin{figure}
	\centering
	\inputtikz{parsing-diagram}
	\caption{Representation of a parsing (sub-)diagram for the sentence
		\emph{the cat eats a mouse}, including the connection between the effect
		handling diagrams from Section \ref{sec:nondet} and the syntactio-semantic
		parsing diagrams formed combining our approach to parsing and
		\cite{coeckeMathematicalFoundationsCompositional2010} approach to diagrams}
	\label{fig:parsing-diagram}
\end{figure}

Categorically, what this means is that we start from a meaning category $\mC$,
our typing category, and take it as our grammatical category.
This is a form of extension on the monoidal version by
\newcite{coeckeMathematicalFoundationsCompositional2010}, as it is seemingly a
typed version, where we change the Pregroup category for the typing category,
possibly taken with a product for representation of the English grammar
representation, to accomodate for syntactic typing on top of semantic typing.
This is, again, just another rewriting of our typing rules.

More formally, we have a first axis of string diagrams in the category
$\mC$ - our string diagrams for effect handling, as in Section
\ref{sec:nondet} - and a second \emph{orthogonal} axis of string diagrams
on a larger category, with endofunctors\footnote{To ensure proper typing of the
	diagrams.} graded by the types in our typing category $\bar{\mC}$ and
with natural transformations mapping the combinators defined in Figures
\ref{fig:combination-cfg} and \ref{fig:combinator-denotations}.
The category in which we consider the second-axis string diagrams does not have
a meaning in our compositional semantics theory, and to be more precise, we
should talk about $1$-cells and $2$-cells instead of functors and natural
transformations, to keep in the idea that this is really just a diagrammatic
way of computing and presenting the operations that are put to work during
semantic parsing.

\medskip

What the lines leading from combinators to functors\footnote{Remember that
	types are objects in $\mC$ and thus functors from
	$\mathds{1} \Rightarrow \mC$} mean categorically, is void in either category.
Those lines are not actually part of the first axis of the string diagram,
nor are they part of the second axis of the string diagram.
They are used to map out the link between the two: they express the
quasi-braiding step proposed above, and present graphically why the order of
the strings in the resulting diagram is as it is, and what the pasting and
quasi-braiding orders should be.
A good approach to these diagrams might be the following: they are an extension
of the parse trees presented in
\newcite{bumfordEffectdrivenInterpretationFunctors2025}, applied to the
formalism of \newcite{coeckeMathematicalFoundationsCompositional2010} and
extended to be integrated with the handling of effects,
as described in Section \ref{sec:nondet}, forming a full diagrammatic calculus
system for semantic parsing.

\medskip

For the combinators $\combJ$, $\combDN$ and $\combC$, which are applied to
reduce the number of effects inside a denotation, it might seem less obvious
how to include them.
While this may seem true, it's actually only true for the \emph{connecting}
strings.
Indeed, applying them to the actual \emph{parsing} part of the diagram is done
in the exact same way as in the CFG, we just apply them when needed, and they
will appear in the resulting denotation as an end for a string, a form of
forced handling, in a sense, as shown in the result of Figure
\ref{fig:parsing-diagram2}.
For the connecting strings, it's simply a matter of adding a new \emph{phantom}
string that will send the associated $2$-cell in the effect handling diagram to
the connected strings.
In particular it is interesting to note that the resulting diagram representing
the sentence can, in a way, be found in the connection strings that arise from
the combinators.

\begin{figure}
	\centering
	\tikzset{
	a/.pic={%
			\path coordinate[dot, label=right:$\w{a}$] (a) + (0, 2) coordinate[label=above:$\e$] (bool) + (0, -1) coordinate (input)
			++ (-1, 1) coordinate (ctla) + (0, 1) coordinate[label=above:$\f{D}$] (effa);
			\node[anchor=north] at (input) {$\e \to \t$};
			\draw (input) -- (bool);
			\draw (a) to[out=180, in=-90] (ctla) -- (effa);
			\coordinate (-north) at (bool);
			\coordinate (-north east) at ($(bool) + (1, 0)$);
			\coordinate (-east) at ($(a) + (1, 0)$);
			\coordinate (-south east) at ($(input) + (1, 0)$);
			\begin{pgfonlayer}{background}
				\fill[catone] (input) rectangle ($(bool) + (1, 0)$);
				\fill[catmca] (input) rectangle ($(effa) + (-1, 0)$);
				\fill[catmc] (a) to [out=180, in=-90] (ctla) -- (effa) -- (bool) -- (a);
			\end{pgfonlayer}
		},
	cat/.pic={
			\path coordinate[dot, label=right:$\w{in\ a\ box}$] (box)
			+ (-1, 1) coordinate (ctlbox)
			+ (-1, 2) coordinate[label=above:$\f{D}$] (effbox)
			+ (0, -1) coordinate[label=below:$\bot$] (input)
			++ (0, 1) coordinate[label=right:$\w{cat} \land$, dot]
			+ (0, 1) coordinate[label=above:$\e\to \t$] (out);
			\draw[dashed] (input) -- (box);
			\draw (box) to[out=180, in=-90] (ctlbox) -- (effbox)
			(box) -- (out);
			\coordinate (-north) at (out);
			\coordinate (-north east) at ($(out) + (2, 0)$);
			\coordinate (-east) at ($(box) + (2, 0)$);
			\coordinate (-south east) at ($(input) + (2, 0)$);
			\begin{pgfonlayer}{background}
				\fill[catone] (input) rectangle ($(out) + (2, 0)$);
				\fill[catmca] (input) rectangle ($(effbox) + (-1, 0)$);
				\fill[catmc] (box) to[out=180, in=-90] (ctlbox) |- (out) -- (box);
			\end{pgfonlayer}
		},
	res/.pic={
			\path coordinate[dot, label=right:$\w{cat} \land$] (rcat)
			++ (0, 1) coordinate[dot, label=right:$\w{a}$] (ra)
			+ (0, 3) coordinate[label=above:$\e$] (rout)
			+ (-1, 1) coordinate[label=right:$\f{D}$] (rctla)
			++ (0, -2) coordinate[dot, label=right:$\w{in\ a\ box}$] (rbox)
			+ (0, -1) coordinate[label=below:$\bot$] (rbot)
			++ (-2, 2) coordinate (rctlb)
			++ (0, 1) coordinate[label=left:$\f{D}$] (rctlbb)
			++ (.5, 1) coordinate[dot, label=below:$\mu_{\f{D}}$] (rmu)
			+ (0, 1) coordinate[label=above:$\f{D}$] (reff);
			\draw[dashed] (rbot) -- (rbox);
			\draw (rout) -- (rbox)
			(rbox) to[out=180, in=-90] (rctlb)
			-- (rctlbb)
			to[out=90, in=180] (rmu)
			to[out=0, in=90] (rctla)
			to[out=-90, in=180] (ra)
			(rmu) -- (reff);
			\begin{pgfonlayer}{background}
				\fill[catone] (rbot) rectangle ($(rout) + (2, 0)$);
				\fill[catmca] (rbot) rectangle ($(reff) + (-1.5, 0)$);
				\fill[catmc] (rbox) to[out=180, in=-90] (rctlb) -- (ra) -- (rbox);
				\fill[catmc] (rctla) |- (ra) to[out=180, in=-90] (rctla);
				\fill[catmc] (rctla) rectangle (rctlb);
				\fill[catmc] (rctlbb) to[out=90, in=180] (rmu) |- (rctlbb);
				\fill[catmc] (rmu) to[out=0, in=90] (rctla) -| (rmu);
			\end{pgfonlayer}
		}
}

\begin{tikzpicture}
	\pic (nbox) at (0, 0) {cat};
	\pic (na) at ($(nbox-north) + (0, 2.5)$) {a};
	\node[rectangle, draw=black] (db) at ($(nbox-east) + (1, 0)$) {$\f{D}(\e \to \t)$};
	\node[rectangle, draw=black] (da) at ($(na-east) + (1.5, 0)$) {$(\e \to \t) \to \f{D}\e$};
	\node[rectangle, draw=black] (fmap) at ($(nbox-north east) + (3, 1)$) {$\combML_{\f{M}}$};
	\path (fmap)
	+ (1.5, 1) coordinate[label=above:$(\e \to \t) \to \f{D}\e$] (ffu)
	+ (1.5, -1) coordinate[label=below:$\e \to \t$] (ffd)
	+ (0, 3) coordinate (tfm)
	++ (3, 0) node[rectangle, draw=black] (app) {$\rotatebox{90}{<}$}
	+ (1, 0) coordinate[label=below:$\f{D}\e$] (fad)
	+ (0, 3) coordinate (tfd)
	++ (2, 0) node[rectangle, draw=black] (join) {$\combJ_{\f{D}}$}
	+ (0, 3) coordinate (tft)
	+ (0, 5) coordinate[label=above:$\e$] (end);
	\begin{scope}
		\draw[postaction={on each segment={mid arrow={black}}}, thick, vulm]
		(db) to[out=0, in=240] (fmap)
		(da) to[out=0, in=120] (fmap)
		(fmap) to[out=60, in=180] (ffu)
		(fmap) to[out=-60, in=180] (ffd)
		(ffu) to[out=0, in=120] (app)
		(ffd) to[out=0, in=240] (app)
		(app) -- (join);
	\end{scope}
	%
	\node[circle, draw=black] (mu) at ($($(tfm)!0.5!(tfd)$) + (0, 1)$) {$\mu_{\f{D}}$};
	\coordinate[label=above:$\f{D}$] (ff) at ($(mu) + (0, 1)$);
	\draw[dashed] (fmap) -- (tfm)
	(app) -- (tfd)
	(join) -- (tft)
	(tfm) to[out=90, in=180] (mu)
	(tfd) to[out=90, in=0] (mu);
	\draw[dotted, thick, postaction={on each segment={mid arrow=black}}] ($(tft) + (0, -1)$) to[out=90, in=-90] (mu);
	\draw	(tft) -- (end) (mu) -- (ff);
	%
	\node (imp) at ($(join) + (.5, 0)$) {$\Rightarrow$};
	\pic (nres) at ($(imp.east) + (3,0)$) {res};
	\draw[->] let
	\p{end} = ($(end) + (4.5, 0)$),
	\p{bond} = ($(ncat-ll) + (1, -1)$),
	in (\x{bond}, \y{bond}) -- node[below] {Direction of Reduction} (\x{end}, \y{bond});
\end{tikzpicture}

	\caption{Example of a parsing (sub-)diagram for the phrase
		\emph{a cat in a box}, presenting the integration of unary combinators
		inside the connector line.}
	\label{fig:parsing-diagram2}
\end{figure}

The main reason why this point of view of diagrammatic parsing is useful
will be clear when looking at the rewriting rules and the normal forms they
induce, because, as seen before, string diagrams make it easy to compute
normal forms, when provided with a confluent reduction system.

\subsubsection{The Coproduct Categorical Approach to Syntax}
\label{subsubsec:coprod}
% TODO: add a reference to Bella's paper explaining the interest of such 
% notions when talking about the syntax-semantic interface/
In this section based on the work by
\newcite{marcollimatildeetchomskynoametberwickrobertc.MathematicalStructureSyntactic},
we explore the integration of our notion inside the structure of syntactic
merge, and why the two formalisms are compatible.
The idea behind that structure is to see the union of trees as a product and
the merging of trees as a coproduct in a well-defined Hopf algebra.
Now of course, with our insights on syntactico-semantic parsing, we can either
think of our parsing structure labeled by multiple modes, as in
\cite{bumfordEffectdrivenInterpretationFunctors2025}, or think of it as string
diagrams, as presented above.
In both cases, we make a more or less implicit use of the merge operation.

Indeed, using trees, we get back to the graded non-associative commutative
magma operation proposed in the coproduct approach.
While using string diagrams, we have a certain set of objects on which we can
define a product - the tensor product of diagrams, that is, the monoidal
operation of the category used to consider the diagrams - and a coproduct.
Defining the coproduct is a bit trickier so we will define it \emph{with our
	hands} in an intuitive manner.
The coproduct on the trees simply connects to the merging of trees, so in our
case it should only connect two different bits of our string diagrams together.
This means that our coproduct should be adding a base type combinator but that
doesn't suffice to fully express our typed semantics.
The question can be thus rephrased as follows: how to get a functorial
correspondance between the labeled trees and our string diagrams that behaves
well with regards to the product and coproduct on those objects?

\medskip

However, since our diagrams are a tool that can be seen \emph{on top} of the
usual parsing trees, answering such complex questions might seem a stretch for
an unaware linguist.
Indeed, what such a point of view would bring is void: our system does not
add any new theoretical concept to the theoretical explanation of the parsing
tree approach, and is simply a more visual explanation of the process, as well
as an explanation of it inside the categorical framework we used to develop the
type and effect system for a natural language.
The interest of such a connection is found in the sheer efficient beauty of a
commutative diagram: we express that - however complicated - the two completely
different point of views we have actually do represent the same concept.
While this is true for trees and coproducts - since we did a rewriting of
semantic combinations as a form of enhanced syntax, previous results stand -
this is not in itself obvious for string diagrams.
When considering basic trees and coproducts, this is not possible, the best
we could have would be a forgetful functor from the string diagrams to the
trees.
We would thus need to consider multi-trees, with possibly multiple edges
between a node and one of its children.


\subsection{Rewriting Rules}
\label{subsec:rewrite}
Here we provide a rewriting system on derivation trees that allows us to
improve our time complexity by reducing the size of the disambigued grammar.
In the worst case, in big o notation there is no improvement in the size of the
sentence, but there is no loss.
The goal is to reduce branching in the definition of $PT$, as this will easily
translate into reductions in the grammar.

\medskip

When considering the grammar version of the semantic system, the reductions
written below cannot be done when using only one step at a time.
To get around this, a simple way might be to expand the size of the grammar to
consider reductions two at a time, by adding an intermediate non-terminal.
Then, while the size of the grammar becomes quadratic in
$\abs{\mFunc\left(\mL\right)}$, because this also reduces the number of
reductions needed to get the derivations, this compensates the increase, and
actually reduces the time complexity when the reducing the number of rules
after equivalence.
Obviously the same reasoning applies when considering reductions of length $3$,
$4$ and more.
We really just need to consider the same set of reductions that the one
proposed below.

\medskip

First, let's consider the reductions proposed in Section \ref{sec:nondet}.
Those define normal forms under Theorem \ref{thm:confluence} and thus we can
use those to reduce the number of trees.
Indeed, we usually want to reduce the number of effects by first using the
rules \emph{inside} the language, understand, the adjunction co-unit and
monadic join.
Thus, when we have a presupposition of the form:
\begin{equation*}
	\PT{\blank}{\cont x: \f{M}\f{M}\tau} \poulpe \text{or} \poulpe
	\PT{\blank}{\cont x: RL\tau}
\end{equation*}
we always simplify it directly, as not always doing it would propagate
multiple trees.
This in turn means we always verify the derivational rules we set up in the
previous section for the join equations of the monad.

\medskip

Secondly, while there is no way to reduce the branching proposed in the
previous section since they end in completely different reductions, there is
another case in which two different reductions arise:
\begin{equation*}
	\PT{\blank}{\cont x: \f{F}\tau_{1}} \poulpe \text{and} \poulpe
	\PT{\blank}{\cont y: \f{G}\tau_{2}}
\end{equation*}
Indeed, in that case we could either get a result with effects $\f{F}\f{G}$ or
with effects $\f{G}\f{F}$.
In general those effects are not the same, but if they happen to be, which
trivially will be the case when one of the effects is external, the plural or
islands functors for example.
When the two functors commute, the order of application does not matter and
thus we choose to get the outer effect the one of the left side of the
combinator.

\medskip

Thirdly, there are modes that clearly encompass other ones\footnote{Here we use
	the grammar notation for ease of explanation}.
One should not use mode $\combUR$ when using $\combMR$ or $\combDN \combMR$
and the same goes for the left side, because the two derivations yield simpler
results.
Same things can be said for certain other derivations containing the lowering
and co-unit combinators.

\noindent We use $\combDN$ when we have not used any of the following, in all
derivations:
\begin{multicols}{2}
	\begin{itemize}
		\item $m_{\f{F}}, \combDN, m_{\f{F}}$ where
		      $m \in \{\combMR, \combML\}$
		\item $\combML_{\f{F}}, \combDN, \combMR_{\f{F}}$
		\item $\combA_{\f{F}}, \combDN, \combMR_{\f{F}}$
		\item $\combML_{\f{F}}, \combDN, \combA_{\f{F}}$
		\item $\combC$
	\end{itemize}
\end{multicols}
\noindent We use $\combJ$ if we have not used any of the following,
for $j \in \{\epsilon, \combJ_{\f{F}}\}$
\begin{multicols}{2}
	\begin{itemize}
		\item $\left\{m_{\f{F}}, j, m_{\f{F}}\right\}$ where
		      $m \in \{\combMR, \combML\}$
		\item $\combML_{\f{F}}, j, \combMR_{\f{f}}$
		\item $\combA_{\f{F}}, j, \combMR_{\f{F}}$,
		\item $\combML_{\f{F}}, j, \combA_{\f{F}}$
		\item $k, \combC$ for $k \in \{\epsilon, \combA_{\f{F}}\}$
		\item If $\f{F}$ is commutative as a monad:
		      \begin{itemize}
			      \item $\combMR_{\f{F}}, \combA_{\f{F}}$
			      \item $\combA_{\f{F}}, \combML_{\f{F}}$
			      \item $\combMR_{\f{F}}, j, \combML_{\f{F}}$
			      \item $\combA_{\f{F}}, j, \combA_{\f{F}}$
		      \end{itemize}
	\end{itemize}
\end{multicols}

\begin{thm}
	The rules proposed above yield equivalent results.
\end{thm}
\begin{proof}
	For the first point, the equivalence has already been proved under Theorem
	\ref{thm:confluence}.
	For the second point, it is obvious since based on an equality.
	For the third point, it's a bit more complicated.
	The rules about not using combinators $\combUL$ and $\combUR$ come from the
	notion of handling and granting termination and decidability to our system.
	The rules about adding $\combJ$ and $\combDN$ after moving two of the same
	effect from the same side (i.e. $\combML \combML$ or $\combMR\combMR$) are
	normalization of a the elevator equations \ref{eq:elevator}.
	Indeed, in the denotation, the only reason to keep two of the same effects
	and not join them\footnote{Provided they can be joined.} is to at some point
	have something get in between the two.
	Joining and cloture should then be done at earliest point in parsing where it
	can be done, and that is equivalent to later points because of the elevator
	equations, or Theorem \ref{thm:isotopy}.
	The last set of rules follows from the following: we should not use $\combJ
		\combML \combMR$ instead of $\combA$, as those are equivalent because of the
	equation defining them.
	The same thing goes for the other two, as we should use the units of monads
	over applicative rules and fmap.
	To prove the last set of rules, we have used the proof assistant Lean.
	We have provided it with a full set of tactics about $\lambda$-calculus,
	to allow to check equivalence of terms and added definitions for our types,
	combinators and so on, to make use of its language server\footnote{Lean was
		also used by Dylan \textsc{Bumford} to rewrite the full parser and make use
		of its dependent type system, allowing for functions about types and
		effects to be typed.}
\end{proof}

\medskip

The observant reader might have noticed that this is simply a scheme to apply
typing rules to a syntactic derivation, but that this will not be enough to
actually gain all reductions possible in polynomial time.
This is actually not feasible (because of the intrinsic ambiguity of the
English language, as proved for example by the sentence \emph{The man sees the
	girl using a telescope}).
Even then, we are far from reducing to a minimum the number of different paths
possible to get a same final denotation.
This can only happen once a confluent reduction scheme is provided for the
denotations.
When this is done, we can combine the reduction schemes for effects along with
the one for denotation and the one for combinations in one large reduction
scheme.
Indeed, trivally, the tensor product of confluent reduction schemes forms a
confluent reduction scheme, whose maximal reduction length is the sum of the
maximal reduction lengths\footnote{Actually it is at most that, but that does
	not	really matter.}.

\medskip

When using our diagrammatic approach to parsing, which, again, is just a
rewriting in a more graphical fashion of our typing rules and our CFG-like
rules, we can write all the reductions described above to our paradigm:
it simply amounts to constructing a set of normal forms for the string diagrams.
This leads to the same algorithms developed in Section \ref{sec:nondet} being
usable here: we just have a new improved version of Theorem
\ref{thm:confluence} which adds the normal forms specified in this section to
the newly added \emph{orthogonal} axis of diagrammatic computations.
What we're actually doing is computing two different normal forms along the
tensor product of our reduction schemes\footnote{Just like we already have (or
	should have) a tensor scheme for the denotations and combination modes.},
but again, that only amounts to computing a larger normal form.
Moreover, considering the possible normal forms of syntactic reductions simply
adds another way to reduce our diagrams to normal forms.
Since all of these forms can be attained in polynomial time, it is clear that
finding a normal-form diagram, which is exactly a normal-form denotation for
the sentence, is doable in polynomial time\footnote{Of course, this is only
	true when the denotations can be normalized in polynomial time.}

\medskip

Once again, there is no evidence that our system is complete, if not the
contrary, so the arguments developed in Section \ref{subsubsec:sanity} are
still valid.
A way to "complete" it, although it would still probably be incomplete would be
to write an automatized prover in Lean, but this is out of the scope of this
project, as it would not do many improvements.


\section{Conclusion}
The functional programming approach developed in
\cite{bumfordEffectdrivenInterpretationFunctors2025} allows for increased
expressiveness in the choice of denotations, especially from a purely
theoretical point of view.
In this paper we have successfully prove that it is well-founded theoretically,
but also that it doesn't come at the cost of comprehensibility or efficiency.

Moreover, while our methods for implementing a type and effect system have been
applied to natural language semantics, it could be applied in any language with
purely compositional semantics.
Of course, there are still improvements that can be made, in particular around
the unorthodox use of effects to define what we have called higher-order
constructs and scope islands, but also in integrating the theory in more
complicated models of denotations, such as the ones learned through a neural
network for example.
In that case, it would suffice to understand what base combinators exist for
the model to implement the formalism we described.

\appendix
\bibliography{tdparse.bib}

\section{Presenting a Language}
In this appendix, we provide tables (\ref{fig:lexicon} and \ref{fig:functors}) describing the
modeling of a subset of the English language in our formalism.

\begin{figure}[H]
	\centering
	\begin{subfigure}{.9\textwidth}
		\centering
			\setcellgapes{2pt}
	\makegapedcells
	\begin{NiceTabular}{>{\bf}LLL}
		Expression & \rm Type & \lambda\text{-Term} \\
		\word{planet}{\e\to\t}{\lambda x. \w{planet} x}{common nouns}
		\word{carnivorous}{\left( \e \to \t \right)}{\lambda x. \w{carnivorous}x}{predicative adjectives}
		\word{skillful}{\left( \e \to \t \right) \to \left( \e \to \t \right)}{\lambda p. \lambda x. px \land \w{skillful} x}{predicate modifier adjectives}
		\word{Jupiter}{\e}{{\bf j}\in \Var}{proper nouns}
		\word{sleep}{\e \to \t}{\lambda x. \w{sleep} x}{intranitive verbs}
		\word{chase}{\e \to \e \to \t}{\lambda o. \lambda s. \mathbf{chase}\left( o \right)\left( s \right)}{transitive verbs}
		\word{be}{\left( \e \to \t \right) \to \e \to \t}{\lambda p. \lambda x. px}{}
		\word{she}{\e\to\e}{\lambda x. x}{}
		\word{it}{\left( \bot \to \f{G}\e \right)}{\lambda g. g_{0}}{}
		\word{which}{\left( \e \to \t \right)\to \f{S}\e}{\lambda p. \left\{x \suchthat px\right\}}{}
		\word{the}{\left( \e \to \t \right) \to \f{M}\e}{\lambda p. x \text{ if } p^{-1}\left( \top \right) = \{x\} \text{ else } \#}{}
		\word{a}{\left( \e \to \t \right) \to \f{D}\e}{\lambda p. \lambda s. \left\{ \scalar{x, x + s}\suchthat p x\right\}}{}
		\word{no}{\left( \e \to \t \right) \to \f{C}\e}{\lambda p. \lambda c. \lnot \exists x. p x \land c\, x}{}
		\word{every}{\left( \e \to \t \right)\to \f{C}\e}{\lambda p. \lambda c. \forall x, px \Rightarrow cx}{}
		\CodeAfter
		\begin{tikzpicture}
			\draw[double] (1|-2) -- (4|-2);
			\foreach \r in {4,6,...,14} {\draw (1|-\r) -- (4|-\r);}
			\foreach \r in {15,...,21} {\draw (1|-\r) -- (4|-\r);}
		\end{tikzpicture}
	\end{NiceTabular}

		\caption{Lexicon for a subset of the English language}
		\label{fig:lexicon}
	\end{subfigure}

	\medskip

	\begin{subfigure}{\textwidth}
		\centering
		\resizebox{\textwidth}{!}{%
				\def\arraystretch{1.3}
	\setcellgapes{2pt}
	\makegapedcells
	\begin{NiceTabular}{LLc}
		\rm Constructor                                              & \fmap                                                                                     & Typeclass \\
		\f{G}\left( \tau \right) = \r \to \tau                       & \f{G}\phi\left( x \right) = \lambda r. \phi \left(x r\right)                              & Monad     \\
		\f{W}\left( \tau \right) = \tau \times \t                    & \f{W}\phi\left( \left( a, p \right) \right) = \left( \phi a, p \right)                    & Monad     \\
		\f{S}\left( \tau \right) = \{ \tau \}                        & \f{S}\phi\left( \left\{ x \right\} \right) = \left\{ \phi(x) \right\}                     & Monad     \\
		\f{C}\left( \tau \right) = \left( \tau \to \t \right) \to \t & \f{C}\phi\left( x \right) = \lambda c. x\left( \lambda a. c \left( \phi a \right) \right) & Monad     \\
		\f{M}\left( \tau \right) = \tau + \bot                       & \f{M}\phi\left( x \right) = \begin{cases}
			                                                                                           \phi\left( x \right) & \text{if } \cont x: \tau \\
			                                                                                           \#                   & \text{if } \cont x: \#
		                                                                                           \end{cases}                                        & Monad                \\
		\CodeAfter
		\begin{tikzpicture}
			\draw[double] (1|-2) -- (5|-2);
			\foreach\r in {3,...,6} {%
					\draw (1|-\r) -- (5|-\r);
				}
		\end{tikzpicture}
	\end{NiceTabular}

		}
		\caption{Definition of a few functors, with their map on functions}
		\label{fig:functors}
	\end{subfigure}
	\caption{Presentation of a lambda-calculus lexicon for the English language}
\end{figure}


\end{document}
