\documentclass[math, english, info, noamsthm]{beamercours}
\makeatletter
\def\tikzimp@rt{1}
\makeatother

\usepackage{bigstrut}
\usepackage{makecell}
\usepackage{emoji}
\usepackage{tikz-dependency}
\usepackage{diag}

\def\cont{\Gamma\vdash}
\def\poulpe{\qquad}

\setlength\belowcaptionskip{0pt}
\setlength\abovecaptionskip{\baselineskip}

\DeclareMathOperator{\Var}{Var}

\def\ppl{\mathbin{+\mkern-12mu+}}

\makeatletter
\renewenvironment{thebibliography}[1]
     {\section{\bibname}
      \@mkboth{\MakeUppercase\bibname}{\MakeUppercase\bibname}%
      \list{\@biblabel{\@arabic\c@enumiv}}%
           {\settowidth\labelwidth{\@biblabel{#1}}%
            \leftmargin\labelwidth
            \advance\leftmargin\labelsep
            \@openbib@code
            \usecounter{enumiv}%
            \let\p@enumiv\@empty
            \renewcommand\theenumiv{\@arabic\c@enumiv}}%
      \sloppy
      \clubpenalty4000
      \@clubpenalty \clubpenalty
      \widowpenalty4000%
      \sfcode`\.\@m}
     {\def\@noitemerr
       {\@latex@warning{Empty `thebibliography' environment}}%
      \endlist}

\def\black@or@white#1#2{%
  \@tempdima#2 pt
  \ifdim\@tempdima>0.5 pt
    \definecolor{temp@c}{gray}{0}%
  \else
    \definecolor{temp@c}{gray}{1}%
  \fi}
\def\letterbox#1#{\protect\letterb@x{#1}}
\def\letterb@x#1#2#3{%
  \colorlet{temp@c}[gray]{#2}%
  \extractcolorspec{temp@c}{\color@spec}%
  \expandafter\black@or@white\color@spec
  {\color#1{temp@c}\tallcbox#1{#2}{#3}}}
\def\tallcbox#1#{\protect\color@box{#1}}
\def\color@box#1#2{\color@b@x\relax{\color#1{#2}}}
\long\def\color@b@x#1#2#3%
 {\leavevmode
  \setbox\z@\hbox{{\set@color#3}}%
  \ht\z@\ht\strutbox
  \dp\z@\dp\strutbox
  {#1{#2\color@block{\wd\z@}{\ht\z@}{\dp\z@}\box\z@}}}
\makeatother

\contourlength{0.005em}
\def\backbox#1{\letterbox{Lavender!40}{\contour{black}{#1}}}

\def\ty#1{\backbox{\tt\color{yulm!90!black}#1}}
\def\f#1{\backbox{\tt\color{vulm}#1}}
\def\w#1{\mathbf{#1}\,}

\def\e{\ty{e}}
\def\t{\ty{t}}
\def\r{\ty{r}}

\newcolumntype{C}{>{$}c<{$}}
\newcolumntype{L}{>{$}l<{$}}
\newcolumntype{R}{>{$}r<{$}}
\def\fmap{\texttt{fmap}}


\makeatletter
\newcommand{\@word}[4][]{%
	#2 & #3 & #4\\
\ifx&#1&%
	%
\else
	&\multicolumn{2}{l}{Generalizes to \textbf{#1}}\\%
\fi%
}
\def\word#1#2#3#4{\@word[#4]{#1}{#2}{#3}}
\makeatother

\usepackage{calc}

\makeatletter
\def\textSq#1{%
\begingroup% make boxes and lengths local
\setlength{\fboxsep}{0.4ex}% SET ANY DESIRED PADDING HERE
\setbox1=\hbox{#1}% save the contents
\setlength{\@tempdima}{\maxof{\wd1}{\ht1+\dp1}}% size of the box
\setlength{\@tempdimb}{(\@tempdima-\ht1+\dp1)/2}% vertical raise
\raise-\@tempdimb\hbox{\fbox{\vbox to \@tempdima{%
  \vfil\hbox to \@tempdima{\hfil\copy1\hfil}\vfil}}}%
\endgroup%
}
\def\Sq#1{\textSq{\ensuremath{#1}}}%

\def\c@lsep{2.3}
\def\r@wsep{.8}

\tikzset{
	uptree/.style={
			draw=green!80!black,
			thick,
		},
	typenode/.style={
			align=center,
			text width=24mm,
			%font={\large},
		},
	treenode/.style={
			align=center,
			text width=24mm,
		},
	wordnode/.style={
			inner sep=0pt,
			align=center,
			font={\large},
		},
	downtree/.style={
			draw=red!80!black,
			thick,
		},
}

\newcommand{\wnode}[3]{%
	\node (#2) at (#1*\c@lsep, 0) [wordnode] {#2};
	\node[anchor=north] (#2-) at ($(#1*\c@lsep, 0) + (0, -.142)$) [typenode] {\ensuremath{#3}};
}
\newcommand{\utnode}[3]{%
	\path let \p1 = (#2.north), \p2 = (#3.north) in coordinate (Q1) at (\x1, {max(\y1, \y2)});
	\path let \p1 = (#2.north), \p2 = (#3.north) in coordinate (Q2) at (\x2, {max(\y1, \y2)});
	\node (#2#3) at ($($(Q1)!0.5!(Q2)$) + (0, 1)$) [treenode] {\ensuremath{#1}};
	\draw[uptree] ($(#2.north) + (0, .142)$) -- (#2#3.south);
	\draw[uptree] ($(#3.north) + (0, .142)$) -- (#2#3.south);
}
\newcommand{\dtnode}[4][0.5]{%
	\path let \p1 = (#3.south), \p2 = (#4.south) in coordinate (Q1) at (\x1, {min(\y1, \y2)});
	\path let \p1 = (#3.south), \p2 = (#4.south) in coordinate (Q2) at (\x2, {min(\y1, \y2)});
	\node (#3#4) at ($($(Q1)!#1!(Q2)$) + (0, -1)$) [treenode] {\ensuremath{#2}};
	\draw[downtree] ($(#3.south) + (0, -.142)$) -- (#3#4.north);
	\draw[downtree] ($(#4.south) + (0, -.142)$) -- (#3#4.north);
}

\def\inputtikz#1{
	\ifnum\tikzimp@rt=1
		\input{figures/#1}
	\else
		\ensuremath{\text{\Huge\color{vulm}A TikZ PICTURE GOES HERE.}}
	\fi
}
\makeatother

\catstyle{catone}{gray!50}
\catstyle{catmc}{vulm!10!yulm}
\catstyle{catmca}{vulm!20!yulm}
\catstyle{catmcb}{vulm!30!yulm}
\catstyle{catmcc}{vulm!40!yulm}
\catstyle{catmcd}{vulm!50!yulm}
\catstyle{catmce}{vulm!60!yulm}
\catstyle{catmcf}{vulm!70!yulm}
\catstyle{catmcg}{vulm!80!yulm}
\catstyle{catmch}{vulm!90!yulm}

\def\din#1{#1\mathrm{.S}}
\def\dnb#1{#1\mathrm{.N}}
\def\dlb#1#2{#1\mathrm{.L}\left(#2\right)}
\def\dl#1{#1\mathrm{.L}}
\def\dnlg#1{#1\mathrm{.h}}
\def\dnin#1{#1\mathrm{.in}}
\def\dnout#1{#1\mathrm{.out}}

\newcounter{lingexcnt}
\newcounter{tmplingexcnt}
\renewcommand*{\thelingexcnt}{(\arabic{lingexcnt})}
\newenvironment{sentence}[1][]{
     \begin{list}{\thelingexcnt}{\refstepcounter{lingexcnt}}\item
     \ifnum\pdfstrcmp{#1}{}=0\else\label{#1}\fi
}{\end{list}}

\newenvironment{nsentence}{%
     \setcounter{tmplingexcnt}{\value{lingexcnt}}
     \addtocounter{tmplingexcnt}{-1}
     \begin{list}{\thelingexcnt}{
         \usecounter{lingexcnt}
         \setcounter{lingexcnt}{\value{tmplingexcnt}}
         \refstepcounter{lingexcnt}
     }
}{\end{list}}

\newcommand*{\oneSentence}[2][]{\begin{sentence}[#1]#2\end{sentence}}


\title{Formalizing Typing Rules for Natural Languages using Effects}
\author{Matthieu Boyer}
% \institute{\includegraphics[height=3em]{ens_psl}\hspace{2cm}\includegraphics[height=3em]{yale_logo}}

\begin{document}
\maketitle

\section{Introduction}
\begin{frame}[fragile]
	\frametitle{What the hell am I doing?}
	% What I'm working on, generic scope => Simon's work
	% What we're going to talk about in math
	% What I have already done
	\begin{center}
			\begin{tikzpicture}[baseline={([yshift=-.5ex]current bounding box.center)}]
			\path coordinate[dot, label=right:$\w{a}\w{mouse}$] (a) + (0, 1) coordinate[dot, label=right:$\w{eats}$] (eats) + (0, 2) coordinate[label=above:$\t$] (bool)
			++ (-1, 1) coordinate (ctla) + (0, 1) coordinate[label=above:$\f{D}$] (effa)
			++ (1, -2) coordinate[dot, label=right:$\w{the}\w{cat}$] (the) + (-2, 2) coordinate (ctlthe) + (-2, 3) coordinate[label=above:$\f{M}$] (effthe)
			++ (0, -1) coordinate[label=below:$\phantom{\e}\bot\phantom{\e}$] (bot);
			\draw[dashed] (bot) -- (the);
			\draw (the) -- (bool);
			\draw (the) to[out=180, in=-90] (ctlthe) -- (effthe);
			\draw (a) to[out=180, in=-90] (ctla) -- (effa);
			\begin{pgfonlayer}{background}
				\fill[catone] (bot) rectangle ($(bool) + (2, 0)$);
				\fill[catmcb] (bot) rectangle ($(effthe) + (-1, 0)$);
				\fill[catmca] (the) to[out=180, in=-90] (ctlthe) -- (effthe) -- (bool) -- cycle;
				\fill[catmc] (a) to[out=180, in=-90] (ctla) -- (effa) -- (bool) -- cycle;
			\end{pgfonlayer}
		\end{tikzpicture}
	\end{center}
\end{frame}

\begin{frame}
	\frametitle{Defining definites and indefinites}

\end{frame}

\section{Mathematical Background}
\subsection{Category Theory}
\begin{frame}[fragile]
	\frametitle{Slide 42}
	\only<1-6>{\begin{definition}[Category]
	A (small) \emph{category} is described by the following data:
	\begin{itemize}
			\only<1-4>{\item[0] A class of objects (nodes of a graph).}
			\only<2-4>{\item[1] For a pair $A, B$ of objects, a set $\Hom(A, B)$ of functions from $A$ to $B$ called \emph{morphisms}, \emph{maps} or \emph{arrows}.
			We denote it by $f: A \to B$ or $A \xrightarrow{f} B$.}
		\only<3-4>{\item[2] For all triplets $A, B, C$, a composition law $\circ_{A, B, C}$:
			\vspace{-4pt}
		      \begin{equation*}
			      \begin{array}{rcl}
				      \Hom(B, C) \times \Hom(A, B) & \rightarrow & \Hom(A, C) \\
				      (g, f)                       & \mapsto     & g\circ f
			      \end{array}
		  \end{equation*}}
	  	\only<4>{\item[2] For all objects $A$, an identity map $\id_{A} \in \Hom(A, A)$.}
		\only<5-6>{\item[3] Associativity: $f\circ \left( g \circ h \right) = \left( f\circ g \right) \circ h = f \circ g \circ h$}
			  \only<6>{\item[3] Unitarity: $f\circ \id_{A} = f = \id_{B} \circ f$.}
	\end{itemize}
	\label{def:categorie}
\end{definition}}
\only<7->{%
	A few examples of categories are:
	\begin{description}
			\only<7->{\item[Set] whose objects are Sets and arrows are functions between sets.}
			\only<8->{\item[Top] whose objects are Topological Spaces and arrows are continuous maps between those.}
			\only<9->{\item[Grp] whose objects are Groups and arrows are Group Homomorphisms.}
			\only<10->{\item[Vec] whose objects are Vector Spaces on a field $k$ and arrows are Linear Maps.}
	\end{description}
}
\end{frame}

\begin{frame}[fragile]
	\frametitle{Commuter Rail, basically}
	The right language for categories is the commutative diagram one. The associativity rewrites as:
			\begin{category}[]
				& B \ar[r, "g"] & C\ar[dr, "h"] & \\
				A\ar[ur, "f"]\ar[urr, "g\circ f"']\arrow[rrr, "h\circ\left(g\circ f\right)"'] & & & D
			\end{category}
		\end{frame}

\begin{frame}[fragile]
	\frametitle{Commuter Rail, basically}
		In $Set$ we have the following commutative diagram:
		\begin{category}[]
			\R\ar[r, "\times 2"]\ar[d, "\cdot^{2}"'] & \R\ar[d, "\cdot^{2}"]\\
			\R^{+}\ar[r, "\times 4"] & \R^{+}
		\end{category}
		\visible<2>{It simply states that:
		\begin{equation*}
			\forall x \in \R, \left( 2x^{2} \right) = 4x^{2}
	\end{equation*}}
\end{frame}

\begin{frame}[fragile]
	\frametitle{Of wolf, and man.}
\begin{definition}
	Let $\A, \B$ be two categories.
	A functor $\mF : \A\to \B$ is:
	\begin{enumerate}
		\item[0] An object $F(A) \in \B$ for each object $A$ of $\A$.
		\item[1] For each pair $A_{1}, A_{2} \in \A$, a function:
		      \begin{equation*}
			      F_{A_{1}, A_{2}}: \applic{\Hom_{\A}(A_{1}, A_{2})}{\Hom_{\B}(FA_{1}, FA_{2})}{f}{F(f)}
		  \end{equation*}
	\end{enumerate}
	\label{def:foncteur}
\end{definition}
\end{frame}

\begin{frame}[fragile]
	\frametitle{Of wolf, and man.}
	We ask the following equations to be satisfied:
				  $F(g \circ f) = F(g) \circ F(f)$, that is
						\begin{category}
							A\ar[r, "g"]\ar[d, "F"] & B\ar[r, "f"]\ar[d, "F"] & C\ar[d, "F"] \\
							FA\ar[r, "Fg"'] & FB\ar[r, "Ff"'] & FC
						\end{category}
					and $F(\id_{A}) = \id_{F(A)}$
\end{frame}

\begin{frame}[fragile]
	\frametitle{O.K. Corral}
\begin{definition}
	A natural transformation is a functor in the category of small categories and functors. If $F, G: \A \Rightarrow \B$ are functors, a natural transformation $\theta$ from $F$ to $G$ is, for each object of $\A$ a function $\theta_{A}$ such that the following diagram commutes for all $f: A \to B$.
\begin{category}
	FA\ar[r, "Ff"]\ar[d, "\theta_{A}"'] & FB\ar[d, "\theta_{B}"]\\
	GA\ar[r, "Gf"] & GB
\end{category}
\end{definition}
\end{frame}

\begin{frame}[fragile]
	\frametitle{O.K. Corral}
	\begin{definition}
		An adjunction $L \dashv R$ between two functors $L: \A \to \B$ and $R: \B \to \A$ is a pair of natural transformations $\eta: \Id_{\A} \Rightarrow R \circ L$ and $\epsilon: L \circ R \Rightarrow \Id_{\B}$ verifying the zigzag equations:
		\begin{center}
			\resizebox{\linewidth}{!}{\begin{tikzcd}
		\A\ar[r, "\id_{\A}" name=idA]\ar[dr, "L"'] &[1.8cm] \A\ar[dr, "L"]\ar[phantom, "" name=A] &[1.8cm] \\[2cm]
		& \B\ar[u, "R"]\ar[phantom, "" name=B]\ddarrow["\Sq{\eta}"]{idA}{B}\ar[r, "\id_{B}" name=idB]\ddarrow["\text{\textcircled{$\epsilon$}}"]{A}{idB} & \B
	\end{tikzcd}
	= \begin{tikzcd}
		\A\ar[r, bend left=45, "L"' name=L]\ar[r, bend right=45, "L" name=R]\ddarrow["\id_{L}"]{L}{R}&[2cm] \B
	\end{tikzcd}
}
	\end{center}
	\end{definition}
\end{frame}

\subsection{Type Theory}
\begin{frame}[fragile]
	\frametitle{Yu-Gi-Oh}
	\begin{definition}
	A product of two objects $A$ and $B$ in a category $\phi$ is a triplet
	\begin{equation*}
		\left(A \times B, \pi_{1}: A\times B \to A, \pi_{2}: A \times B \to B\right)
	\end{equation*}
	\begin{category}[]
		A & & B \\
		& A\times B \arrow{ul}{\pi_{1}}\arrow{ur}{\pi_{2}} &
	\end{category}
	such that for all pair of arrows $X\xrightarrow{f} A$ et $X\xrightarrow{g} B$, there is a unique $h: X \to A \times B$ such that
	$f = \pi_{1} \circ h, g = \pi_{2} \circ h$.
	\label{def:prodcart}
\end{definition}
\end{frame}

\begin{frame}
	\frametitle{Yu-Gi-Oh 2}
	\begin{definition}
		A terminal object $\mathds{1}$ in a category $\cont$ is an object such that for all $A$ in $\cont$ there is one and only one arrow $A \to \mathds{1}$.
	\end{definition}
	An initial object is the same thing with the arrows reversed.
	\begin{definition}
		A category is cartesian if all products exist and it has a terminal object $\mathds{1}$.
	\end{definition}
	Cartesian products and terminal objects are unique, up to isomorphism.
\end{frame}

\begin{frame}[fragile]
	\frametitle{Yu-Gi-Oh 3}
	\begin{definition}
		A cartesian closed category is a cartesian category where we define for each object $A$ a functor $A \Rightarrow \cont \to \cont$ right adjunct to $A \times \cont \to \cont$.
		This means we have bijections $\Phi_{X, Y}: \Hom\left( A \times X, Y \right) \to \Hom\left( X, A \Rightarrow Y \right)$ called currification bijections..
	\end{definition}
	\visible<2>{$Set$ is a cartesian closed category, where currification is defined by the partial application of high arity functions to a subset of their arguments.}
\end{frame}

\begin{frame}
	\frametitle{Use the Types, Luke}
	\only<1>{\begin{definition}
		The $\lambda$-calculus is defined by the following basic grammar:
		\begin{grammar}
			\firstrule{E}{$x \in Var$}{Variables}
			\grule{$App(E, E)$}{Application}
			\grule{$\lambda x.E$}{Evaluation}
		\end{grammar}
\end{definition}}
\only<2>{\begin{definition}
We give ourselves a set of type variables $TyVar$ and define types by:
\begin{center}
\begin{mgrammar}
	\firstrule{A, B}{\alpha\in TyVar}{}
	\grule{A\times B}{}
	\grule{\mathds{1}}{}
	\grule{A \Rightarrow B}{}
\end{mgrammar}
\end{center}
A context $M = x_{1}: A_{1}, \ldots, x_{n}: A_{n}$ is a list of pairs $x_{i}: A_{i}$ with a variable $x_{i}$ and a type $A_{i}$, all the variables being different.
\end{definition}}
\end{frame}

\begin{frame}
	\frametitle{Fudge Supreme}
\begin{definition}
	A typing $\Gamma\vdash M: A$ is a triplet composed of a context $\Gamma$, a $\lambda$-term $M$ and a type $A$, such that all free variables of $M$ are in $\Gamma$.
	A proof tree for a typing judgement is constructed inductively from a set of rules of the form:
	\begin{align*}
		\frac{}{x:A \vdash x:A}\fracnotate{Var}\\
		\frac{\Gamma, x:A\vdash M:B}{\Gamma\vdash \lambda x.M:A\Rightarrow B}\fracnotate{Lam}\\
		\frac{\Gamma\vdash M:A \Rightarrow B \hspace{1cm} \Delta \vdash N:A}{\Gamma, \Delta\vdash App(M, N):B}\fracnotate{App}
	\end{align*}
\end{definition}
\end{frame}

\section{Effects, Semantics}


\begin{frame}[allowframebreaks]
	\frametitle{Bibliography}
\bibliographystyle{nals}
\bibliography{tdparse.bib}
\end{frame}

\end{document}
