\documentclass[math, english, info]{beamercours}
\makeatletter
\def\tikzimp@rt{1}
\makeatother

\bibliography{tdparse.bib}

\usepackage{bigstrut}
\usepackage{makecell}
\usepackage{emoji}
\usepackage{tikz-dependency}
\usepackage{diag}

\def\cont{\Gamma\vdash}
\def\poulpe{\qquad}

\setlength\belowcaptionskip{0pt}
\setlength\abovecaptionskip{\baselineskip}

\DeclareMathOperator{\Var}{Var}

\def\ppl{\mathbin{+\mkern-12mu+}}

\makeatletter
\renewenvironment{thebibliography}[1]
     {\section{\bibname}
      \@mkboth{\MakeUppercase\bibname}{\MakeUppercase\bibname}%
      \list{\@biblabel{\@arabic\c@enumiv}}%
           {\settowidth\labelwidth{\@biblabel{#1}}%
            \leftmargin\labelwidth
            \advance\leftmargin\labelsep
            \@openbib@code
            \usecounter{enumiv}%
            \let\p@enumiv\@empty
            \renewcommand\theenumiv{\@arabic\c@enumiv}}%
      \sloppy
      \clubpenalty4000
      \@clubpenalty \clubpenalty
      \widowpenalty4000%
      \sfcode`\.\@m}
     {\def\@noitemerr
       {\@latex@warning{Empty `thebibliography' environment}}%
      \endlist}

\def\black@or@white#1#2{%
  \@tempdima#2 pt
  \ifdim\@tempdima>0.5 pt
    \definecolor{temp@c}{gray}{0}%
  \else
    \definecolor{temp@c}{gray}{1}%
  \fi}
\def\letterbox#1#{\protect\letterb@x{#1}}
\def\letterb@x#1#2#3{%
  \colorlet{temp@c}[gray]{#2}%
  \extractcolorspec{temp@c}{\color@spec}%
  \expandafter\black@or@white\color@spec
  {\color#1{temp@c}\tallcbox#1{#2}{#3}}}
\def\tallcbox#1#{\protect\color@box{#1}}
\def\color@box#1#2{\color@b@x\relax{\color#1{#2}}}
\long\def\color@b@x#1#2#3%
 {\leavevmode
  \setbox\z@\hbox{{\set@color#3}}%
  \ht\z@\ht\strutbox
  \dp\z@\dp\strutbox
  {#1{#2\color@block{\wd\z@}{\ht\z@}{\dp\z@}\box\z@}}}
\makeatother

\contourlength{0.005em}
\def\backbox#1{\letterbox{Lavender!40}{\contour{black}{#1}}}

\def\ty#1{\backbox{\tt\color{yulm!90!black}#1}}
\def\f#1{\backbox{\tt\color{vulm}#1}}
\def\w#1{\mathbf{#1}\,}

\def\e{\ty{e}}
\def\t{\ty{t}}
\def\r{\ty{r}}

\newcolumntype{C}{>{$}c<{$}}
\newcolumntype{L}{>{$}l<{$}}
\newcolumntype{R}{>{$}r<{$}}
\def\fmap{\texttt{fmap}}


\makeatletter
\newcommand{\@word}[4][]{%
	#2 & #3 & #4\\
\ifx&#1&%
	%
\else
	&\multicolumn{2}{l}{Generalizes to \textbf{#1}}\\%
\fi%
}
\def\word#1#2#3#4{\@word[#4]{#1}{#2}{#3}}
\makeatother

\usepackage{calc}

\makeatletter
\def\textSq#1{%
\begingroup% make boxes and lengths local
\setlength{\fboxsep}{0.4ex}% SET ANY DESIRED PADDING HERE
\setbox1=\hbox{#1}% save the contents
\setlength{\@tempdima}{\maxof{\wd1}{\ht1+\dp1}}% size of the box
\setlength{\@tempdimb}{(\@tempdima-\ht1+\dp1)/2}% vertical raise
\raise-\@tempdimb\hbox{\fbox{\vbox to \@tempdima{%
  \vfil\hbox to \@tempdima{\hfil\copy1\hfil}\vfil}}}%
\endgroup%
}
\def\Sq#1{\textSq{\ensuremath{#1}}}%

\def\c@lsep{2.3}
\def\r@wsep{.8}

\tikzset{
	uptree/.style={
			draw=green!80!black,
			thick,
		},
	typenode/.style={
			align=center,
			text width=24mm,
			%font={\large},
		},
	treenode/.style={
			align=center,
			text width=24mm,
		},
	wordnode/.style={
			inner sep=0pt,
			align=center,
			font={\large},
		},
	downtree/.style={
			draw=red!80!black,
			thick,
		},
}

\newcommand{\wnode}[3]{%
	\node (#2) at (#1*\c@lsep, 0) [wordnode] {#2};
	\node[anchor=north] (#2-) at ($(#1*\c@lsep, 0) + (0, -.142)$) [typenode] {\ensuremath{#3}};
}
\newcommand{\utnode}[3]{%
	\path let \p1 = (#2.north), \p2 = (#3.north) in coordinate (Q1) at (\x1, {max(\y1, \y2)});
	\path let \p1 = (#2.north), \p2 = (#3.north) in coordinate (Q2) at (\x2, {max(\y1, \y2)});
	\node (#2#3) at ($($(Q1)!0.5!(Q2)$) + (0, 1)$) [treenode] {\ensuremath{#1}};
	\draw[uptree] ($(#2.north) + (0, .142)$) -- (#2#3.south);
	\draw[uptree] ($(#3.north) + (0, .142)$) -- (#2#3.south);
}
\newcommand{\dtnode}[4][0.5]{%
	\path let \p1 = (#3.south), \p2 = (#4.south) in coordinate (Q1) at (\x1, {min(\y1, \y2)});
	\path let \p1 = (#3.south), \p2 = (#4.south) in coordinate (Q2) at (\x2, {min(\y1, \y2)});
	\node (#3#4) at ($($(Q1)!#1!(Q2)$) + (0, -1)$) [treenode] {\ensuremath{#2}};
	\draw[downtree] ($(#3.south) + (0, -.142)$) -- (#3#4.north);
	\draw[downtree] ($(#4.south) + (0, -.142)$) -- (#3#4.north);
}

\def\inputtikz#1{
	\ifnum\tikzimp@rt=1
		\input{figures/#1}
	\else
		\ensuremath{\text{\Huge\color{vulm}A TikZ PICTURE GOES HERE.}}
	\fi
}
\makeatother

\catstyle{catone}{gray!50}
\catstyle{catmc}{vulm!10!yulm}
\catstyle{catmca}{vulm!20!yulm}
\catstyle{catmcb}{vulm!30!yulm}
\catstyle{catmcc}{vulm!40!yulm}
\catstyle{catmcd}{vulm!50!yulm}
\catstyle{catmce}{vulm!60!yulm}
\catstyle{catmcf}{vulm!70!yulm}
\catstyle{catmcg}{vulm!80!yulm}
\catstyle{catmch}{vulm!90!yulm}

\def\din#1{#1\mathrm{.S}}
\def\dnb#1{#1\mathrm{.N}}
\def\dlb#1#2{#1\mathrm{.L}\left(#2\right)}
\def\dl#1{#1\mathrm{.L}}
\def\dnlg#1{#1\mathrm{.h}}
\def\dnin#1{#1\mathrm{.in}}
\def\dnout#1{#1\mathrm{.out}}

\newcounter{lingexcnt}
\newcounter{tmplingexcnt}
\renewcommand*{\thelingexcnt}{(\arabic{lingexcnt})}
\newenvironment{sentence}[1][]{
     \begin{list}{\thelingexcnt}{\refstepcounter{lingexcnt}}\item
     \ifnum\pdfstrcmp{#1}{}=0\else\label{#1}\fi
}{\end{list}}

\newenvironment{nsentence}{%
     \setcounter{tmplingexcnt}{\value{lingexcnt}}
     \addtocounter{tmplingexcnt}{-1}
     \begin{list}{\thelingexcnt}{
         \usecounter{lingexcnt}
         \setcounter{lingexcnt}{\value{tmplingexcnt}}
         \refstepcounter{lingexcnt}
     }
}{\end{list}}

\newcommand*{\oneSentence}[2][]{\begin{sentence}[#1]#2\end{sentence}}


\title{Effect-Driven Parsing}
\subtitle{Formal studies on a categorical approach to semantic parsing}
\institute{École Normale Supérieure | Yale University}
\date{7\textsuperscript{th} September 2025}
\addlogo{~/Pictures/Yale_University_logo.svg.png}
\addlogo{~/DEV/latex/source/ens_psl.pdf}

% \colorlet{main-below}{yulm}
% \colorlet{main-accents}{black}
% \colorlet{main-background}{vulm!5}
% \colorlet{main-text}{black}

\begin{document}
\fancytitleframe

\section{Introduction}
\subsection{General Introduction}
\begin{frame}
	\frametitle{Context}
	This work, based on \cite{bumfordEffectdrivenInterpretationFunctors2025} aims
	to provide a categorical formalization of a type and effects system for
	semantic interpretation of the natural language.

	\medskip

	We will develop a graphical formalism for semantic type-driven parsing that
	explains how to derive the meaning of a sentence from the meaning of its
	words.
\end{frame}

\begin{frame}[fragile]
	\frametitle{Typed Semantics for Natural Languages}
	\setcellgapes{3pt}
	\makegapedcells
	\begin{NiceTabular}{>{\bf}LLL}
		Expression & \rm Type & \lambda\text{-Term} \\
		\word{planet}{\e\to\t}{\lambda x. \w{planet} x}{common nouns}
		\word{carnivorous}{\left( \e \to \t \right)}{\lambda x. \w{carnivorous}x}{predicative adjectives}
		\word{skillful}{\left( \e \to \t \right) \to \left( \e \to \t \right)}{\lambda p. \lambda x. px \land \w{skillful} x}{predicate modifier adjectives}
		\word{Jupiter}{\e}{{\bf j}\in \Var}{proper nouns}
		\word{sleep}{\e \to \t}{\lambda x. \w{sleep} x}{intransitive verbs}
		\CodeAfter
		\begin{tikzpicture}
			\draw[double] (1|-2) -- (4|-2);
			\foreach \r in {4,6,...,10} {\draw (1|-\r) -- (4|-\r);}
		\end{tikzpicture}
	\end{NiceTabular}
\end{frame}

\begin{frame}[fragile]
	\frametitle{Syntactic Types and Semantic Types}
	What should be the type of expressions such as \textbf{a cat} or \textbf{Jupiter, a planet}?
	\pause

	\smallskip

	Since we should be able to use \textbf{a cat} and \textbf{the cat} interexchangebly - from a syntactic point of view - they should have the same type.
	There is no one \textbf{cat} that is a representant of the cat.

	\pause

	We will use \emph{(side-)effects} to do the difference between them:
	\begin{equation*}
		\w{a\ cat} = \{c \mid \w{cat} c\}
	\end{equation*}
	\begin{equation*}
		\w{the\ cat} = x \text{ if } \mathbf{cat}^{-1}(\top) = \{x\} \text{ else } \#
	\end{equation*}
\end{frame}

\subsection{Categorical Introduction}
\begin{frame}
	\frametitle{Effects as Functors}
	Since \cite{moggiComputationalLambdacalculusMonads1989}, we know that monads
	are a good model for programming side-effects.

	\pause

	\medskip

	In our case however, functors are enough, because we might sometimes want to be unable
	to merge effects together nor create them from nothing (see below).
\end{frame}

\begin{frame}
	\frametitle{String Diagrams}
	String Diagrams are a formalism (see \cite{hinzeIntroducingStringDiagrams2023}
	for example) that allows to visually represent the different threads of a
	computation and the possible side-effects that appear.

	\pause

	\medskip

	They were presented as a categorical object in
	\cite{joyalGeometryTensorCalculus1991} and will be the focus of our study, as
	a tool introduced for parsing, similarly to
	\cite{coeckeMathematicalFoundationsCompositional2010}.
\end{frame}

\begin{frame}
	\frametitle{Other Categorical Theories}
	In \cite{marcollimatildeetchomskynoametberwickrobertc.MathematicalStructureSyntactic},
	and \cite{senturiaAlgebraicStructureMorphosyntax2025}, a mathematical model
	of parsing trees based on a Hopf algebra is presented.

	\medskip

	This allows to integrate morphological features inside the syntactic
	structures without breaking the model chosen for syntax.
\end{frame}

\section{Category-theoretical type system}
\subsection{Type system}
\begin{frame}
	\frametitle{Notations}
	Let $\mL$ be our language.
	Our only requirement is that words can be \textit{applied} to one another in
	some sense in the	denotation system.

	\pause
	\medskip

	Let $\mC$ be a cartesian closed category used for typing the lexicon and
	$\mathcal{F}(\mL)$ be a set of functors used for representing the words
	that add an effect to our language.

	\pause
	\smallskip

	We consider $\bar{\mC}$ the categorical closure of $\mC$ under the action
	of $\mathcal{F}(\mL)^{*}$.
	We close it for the cartesian product and exponential of $\mC$.

	$\bar{\mC}$ represents all possible combinations of a sequence of effects
	and a base type, and contains functions and products.
\end{frame}

\begin{frame}[fragile]
	\frametitle{Intuitionistic-style Typing Judgements}
	\only<1-2>{We then have typing judgements for basic combinations:
		\begin{align*}
			\only<1>{
			\frac{\cont x: \tau \poulpe \cont F \in \mathcal{F}(\mL)}{\cont Fx: F\tau }\fracnotate{Cons}                                          \\[.25cm]
				\frac{\cont x: F\tau_{1} \poulpe \cont \phi: \tau_{1} \to \tau_{2}}{\cont \phi x: F\tau_{2} }\fracnotate{\texttt{fmap}}
			}
			\only<2>{
				\frac{\cont x: \tau_{1} \poulpe \cont \phi: \tau_{1} \to \tau_{2}}{\cont \phi x: \tau_{2}}\fracnotate{App}
			\frac{\cont x: A\tau_{1} \poulpe \cont \phi: A\left( \tau_{1} \to \tau_{2} \right)}{\cont \phi x: A\tau_{2}}\fracnotate{\texttt{<*>}} \\[.25cm]
			}
		\end{align*}
	}
	\only<3>{Typing judgements for natural transformations:
		\begin{align*}
			\frac{\cont x: \tau}{\cont x: A\tau}\fracnotate{\texttt{pure/return}} \\[.25cm]
			\frac{\cont x: MM\tau}{\cont x: M\tau}\fracnotate{\texttt{>>=}}
		\end{align*}
		\begin{align*}
			\forall F \overset{\theta}{\Longrightarrow} G,\poulpe \frac{\cont x: F\tau \poulpe \cont G: S' \subseteq \star \poulpe \tau \in S'}{\cont x : G\tau}\fracnotate{\texttt{nat}}
		\end{align*}
	}
\end{frame}

\subsection{Introducing a language}

\begin{frame}
	\frametitle{Information}
	To present the language, we of course need the syntax of the language, as
	well as an increased model of our lexicon.
\end{frame}

\begin{frame}[fragile]
	\frametitle{Updated Lexicon}
	\resizebox{\textwidth}{!}{
		\setcellgapes{3pt}
		\makegapedcells
		\begin{NiceTabular}{>{\bf}LLL}
			Expression & \rm Type & \lambda\text{-Term} \\
			\word{it}{\f{G}\e}{\lambda g. g_{0}}{}
			\word{\cdot\w{, a} \cdot}{\e \to \left(\e \to \t\right) \to \f{W}\e}{\lambda x. \lambda p. \scalar{x, p x}}{}
			\word{which}{\left( \e \to \t \right)\to \f{S}\e}{\lambda p. \left\{x \suchthat px\right\}}{}
			\word{the}{\left( \e \to \t \right) \to \f{M}\e}{\lambda p. x \text{ if } p^{-1}\left( \top \right) = \{x\} \text{ else } \#}{}
			\word{a}{\left( \e \to \t \right) \to \f{D}\e}{\lambda p. \lambda s. \left\{ \scalar{x, x \ppl s}\suchthat p x\right\}}{}
			\word{every}{\left( \e \to \t \right)\to \f{C}\e}{\lambda p. \lambda c. \forall x, px \Rightarrow cx}{}
			\CodeAfter
			\begin{tikzpicture}
				\draw[double] (1|-2) -- (4|-2);
				\foreach \r in {3,...,8} {\draw (1|-\r) -- (4|-\r);}
			\end{tikzpicture}
		\end{NiceTabular}
	}
\end{frame}

\begin{frame}
	\frametitle{Introducing Higher-Order Constructs}
	\only<1-2>{Using the notion of functors, we can also implement higher-order semantic
		constructions in our lexicon, such as the future, without caring about
		morphological markers:}
	\only<2>{%
		\begin{equation*}
			\columneqs{
				\bf future\left(be\left( I, a\ cat \right)\right)
				\xrightarrow{\beta} be\left( future\left( I \right), a\ cat \right)
				\xrightarrow{\beta} be\left( future\left( I \right), a\ cat \right)
			}
		\end{equation*}
		Those constructs are integrated by using natural transformations explaining
		their propagation through other effects, as those are purely
		semantic predicates.
	}
	\only<3>{
		For the plural, this gives:
		\resizebox{\textwidth}{!}{%
			\def\arraystretch{1.3}
			\setcellgapes{3pt}
			\makegapedcells
			\begin{NiceTabular}{>{\bf}c>{\cont}C>{\Pi(p) = }L}
				CN(P)                       & p: \left(\e \to \t\right)                                & \lambda x.\left( px \land \abs{x} \geq 2 \right)                                                \\
				\multirow{2}{*}{\bf ADJ(P)} & p: \left( \e \to \t \right)                              & \lambda x. \left( px \land \abs{x} \geq 2 \right)                                               \\
				                            & p: \left( \e \to \t \right) \to \left( \e \to \t \right) & \lambda \nu. \lambda x. \left( p\left( \nu \right)\left( x \right) \land \abs{x} \geq 2 \right) \\
				\multirow{2}{*}{\bf NP}     & p: \e                                                    & p                                                                                               \\
				                            & p: \left( \e \to \t \right) \to \t                       & \lambda \nu. p\left( \Pi \nu \right)                                                            \\
				IV(P)/VP                    & p: \e \to \t                                             & \lambda o. \left( po \land \abs{x} \geq 2 \right)                                               \\
				TV(P)                       & p: \e \to \e \to \t                                      & \lambda s. \lambda o. \left( p\left( s \right)\left( o \right) \land \abs{s} \geq 2 \right)     \\
				\CodeAfter
				\begin{tikzpicture}
					\foreach \r in {2,...,7} {\draw[dashed] ($(2|-\r) + (.1, 0)$) -- (4|-\r);}
					\foreach \r in {2, 4, 6, 7} {\draw (1|-\r) -- (4|-\r);}
				\end{tikzpicture}
			\end{NiceTabular}
		}
	}
\end{frame}

\begin{frame}
	\frametitle{Ambiguity}
	\resizebox{\textwidth}{!}{\begin{tikzpicture}
		\wnode{2}{sees}{\left( \e \to \e \to \t \right)}
		\wnode{3}{the}{\left( \e \to \t \right) \to \f{M}\e}
		\wnode{4}{girl}{\left( \e \to \t \right)}
		\dtnode{\f{M}\e}{the-}{girl-}
		\wnode{5}{using}{\left( \e \to \t \right) \to \e \to \left( \e \to \t \right)}
		\wnode{6}{a}{\left( \e \to \t \right) \to \f{D}\e}
		\wnode{7}{telescope}{\left( \e \to \t \right)}
		\utnode{\f{D}\e}{a}{telescope}
		\dtnode{\f{D}\e}{a-}{telescope-}
		\utnode{\f{D}\left(\left( \e \to \t \right) \to \left( \e \to \t \right)\right)}{using}{atelescope}
		\dtnode{\f{D}\left(\left( \e \to \t \right) \to \left( \e \to \t \right)\right)}{using-}{a-telescope-}
		\dtnode{\f{M}\left( \e \to \t \right)}{sees-}{the-girl-}
		\dtnode{\f{D}\f{M}\left( \e \to \t \right)}{sees-the-girl-}{using-a-telescope-}
		\utnode{\f{D}\left( \e \to \t \right)}{girl}{usingatelescope}
		\utnode{\f{M}\f{D}\e}{the}{girlusingatelescope}
		\utnode{\f{M}\f{D}\left( \e \to \t \right)}{sees}{thegirlusingatelescope}
		%
		\wnode{0}{the}{\left( \e \to \t \right) \to \f{M}\e}
		\wnode{1}{man}{\left( \e \to \t \right)}
		\utnode{\f{M}\e}{the}{man}
		\utnode{\f{M}\f{M}\f{D}\t}{theman}{seesthegirlusingatelescope}
		\utnode{\f{M}\f{D}\t}{themanseesthegirlusingatelescope}{themanseesthegirlusingatelescope}
		%
		\dtnode{\f{M}\e}{the-}{man-}
		\dtnode{\f{D}\f{M}\f{M}\t}{the-man-}{sees-the-girl-using-a-telescope-}
		\dtnode{\f{D}\f{M}\t}{the-man-sees-the-girl-using-a-telescope-}{the-man-sees-the-girl-using-a-telescope-}
		%
		%
	\end{tikzpicture}
}
\end{frame}

\section{Effect Handling}
\subsection{Notion and usage of handlers}
\begin{frame}
	\frametitle{Handlers}
	A handler for an effect $F$ is a natural transformation $F \Rightarrow \Id$,
	as proposed in \cite{wuEffectHandlersScope2014}.

	\smallskip

	Handlers should also be exact inverses to monadic and applicative units:
	this partially justifies semantically why we can remove the usage of the
	unit rule out	of certain situations.
\end{frame}


\begin{frame}
	\frametitle{Defining Handlers}
	There are two main types of handlers that are of interest to us:
	\pause
	\begin{enumerate}
		\item Language-Defined Handlers, which are defined with
		      adjunctions and comonads, for example.
		      Those arise from fundamental properties of the considered effects.
		      \pause
		\item Speaker-dependant handlers, which are considered when
		      retrieving the denotation from a sentence from under the effects
		      that arose in the computation of its meaning.
		      Those need to be considered dependent on the speaker because for
		      example of the multiple ways to solve non-determinism.
	\end{enumerate}
\end{frame}

\begin{frame}
	\frametitle{Scope Islands}
	The notion of handlers allows us to enforce the notion of scope islands.
	To do so, it would suffice to ask that the words enclosing the island,
	are not defined on not certain effectful types and make handlers a part of
	the combination modes, as introduced in
	\cite{bumfordEffectdrivenInterpretationFunctors2025}.

	\pause
	\smallskip

	We would for example have:
	\begin{equation*}
		\w{if}: \left(\t \setminus \mF{\mL}^{*}\f{C}\t \right) \to \t \to \t
	\end{equation*}

\end{frame}
\subsection{String diagrams for effect handling}
\begin{frame}[allowframebreaks]
	\frametitle{String Diagrams Representation of Sentences}
	Here, a string diagram is a representation of the side-effects and types of a
	sentence across its computation.

	\begin{center}
		\begin{minipage}[c]{.45\textwidth}
			\inputtikz{sd-thecatsleeps}
		\end{minipage}
		\begin{minipage}[c]{.45\textwidth}
			This diagram for example represents the sentence \textsl{The cat sleeps}.
			The order of the words and position of the strings will be explained in
			detail in the next section.
		\end{minipage}
	\end{center}

\end{frame}

\subsection{Reducing in string diagrams}
\begin{frame}
	\frametitle{Deformation of String Diagrams}
	String diagrams will be the formalism we use to compute equality between
	denotations, and especially handling the denotations.
	\begin{thm}[Theorem 3.1 \cite{selingerSurveyGraphicalLanguages2010}, Theorem 1.2 \cite{joyalGeometryTensorCalculus1991}]
		\label{thm:isotopy}
		A well-formed equation between morphism terms in the language of monoidal
		categories follows from the axioms of monoidal categories if and only if it
		holds, up to planar isotopy, in the graphical language.
	\end{thm}
\end{frame}

\begin{frame}[allowframebreaks]
	\frametitle{Equations on String Diagrams}
	Every property of the functors, monads, natural transformations, adjunctions
	and more can be explained in terms of commutative diagrams, but also as
	string diagrams.

	First, the elevator equations are a consequence of \ref{thm:isotopy}:
	\begin{equation}
		\resizebox{.8\textwidth}{!}{\inputtikz{sd-elevator}}
		\tag{\emoji{elevator}}
	\end{equation}

	The Snake equations are a rewriting of the properties of an adjunction:
	\begin{equation}
		\resizebox{.5\textwidth}{!}{\inputtikz{sd-snake1}}
		\tag{\rotatebox[origin=c]{90}{\emoji{snake}}}
	\end{equation}

	The Monadic equations are a rewriting of the properties of a monad:
	\begin{equation}
		\resizebox{.5\textwidth}{!}{\inputtikz{sd-monad-mult}}
		\tag{$\mu$}
	\end{equation}

\end{frame}

\begin{frame}[allowframebreaks]
	\frametitle{Confluence of Reductions}
	\begin{thm}[Confluence]\label{thm:confluence}
		Our reduction system is confluent and therefore defines normal forms:
		\begin{enumerate}
			\item Right reductions are confluent and therefore define \emph{right} normal forms for
			      diagrams under the equivalence relation induced by exchange.
			\item Equational reductions are confluent and therefore define \emph{equational}
			      normal forms for diagrams under the equivalence relation induced by exchange.
		\end{enumerate}
	\end{thm}

	\smallskip

	\begin{thm}[Normalization Complexity]
		\label{thm:normalize}
		Reducing a diagram to its normal form is done in quadratic time in
		the number of natural transformations in it.
	\end{thm}
	This is accomplished using a formalism based on \cite{delpeuchNormalizationPlanarString2022}.
\end{frame}

\section{Semantic Parsing}
\subsection{The general method}
\begin{frame}[allowframebreaks]
	\frametitle{CFG Model of Parsing}
	We use a Context-Free Grammar to model our typing system and take its product with the syntax defining grammar.

	\medskip

	\begin{minipage}{.9\textwidth}
		\begin{multicols}{2}
			\def\arraystretch{1.1}
			\begin{mgrammar}
				\firstrule{>, \beta}{\left(\alpha\to \beta\right), \alpha}{}
				\firstrule{<, \beta}{\alpha, \left(\alpha \to \beta\right)}{}
				\gskip
				\firstrule{\combJ_{\f{F}}\  \f{F}\tau}{\f{F}\f{F}\tau}{}
				\firstrule{\combDN_{\f{C}}\  \tau}{\f{C}_{\tau}\tau}{}
				\gskip
				\firstrule{\combML_{\f{F}} \left(\alpha, \beta\right)}{\f{F}\alpha, \beta}{}
				\firstrule{\combMR_{\f{F}} \left(\alpha, \beta\right)}{\alpha, \f{F}\beta}{}
			\end{mgrammar}

			\def\arraystretch{1.3}
			\begin{mgrammar}
				\firstrule{\combA_{\f{F}} \left(\alpha, \beta\right)}{\f{F}\alpha, \f{F}\beta}{}
				\firstrule{\combUR_{\f{F}} \left(\alpha \to \alpha', \beta\right)}{\f{F}\alpha\to \alpha', \beta}{}
				\firstrule{\combUL_{\f{F}} \left(\alpha, \beta\to \beta'\right)}{\alpha, \f{F}\beta \to \beta'}{}
				\firstrule{\combC_{\f{L}\f{R}} \left(\f{L} \alpha, \f{R}\beta\right)}{\left(\alpha, \beta\right)}{}
				\firstrule{\combER_{\f{R}} \left(\f{R}\left(\alpha \to \alpha'\right), \beta\right)}{\alpha\to \f{R}\alpha', \beta}{}
				\firstrule{\combEL_{\f{R}} \left(\alpha, \f{R}\left(\beta \to \beta'\right)\right)}{\alpha, \beta\to \f{R}\beta'}{}
			\end{mgrammar}
		\end{multicols}
	\end{minipage}

	\medskip

	This grammar works in five major sections:
	\begin{enumerate}
		\item We reintroduce the grammar defining the type and effect system.
		\item We introduce a structure for the semantic parse trees and their labels,
		      the combination modes from
		      \cite{bumfordEffectdrivenInterpretationFunctors2025}.
		\item We introduce rules for basic type combinations.
		\item We introduce rules for higher-order unary type combinators.
		\item We introduce rules for higher-order binary type combinators.
	\end{enumerate}
	Computationally, the introduction of this labeling system increases the size
	of the grammar by a factor linear in $\abs{\mF\left(\mL\right)}$.
	The parsing algorithms are then still polynomial, and scaling in sentence
	size does not change.
\end{frame}

\begin{frame}[allowframebreaks, fragile]
	\frametitle{Semantic Parse Trees}
	\includegraphics{aux/figures/parse-tree-1.pdf}
	\centering
	\includegraphics{aux/figures/parse-tree-2.pdf}

	{\centering \includegraphics[height=.7\pageheight]{aux/figures/parse-tree-3.pdf}}
\end{frame}

\subsection{String diagrams for parsing}
\begin{frame}
	\frametitle{Combinators as String Diagrams}
	\includegraphics[width=\textwidth]{aux/figures/combinators-sd.pdf}
\end{frame}

\begin{frame}
	\frametitle{A Parsing Diagram Step}
	\rotatebox{-90}{\includegraphics[height=\textwidth]{aux/figures/parsing-diagram.pdf}}
\end{frame}

\begin{frame}
	\frametitle{A Parsing Diagram Step}
	\rotatebox{-90}{\includegraphics[height=\textwidth]{aux/figures/parsing-diagram.pdf}}
\end{frame}

\begin{frame}
	\frametitle{Building an Intuition}
	\centering
	\includegraphics[width=.7\textwidth]{aux/figures/knitting-example.jpeg}
\end{frame}

\begin{frame}
	\frametitle{A Full Parsing Diagram}
	\vspace{-.25cm}
	\centering
	\rotatebox{-90}{\includegraphics[width=.7\pageheight]{aux/figures/3d-parsing-diagram.pdf}}
\end{frame}

\subsection{String diagrams for reductions}
\begin{frame}
	\frametitle{Reducing the grammar}
	We introduce denotations for our combinators, to allow us to define
	reductions that relieve a bit the ambiguity of the parsing grammar:
	\begin{center}
		$	>                    = \lambda \phi. \lambda x. \phi x $\\[1.5ex]
		$ \combML_{\f{F}}      = \lambda M. \lambda x. \lambda y. (\fmap_{\f{F}} \lambda a. M(a, y)) x $\\[1.5ex]
		$	\combA_{\f{F}}       = \lambda M. \lambda x. \lambda y. (\fmap_{\f{F}}\lambda a. \lambda b. M(a, b))(x) \texttt{<*>} y $\\[1.5ex]
		$	\combUL_{\f{F}}      = \lambda M. \lambda x. \lambda \phi. M(x, \lambda b. \phi(\eta_{\f{F}} b))$\\[1.5ex]
		$ \combJ_{\f{F}}       = \lambda M. \lambda x. \lambda y. \mu_{\f{F}} M(x, y) $\\[1.5ex]
	\end{center}
\end{frame}

\begin{frame}
	\frametitle{Reductions}
	Reductions include the following:
	\begin{itemize}
		\item When two effects commute, we choose an order to apply them.
		\item We use $\combUR$ instead of using $\combMR$ or $\combDN \combMR$ when
		      possible.
		\item We always use modes $\combJ$, $\combC$ and $\combDN$ as early as
		      possible.
		      The fact this works is a consequence of Theorem \ref{thm:isotopy} on
		      the denotation diagrams.
		\item All the rules provided in the previous section can be rewritten here
		      too.
	\end{itemize}
\end{frame}

\begin{frame}
	\frametitle{Conclusion}
	We have formalised an enhancement of usual type systems for natural language
	semantics.

	\medskip

	Thanks to the introduction of the string diagrams, this did not come at the
	cost of comprehension of the system nor efficiency.

	\pause\medskip

	I would like to thank Simon Charlow for his advice, my mother for the
	knitting, Antoine Groudiev for the rotation of the snakes in equation labels,
	Bella Senturia, Bob Frank and Paul-André Melliès for their suggestions of
	papers to read about categories and linguistics.
\end{frame}

\questionsframe

\appendix

\begin{frame}
	\frametitle{Functor Denotations}
	\resizebox{\textwidth}{!}{\def\arraystretch{1.3}
\setcellgapes{3pt}
\makegapedcells
\begin{NiceTabular}{LLc}
	\rm Constructor                                                          & \fmap                                                                                                                                                             & Typeclass \\
	\f{G}\left( \tau \right) = \r \to \tau                                   & \f{G}\phi\left( x \right) = \lambda r. \phi \left(x r\right)                                                                                                      & Monad     \\
	\f{W}\left( \tau \right) = \tau \times \t                                & \f{W}\phi\left( \scalar{a, p} \right) = \scalar{\phi a, p}                                                                                                        & Monad     \\
	\f{S}\left( \tau \right) = \{ \tau \}                                    & \f{S}\phi\left( \left\{ x \right\} \right) = \left\{ \phi(x) \right\}                                                                                             & Monad     \\
	\f{C}\left( \tau \right) = \left( \tau \to \t \right) \to \t             & \f{C}\phi\left( x \right) = \lambda c. x\left( \lambda a. c \left( \phi a \right) \right)                                                                         & Monad     \\
	\f{T}\left( \tau \right) = \ty{s} \to \left( \tau \times \ty{s} \right)  & \f{D}\phi\left( \lambda s. \left\{ \scalar{x, x \ppl s} \suchthat p x \right\} \right) = \lambda s. \scalar{\phi x, \phi x \ppl s}                                & Monad     \\
	\f{F}\left( \tau \right) = \tau \times \f{S}\tau                         & \f{F}\phi\left( \scalar{v, \left\{ x \suchthat x \in D_{e} \right\}} \right) = \scalar{\phi\left(v\right), \left\{ x \suchthat x \in D_{e} \right\}}              & Monad     \\
	\f{D}\left( \tau \right) = \ty{s} \to \f{S}\left(\e \times \ty{s}\right) & \f{D}\phi\left( \lambda s. \left\{ \scalar{x, x \ppl s} \suchthat p x \right\} \right) = \lambda s. \left\{ \scalar{\phi x, \phi x \ppl s} \suchthat p x \right\} & Monad     \\
	\f{M}\left( \tau \right) = \tau + \bot                                   & \f{M}\phi\left( x \right) = \begin{cases}
		                                                                                                       \phi\left( x \right) & \text{if } \cont x: \tau \\
		                                                                                                       \#                   & \text{if } \cont x: \#
	                                                                                                       \end{cases}                                                                                                                & Monad                \\
	\CodeAfter
	\begin{tikzpicture}
		\draw[double] (1|-2) -- (4|-2);
		\foreach \r in {3,...,9} {\draw (1|-\r) -- (4|-\r);}
	\end{tikzpicture}
\end{NiceTabular}
}
\end{frame}

\begin{frame}
	\frametitle{CFG of English}
	\centering
	\small
	\begin{minipage}{.45\textwidth}
		\setlength\tabcolsep{4pt}
		\begin{tabular}{>{\tt}l r >{\tt}l r}
			\firstrule{CP}{DP, VP}{}
			\grule{Cmp, CP}{}
			\grule{CP, CBar}{}
			\gskip
			\firstrule{CBar}{Cor, CP}{}
			\gskip
			\firstrule{Dbar}{Cor, DP}{}
			\gskip
			\firstrule{DP}{DP, Dbar}{}
			\grule{Dmp, DP}{}
			\grule{Det, NP}{}
			\grule{Gen, TN}{}
			\gskip
			\firstrule{Gen}{DP, GenD}{}
		\end{tabular}
	\end{minipage}
	\begin{minipage}{.45\textwidth}
		\setlength\tabcolsep{4pt}
		\begin{tabular}{>{\tt}l r >{\tt}l r}
			\firstrule{NP}{AdjP, NP}{}
			\grule{NP, AdjP}{}
			\gskip
			\firstrule{AdjP}{TAdj, DP}{}
			\grule{Deg, AdjP}{}
			\gskip
			\firstrule{VP}{TV, DP}{}
			\grule{AV, CP}{}
			\grule{VP, AdvP}{}
			\gskip
			\firstrule{TV}{DV, DP}{}
			\gskip
			\firstrule{AdvP}{TAdv, DP}{}
		\end{tabular}
	\end{minipage}

\end{frame}

\begin{frame}
	\frametitle{Combinator Denotations}
	\scriptsize
	\centering
	$	>                    = \lambda \phi. \lambda x. \phi x $\\[1.5ex]
	$ <                    = \lambda x. \lambda \phi. \phi x $\\[1.5ex]
	$ \combML_{\f{F}}      = \lambda M. \lambda x. \lambda y. (\fmap_{\f{F}} \lambda a. M(a, y)) x $\\[1.5ex]
	$ \combMR_{\f{F}}      = \lambda M. \lambda x. \lambda y. (\fmap_{\f{F}} \lambda b. M(x, b)) y $\\[1.5ex]
	$	\combA_{\f{F}}       = \lambda M. \lambda x. \lambda y. (\fmap_{\f{F}}\lambda a. \lambda b. M(a, b))(x) \texttt{<*>} y $\\[1.5ex]
	$	\combUL_{\f{F}}      = \lambda M. \lambda x. \lambda \phi. M(x, \lambda b. \phi(\eta_{\f{F}} b))$\\[1.5ex]
	$	\combUR_{\f{F}}      = \lambda M. \lambda \phi. \lambda y. M(\lambda a. \phi(\eta_{\f{F}} a),y) $\\[1.5ex]
	$ \combJ_{\f{F}}       = \lambda M. \lambda x. \lambda y. \mu_{\f{F}} M(x, y) $\\[1.5ex]
	$\combC_{\f{L}\f{R}}  = \lambda M. \lambda x. \lambda y. \epsilon_{\f{L}\f{R}}(\fmap_{\f{L}}(\lambda l. \fmap_{\f{R}}(\lambda r. M(l, r))(y)) (x))$\\[1.5ex]
	$	\combEL_{\f{R}}      = \lambda M. \lambda \phi. \lambda y. M(\Upsilon_{\f{R}} \phi, y)$\\[1.5ex]
	$	\combER_{\f{R}}      = \lambda M. \lambda x. \lambda \phi. M(x, \Upsilon_{\f{R}} \phi)$\\[1.5ex]
	$	\combDN_{\Downarrow} = \lambda M. \lambda x. \lambda y. \Downarrow M(x, y)$
\end{frame}
\end{document}

