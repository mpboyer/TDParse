\section{Categorical Semantics of Effects: A Typing System}
\label{sec:typingsystem}
In this section, we will designate by $\mL$ our language, as a set of words
(with their semantics) and syntactic rules to combine them semantically.
We denote by $\O\left( \mL \right)$ the set of words in the language whose
semantic representation is a low-order function and $\mF\left( \mL \right)$ the
set of words whose semantic representation is a functor or high-order function.
Our goals here are to describe more formally, using a categorical vocabulary,
the environment in which the typing system for our language will exist, and how
we connect words and other linguistic objects to the categorical formulation.

\subsection{Typing Category}\label{subsec:typingcategory}
\subsubsection{Types}\label{subsubsec:types}
Let $\mC$ be a closed cartesian category.
This represents our main typing system, consisting of words $\O(\mL)$ that can
be expressed without effects.
Remember that $\mC$ contains a terminal object $\bot$ representing the empty
type or the lack thereof.
We can consider $\bar{\mC}$ the category closure of
$\mF\left( \mL \right)\left(\O\left( \mL \right)\right)$, that is consisting of
all the different type constructors (ergo, functors) that could be formed in
the language.
What this means is that we consider for our category objects any object that
can be attained in a finite number from a finite number of functorial
applications from an object of $\mC$.
In that sense, $\bar{\mC}$ is still a closed cartesian category (since all our
functors induce isomorphisms on their image)\footnote{Our definition does not
	yield a closed cartesian category as there are no exponential objects between
	results of two different functors, but this is not a real issue as we can just
	say we add those.}.
$\bar{\mC}$ will be our typing category (in a way).

We consider for our types the quotient set
$\star = \mathrm{Obj}\left( \bar{\mC} \right)/\mF\left( \mL \right)$.
Since $\mF\left( \mL \right)$ does not induce an equivalence relation on
$\Obj\left( \bar{\mC} \right)$ but a preorder, we consider the chains obtained
by the closure of the relation $x\succeq y \Leftrightarrow \exists F, y = F(x)$
(which is seen as a subtyping relation as proposed in
\newcite{melliesFunctorsAreType2015}).
We also define $\star_{0}$ to be the set obtained when considering types which
have not yet been \emph{affected}, that is $\Obj(\mC)$.
In contexts of polymorphism, we identify $\star_{0}$ to the adequate subset of
$\star$.
In this paradigm, constant objects (or results of fully done computations) are
functions with type $\bot \to \tau$ which we will denote directly by
$\tau \in \star_{0}$.

What this construct actually means, in categorical terms, is that a type is an
object of $\bar{\mC}$ (as intended) but we add a subtyping relationship based
on the procedure used to construct $\bar{\mC}$.
Note that we can translate that subtyping relationship on functions as
$F\left( A \xrightarrow{\phi} B \right)$ has types
$F\left( A\Rightarrow B \right)$ and $FA \Rightarrow FB$.

We will provide in Figure \ref{fig:sctypes} a list of the effect-less usual
types associated to regular linguistic objects.

\subsubsection{Functors, Applicatives and Monads}
\label{subsubsec:functors}
Our point of view leads us to consider \emph{language functors}\footnote{The
	elements of our language, not the categorical construct.} as polymorphic
functions: for a (possibly restrained, though it seems to always be $\star$)
set of base types $S$, a functor is a function
\begin{equation*}
	x: \tau\in S\subseteq \star \mapsto F x: F\tau
\end{equation*}
Remember that $\star$ is a fibration of the types in $\bar{\mC}$.
This means that if a functor can be applied to a type, it can also be applied
to all \emph{affected} versions of that type, i.e.
$\mF\left( L \right)(\tau\in \star)$.
More importantly, while it seems that $F$'s type is the identity on $\star$,
the important part is that it changes the effects applied to $x$ (or $\tau$).
In that sense, $F$ has the following typing judgements:
\begin{equation*}
	\frac{\Gamma\vdash x: \tau \in \star_{0}}{\Gamma\vdash F x: F\tau \notin
		\star_{0}}\fracnotate{$\text{Func}_{0}$} \hspace{2cm} \frac{\Gamma\vdash x:
		\tau}{\Gamma\vdash Fx : F\tau\preceq \tau}\fracnotate{Func}
\end{equation*}
We use the same notation for the \emph{language functor} and the
\emph{type functor} in the examples, but it is important to note those are two
different objects, although connected.
More precisely, the \emph{language functor} is to be seen as a function whose
computation yields an effect, while the \emph{type functor} is the endofunctor
of $\bar{\mC}$ (so a functor from $\mC$) that represents the effect in our
typing category.

In the same fashion, we can consider functions to have a type in $\star$ or
more precisely of the form $\star \to \star$ which is a subset of $\star$.
This justifies how functors can act on functions in our typing system, thanks
to the subtyping judgement introduced above, as this provides a way to ensure
proper typing while just propagating the effects.
Because of propagation, this also means we can resolve the effects or keep on
with the computation at any point during parsing, without any fear that the
results may differ.

\medskip

In that sense, applicatives and monads only provide with more flexibility on
the ways to combine functions:
they provide intermediate judgements to help with the combination of trees.
For example, the multiplication of the monad provides a new
\emph{type conversion} judgement:
\begin{equation*}
	\frac{\Gamma\vdash x: MM\tau}{\Gamma\vdash x: M\tau \succeq MM\tau}
	\fracnotate{Monad}
\end{equation*}
This is actually a special case of the natural transformation rule that we
define below, which means that, in a way, types $MM\star$ and $M\star$ are
equivalent, as there is a canonical way to go from one type to another.
Remember however that $M\star$ is still a proper subtype of $MM\star$ and that
the objects are not actually equal: they are simply equivalent.

\subsubsection{Natural Transformations}
\label{subsubsec:transnat}
We could also add judgements for adjunctions, but the most interesting thing is
to add judgements for natural transformations, as adjunctions are particular
examples of natural transformations which arise from \emph{natural} settings.
While in general we do not want to find natural transformations, we want to be
able to express these in three situations:
\begin{enumerate}
	\item If we have an adjunction $L\dashv R$, we have natural transformations
	      for $\Id_{\mC} \Rightarrow L \circ R$ and $R\circ L \Rightarrow \Id_{\mC}$.
	      In particular we get a monad and a comonad from a canonical setting.
	\item To deal with the resolution of effects, we can map handlers to natural
	      transformations which go from some functor $F$ to the $\Id$ functor,
	      allowing for a sequential\footnote{In particular, non-necessarily
		      commutative} computation of the effects added to the meaning.
	      We will develop a bit more on this idea in Paragraph \ref{par:handlers}
	      and in Section \ref{sec:nondet}.
	\item To create \emph{higher-order} constructs which transform words from our
	      language into other words, while keeping the functorial aspect.
	      This idea is developed in \ref{par:higherorder}.
\end{enumerate}
To see why we want this rule, which is a larger version of the monad
multiplication and the monad/applicative unit, it suffices to see that the
diagram below provides a way to construct the ``correct function'' on the
``correct functor'' side of types. If we have a natural transformation
\begin{tikzcd}
	F\ar[r, Rightarrow, "\theta"] & G
\end{tikzcd}
then for all arrows $f: \tau_{1} \to \tau_{2}$ we have:
\begin{category}
	F\tau_{1}\ar[r, "Ff"]\ar[d, "\theta_{\tau_{1}}"'] &
	F\tau_{2}\ar[d, "\theta_{\tau_{2}}"]\\
	G\tau_{1}\ar[r, "Gf"] & G\tau_{2}
\end{category}
and this implies, from it being true for all arrows, that from $\cont x: F\tau$
we have an easy construct to show $\cont x: G\tau$.

Remember that in the Haskell programming language, any polymorphic function is
a natural transformation from the first type constructor to the second type
constructor, as proved by \newcite{wadlerTheoremsFree1989}.
This will guarantee for us that given a \emph{Haskell} construction for a
polymorphic function, we will get the associated natural transformation.

\paragraph{Adjunctions}
\label{par:adjunctions}
We will not go in much details about adjunctions, as a full example and
generalization process is provided in \ref{subsec:effects}.
First, we remind the definition: an adjunction $L \dashv R$ is a pair of
functors $L: \A \to \B$ and $R: \B \to \A$, and a pair of natural
transformations $\eta: \Id_{\A}  \Rightarrow R \circ L$ and
$\epsilon: L\circ R \Rightarrow \Id_{\A}$ such that the two following equations
are satisfied:
\begin{equation}
	\inputtikz{cd-zigzag}
	\tag{\emoji{cloud-with-lightning}}
	\label{eq:zigzag}
\end{equation}
\begin{equation}
	\inputtikz{cd-zagzig}
	\label{eq:zagzig}
	\tag{\reflectbox{\emoji{cloud-with-lightning}}}
\end{equation}

An adjunction defines two different structures over itself: a monad $L \circ R$
and a comonad $R\circ L$.
The fact these structures arise from the interaction between two effects
renders them an intrinsic property of the language.
In this lies the usefulness of adjunction in a typing system which uses
effects: adjunctions provide a way to combine effects and to handle effects,
allowing to simplify the computations on the free monoid on the set of functors.

\paragraph{Handlers}
\label{par:handlers}
As introduce by \newcite{marsikAlgebraicEffectsHandlers}, the use of handlers
as annotations to the syntactic tree of the sentence is an appropriate
formalism.
This could also give us a way to construct handlers for our effects as per
\newcite{bauerEffectSystemAlgebraic2014}, or
\newcite{plotkinHandlingAlgebraicEffects2013}.
As considered by \newcite{wuEffectHandlersScope2014} and
\newcite{vandenbergFrameworkHigherorderEffects2024}, handlers are to be seen
as natural transformations describing the free monad on an algebraic effect.
Considering handlers as so, allows us to directly handle our computations
inside our typing system, by ``transporting'' our functors one order higher up
without loss of information or generality since all our functors undergo the
same transformation.
Using the framework proposed in \cite{vandenbergFrameworkHigherorderEffects2024}
we simply need to create handlers for our effects/functors and we will then
have in our language the result needed.
The only thing we will require from an algebraic handler $h$ is that for any
applicative functor of unit $\eta$, $h\circ \eta = \id$.

\medskip

What does this mean in our typing category?
It means that either our language or our parser for the language
\footnote{Depending whether we think handling effects is an intrinsic construct
	of the semantics of the language or whether it is associated with a speaker.}
should contain natural transformations $F \Rightarrow \Id$ for
$F \in \mF\left( \mL \right)$.
In this goal, we remember that from any polymorphic function in \emph{Haskell}
we get a natural transformation \cite{wadlerTheoremsFree1989} meaning that it
is enough to be able to define our handler in \emph{Haskell} to be ensured of
its good definition.
Note that the choice of the handler being part of the lexicon or the parser
over the other is a philosophical question more than a semantical one, as both
options will result in semantically equivalent models, the only difference will
be in the way we consider the resolution of effects.
Mathematically\footnote{Computer-wise, actually.}, this means the choice of
either one of the options is purely of detail left during the implementation.

However, this choice does not arise in the case of the adjunction-induced
handlers.
Indeed here, the choice is caused by the non-uniqueness of the choices for
the handlers.
For example, two different speakers may have different ways to resolve the
$\f{S}$ (which will be introduced as the \emph{Powerset} monad in Section
\ref{subsec:effects}) that arises from the phrase \textsl{A chair}.
This usual example of the differences between the cognitive representation of
words is actually a specific example of the different possible handlers for the
powerset representation of non-determinism/indefinites:
there are $\abs{S}$ arrows from the initial object to $S$ in $\mathit{Set}$,
representing the different elements of $S$.
In that sense, while handlers may have a normal form or representation purely
dependant on the effect, the actual handler does not necessarily have
a canonical form.
This is the difference with the adjunctions: adjunctions are intrinsic
properties of the coexistence of the effects\footnote{Which we identify to
	their functorial representations, which we may identify to their free monad in
	the framework \cite{vandenbergFrameworkHigherorderEffects2024}.}, while the
handlers are user-defined.
As such, we choose to say that our handlers are implemented parser-side but
again, this does not change our modelisation of handlers as natural
transformations and most importantly, this does not add non-determinism to our
model: The non-determinism that arises from the variety of possible handlers
does not add to the non-determinism in the parsing.

\paragraph{Higher-Order Constructs}
\label{par:higherorder}
We might want to add plurals, superlatives, tenses, aspects and other similar
constructs which act as function modifiers.
For each of these, we give a functor $\Pi$ corresponding to a new class of
types along with natural transformations for all other functors $F$ which
allows to propagate down the high-order effect.
This transformation will need to be from $\Pi \circ F$ to
$\Pi \circ F \circ \Pi$ or simply $\Pi \circ F \Rightarrow F \circ \Pi$
depending on the situation.
This allows us to add complexity not in the compositional aspects but
in the lexicon aspects.

One of the main issues with this is the following:
In the English language, plural is marked on all words (except verbs, and even
then case could be made for it to be marked), while future is marked only on
verbs (through the \textit{will + verb} construct which creates a ``\emph{new}''
verb in a sense) though it applies also to the arguments on the verb.
A way to solve this would be to include in the natural transformations rules to
propagate the functor depending on the type of the object.
Consider the superlative effect \textbf{most}\footnote{We do not care about
	morphological markings here, we say that
	$\mathbf{largest} = \mathbf{most} \left(\mathbf{large}\right)$}.
As it can only be applied on adjectives, we can assume its argument is a
function (but the definition would hold anyway taking $\tau_{1} = \bot$).
It is associated with the following function (which is a rewriting of the
natural transformation rule):
\begin{equation*}
	\frac{\cont x: \tau_{1} \to \tau_{2}}{\cont \mathbf{most}\, x \coloneqq
		\Pi_{\tau_{2}} \circ x = x \circ \Pi_{\tau_{1}}}
\end{equation*}
What curryfication implies, is that higher-order constructs can be passed down
to the arguments of the functions they are applied to, explaining how we can
reconciliate the following semantic equation even if some of the words are not
marked properly:
\begin{equation*}
	\bf future\left(be\left( I, a\ cat \right)\right)
	= will\ be\left( future\left( I \right), a\ cat \right)
	= be\left( future\left( I \right), a\ cat \right)
\end{equation*}
Indeed the above equation could be simply written by our natural transformation
rule as:
\begin{equation*}
	\bf future\left( be \right)\left( arg_{1}, arg_{2} \right) = future\left( be \right)\left( arg_{2} \right)\left( future\left( arg_{1} \right) \right) = future \left( be \right) \left( future \left( arg_{2} \right) \right) \left( future \left( arg_{1} \right) \right)
\end{equation*}
This is not a definitive rule as we could want to stop at a step in the
derivation, depending on our understanding of the notion of future in
the language.

\paragraph{Monad Transformers}
In \cite{bumfordEffectdrivenInterpretationFunctors2025}, the authors present
constructions which they call monad transformers or \emph{higher-order
	constructors} and which take a monad as input and return a monad as output.
One way to type those easily would be to simply create, for each such
construct, a monad (the result of the application to any other monad) and a
natural transformation which mimics the application and can be seen as the
constructor.

\subsection{Typing Judgements}\label{subsec:judgements}
To complete this section, Figure \ref{tab:judgements} gives a simple list of different typing composition judgements through which we also re-derzive the subtyping judgement to allow for its implementation.
\begin{figure}
	\inputtikz{typing-judgements}
	\caption{Typing and Subtyping Judgements}
	\label{tab:judgements}
\end{figure}
Note that here, the syntax is not taken into account: a function is always written left of its arguments, whether or not they are actually in that order in the sentence.
This issue will be resolved by giving the syntactic tree of the sentence (or infering it at runtime).
We could also add symmetry induced rules for application.

Furthermore, note that the App rule is the \texttt{fmap} rule for the identity functor and that the \texttt{pure/return} and \texttt{>>=} is the \texttt{nat} rule for the monad unit (or applicative unit) and multiplication.

\medskip

Using these typing rules for our combinators, it is important to see that our grammar will still be ambiguous and thus our reduction process will be non-deterministic.
As an example, we provide the typing reductions for the classic example: \textsl{The man sees the girl using a telescope} in Figure \ref{fig:ud}.

\begin{figure}
	\centering
	\inputtikz{parse-tree-ex}
	\caption{Parsing trees for the typing of \textsl{The man sees the girl using a telescope}.}
	\label{fig:ud}
\end{figure}

This non-determinism is a component of our language's grammar and semantics: a same sentence can have multiple interpretation without context.
By the same reasoning we applied on handlers, we choose not to resolve the ambiguity in the parser but we will try in Section \ref{sec:nondet} to reduce it to a minimum.
Even so some of our derivations will include ways to read from a context (for example the \textbf{Jupiter, a planet} construct), we forget about the definition of the context, or its updating after a sentence, in our first proposition.

\subsection{Decidability}
The observant reader might have notice (without too much trouble), that our
typing system is not decidable, because of the \texttt{nat/pure/return} rules
which may allow for infinite derivations.
The good news is, this is only an issue when trying to type things that cannot
be typed.
Indeed, because of the considerations on handling, and the fact that
semantically void units will in the end get deleted without any modifications
on the end meaning, we can just remove the rules that allow for unbounded
derivations, especially the usage of units of applicatives and natural
transformations that allow to switch back and forth between two or more
functors.
This leads to derivations of sentences to be of bounded height, linear in the
length of the sentence.
