\usepackage{bigstrut}
\usepackage{makecell}
\usepackage{emoji}
\usepackage{tikz-dependency}
\usepackage{diag}

\renewcommand{\thefootnote}{\roman{footnote}}

\def\cont{\Gamma\vdash}
\def\poulpe{\qquad}

\setlength\belowcaptionskip{0pt}
\setlength\abovecaptionskip{\baselineskip}

\DeclareMathOperator{\Var}{Var}

\def\ppl{\mathbin{+\mkern-12mu+}}

\makeatletter
\renewenvironment{thebibliography}[1]
{\section{\bibname}
	\@mkboth{\MakeUppercase\bibname}{\MakeUppercase\bibname}%
	\list{\@biblabel{\@arabic\c@enumiv}}%
	{\settowidth\labelwidth{\@biblabel{#1}}%
		\leftmargin\labelwidth
		\advance\leftmargin\labelsep
		\@openbib@code
		\usecounter{enumiv}%
		\let\p@enumiv\@empty
		\renewcommand\theenumiv{\@arabic\c@enumiv}}%
	\sloppy
	\clubpenalty4000
	\@clubpenalty \clubpenalty
	\widowpenalty4000%
	\sfcode`\.\@m}
{\def\@noitemerr
	{\@latex@warning{Empty `thebibliography' environment}}%
	\endlist}

\def\black@or@white#1#2{%
	\@tempdima#2 pt
	\ifdim\@tempdima>0.5 pt
		\definecolor{temp@c}{gray}{0}%
	\else
		\definecolor{temp@c}{gray}{1}%
	\fi}
\def\letterbox#1#{\protect\letterb@x{#1}}
\def\letterb@x#1#2#3{%
	\colorlet{temp@c}[gray]{#2}%
	\extractcolorspec{temp@c}{\color@spec}%
	\expandafter\black@or@white\color@spec
	{\color#1{temp@c}\tallcbox#1{#2}{#3}}}
\def\tallcbox#1#{\protect\color@box{#1}}
\def\color@box#1#2{\color@b@x\relax{\color#1{#2}}}
\long\def\color@b@x#1#2#3%
{\leavevmode
	\setbox\z@\hbox{{\set@color#3}}%
	\ht\z@\ht\strutbox
	\dp\z@\dp\strutbox
	{#1{#2\color@block{\wd\z@}{\ht\z@}{\dp\z@}\box\z@}}}
\makeatother

\contourlength{0.005em}
\def\backbox#1{\contour{black}{#1}}

\def\ty#1{\backbox{\tt\color{yulm!90!black}#1}}
\def\f#1{\backbox{\tt\color{vulm}#1}}
\def\w#1{\mathbf{#1}\,}

\def\e{\ty{e}}
\def\t{\ty{t}}
\def\r{\ty{r}}
\def\ta{\ty{a}}
\def\tb{\ty{b}}

\newcolumntype{C}{>{$}c<{$}}
	\newcolumntype{L}{>{$}l<{$}}
\newcolumntype{R}{>{$}r<{$}}
\def\fmap{\texttt{fmap}}

\newcommand{\suppfrac}[2]{%
	\sbox0{$\genfrac{}{}{0pt}{0}{\ensuremath{#1}}{\ensuremath{#2}}$}%
	\ooalign{%
		\hidewidth
		$\vcenter{\moveright\nulldelimiterspace
				\hbox to\wd0{%
					\xleaders\hbox{\kern.5pt\vrule height 0.4pt width 1.5pt\kern.5pt}\hfill
					\kern-1.5pt
				}%
			}$
		\hidewidth\cr
		$\genfrac{}{}{0pt}{0}{\raisebox{5pt}{\ensuremath{#1}}}{\ensuremath{#2}}$\cr}%
}


\newcommand{\prooffrac}[2]{\genfrac{}{}{}{0}{\ensuremath{#1}}{\ensuremath{#2}}}%
\DeclareMathOperator{\mFunc}{\mathcal{F}}

\def\blank{\phantom{.}}
\def\PT#1#2{\suppfrac{#1}{#2}\fracnotate{$\top$}\phantom{\top\quad}}

\def\Japp#1#2#3#4{%
	\prooffrac{%
		\PT{\blank}{\cont #1: #2 \to #3} \poulpe \PT{\blank}{\cont #4: #2}
	}{%
		\cont #1 #4: #3
	}\fracnotate{App}\phantom{\quad App}
}

\def\Jconj#1#2#3{%
	\prooffrac{%
		\PT{\blank}{\cont #1: #3 \to \t} \poulpe
		\PT{\blank}{\cont #2: #3 \to \t}
	}{%
		\cont #1 \land #2: #3 \to \t
	}\fracnotate{$\land$}\phantom{\quad \land}
}


\def\Jfmap#1#2#3#4#5{%
	\prooffrac{%
		\PT{\blank}{\cont #1: \mC \Rightarrow \mC} \poulpe \PT{\blank}{\cont #2 = #1#2': #1#3} \poulpe \PT{\PT{\blank}{\cont #2':#3} \poulpe \PT{\blank}{\cont #4:#3 \to #5}}{\cont #4#2': #5}
	}{%
		\cont #2#4: #1#5
	}\fracnotate{\texttt{fmap}}\phantom{\quad\tt fmap}
}

\def\Junit#1#2#3#4#5{%
	\prooffrac{%
		\PT{\blank}{\cont #1: \mC \Rightarrow \mC} \poulpe
		\PT{\blank}{\cont #2 = #1#2': #1\left(#3 \to #5\right)} \poulpe
		\PT{%
			\PT{\blank}{\cont #2':#3 \to #5} \poulpe
			\PT{\blank}{\cont #4:#3}
		}{\cont #2'#4: #5}
	}{%
		\cont #2#4: #1#5
	}\fracnotate{\tt pure/return}\phantom{\quad \tt pure/return}
}

\def\sobj{\mathit{Sem}}
\def\combMR{\text{MR}}
\def\combML{\text{ML}}
\def\combUR{\text{UR}}
\def\combUL{\text{UL}}
\def\combC{\text{C}}
\def\combJ{\text{J}}
\def\combA{\text{A}}
\def\combER{\text{ER}}
\def\combEL{\text{EL}}
\def\combDN{\text{DN}}

\makeatletter
\newcommand{\@word}[4][]{%
	#2 & #3 & #4\\
	\ifx&#1&%
	%
	\else
	&\multicolumn{2}{l}{Generalizes to \textbf{#1}}\\%
	\fi%
}
\def\word#1#2#3#4{\@word[#4]{#1}{#2}{#3}}
\makeatother

\usepackage{calc}

\makeatletter
\def\textSq#1{%
	\begingroup% make boxes and lengths local
	\setlength{\fboxsep}{0.4ex}% SET ANY DESIRED PADDING HERE
	\setbox1=\hbox{#1}% save the contents
	\setlength{\@tempdima}{\maxof{\wd1}{\ht1+\dp1}}% size of the box
	\setlength{\@tempdimb}{(\@tempdima-\ht1+\dp1)/2}% vertical raise
	\raise-\@tempdimb\hbox{\fbox{\vbox to \@tempdima{%
				\vfil\hbox to \@tempdima{\hfil\copy1\hfil}\vfil}}}%
	\endgroup%
}
\def\Sq#1{\textSq{\ensuremath{#1}}}%

\def\c@lsep{2.3}
\def\r@wsep{.8}

\tikzset{
	uptree/.style={
			draw=green!80!black,
			thick,
		},
	typenode/.style={
			align=center,
			text width=24mm,
			%font={\large},
		},
	treenode/.style={
			align=center,
			text width=24mm,
		},
	wordnode/.style={
			inner sep=0pt,
			align=center,
			font={\large},
		},
	downtree/.style={
			draw=red!80!black,
			thick,
		},
}

\newcommand{\wnode}[3]{%
	\node (#2) at (#1*\c@lsep, 0) [wordnode] {#2};
	\node[anchor=north] (#2-) at ($(#1*\c@lsep, 0) + (0, -.142)$) [typenode] {\ensuremath{#3}};
}
\newcommand{\utnode}[3]{%
	\path let \p1 = (#2.north), \p2 = (#3.north) in coordinate (Q1) at (\x1, {max(\y1, \y2)});
	\path let \p1 = (#2.north), \p2 = (#3.north) in coordinate (Q2) at (\x2, {max(\y1, \y2)});
	\node (#2#3) at ($($(Q1)!0.5!(Q2)$) + (0, 1)$) [treenode] {\ensuremath{#1}};
	\draw[uptree] ($(#2.north) + (0, .142)$) -- (#2#3.south);
	\draw[uptree] ($(#3.north) + (0, .142)$) -- (#2#3.south);
}
\newcommand{\dtnode}[4][0.5]{%
	\path let \p1 = (#3.south), \p2 = (#4.south) in coordinate (Q1) at (\x1, {min(\y1, \y2)});
	\path let \p1 = (#3.south), \p2 = (#4.south) in coordinate (Q2) at (\x2, {min(\y1, \y2)});
	\node (#3#4) at ($($(Q1)!#1!(Q2)$) + (0, -1)$) [treenode] {\ensuremath{#2}};
	\draw[downtree] ($(#3.south) + (0, -.142)$) -- (#3#4.north);
	\draw[downtree] ($(#4.south) + (0, -.142)$) -- (#3#4.north);
}

\def\inputtikz#1{
	\ifnum\tikzimp@rt=1
		\input{aux/figures/#1}
	\else
		\ensuremath{\text{\Huge\color{vulm}A TikZ PICTURE GOES HERE.}}
	\fi
}
\makeatother

\catstyle{catone}{gray!50}
\catstyle{catmc}{vulm!10!yulm}
\catstyle{catmca}{vulm!20!yulm}
\catstyle{catmcb}{vulm!30!yulm}
\catstyle{catmcc}{vulm!40!yulm}
\catstyle{catmcd}{vulm!50!yulm}
\catstyle{catmce}{vulm!60!yulm}
\catstyle{catmcf}{vulm!70!yulm}
\catstyle{catmcg}{vulm!80!yulm}
\catstyle{catmch}{vulm!90!yulm}

\def\din#1{#1\mathrm{.S}}
\def\dnb#1{#1\mathrm{.N}}
\def\dlb#1#2{#1\mathrm{.L}\left(#2\right)}
\def\dl#1{#1\mathrm{.L}}
\def\dnlg#1{#1\mathrm{.h}}
\def\dnin#1{#1\mathrm{.in}}
\def\dnout#1{#1\mathrm{.out}}

\newcounter{lingexcnt}
\newcounter{tmplingexcnt}
\renewcommand*{\thelingexcnt}{(\arabic{lingexcnt})}
\newenvironment{sentence}[1][]{
	\begin{list}{\thelingexcnt}{\refstepcounter{lingexcnt}}\item
		      \ifnum\pdfstrcmp{#1}{}=0\else\label{#1}\fi
		      }{\end{list}}

\newenvironment{nsentence}{%
	\setcounter{tmplingexcnt}{\value{lingexcnt}}
	\addtocounter{tmplingexcnt}{-1}
	\begin{list}{\thelingexcnt}{
			\usecounter{lingexcnt}
			\setcounter{lingexcnt}{\value{tmplingexcnt}}
			\refstepcounter{lingexcnt}
		}
		}{\end{list}}

\newcommand*{\oneSentence}[2][]{\begin{sentence}[#1]#2\end{sentence}}
