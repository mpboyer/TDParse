\section{Semantic Parsing}
\label{sec:parsing}
In this section we explain our algorithms and heuristics for efficient parsing with as little non-determinism as possible, and reducing time complexity of our parsing strategy.

\subsection{Naïve Semantic Parsing on Syntactic Parsing}
In this section we suppose that we have a set of syntax trees (or parsing trees) corresponding to a certain
sentence.
We will now focus on how to derive proofs of typing from a syntax tree.
First, note that \cite{bumfordEffectdrivenInterpretationFunctors2025} provides a way to do so by constructing
semantic trees labelled by sequence of \emph{combinators}.
In our formalism, this amounts to constructing proof trees by mapping combination modes to their equivalent proof
trees, inverting if needed the order of the presuppositions to account for the order of the words.
This is easily done in linear time in the number of nodes in the parsing tree (which is linear in the input size,
more on that in the next section), multiplied by a constant whose size depends on the size of the inference
structure.
The main idea is that to each node of the tree, both nodes have a type and there is only a finite set of rules
that can be applied, provided by the following rules, which are a rewriting of Table \ref{tab:judgements}.
In Table \ref{tab:proof-trees} we provide \emph{matching-like} rules for different possibilities on the types to
combine and the associated possible proof tree(s).
Note that there is no condition on what the types \emph{look like}, they can be effectful.
If multiple cases are possible, all different possible proof trees should be added to the set of derivations: the
set of proof trees for the union of cases is the union of set of proof trees for each case, naturally:
\begin{equation*}
	PT\left( \cup C \right) = \cup PT(C)
\end{equation*}
Remember again that the order of presuppositions for an inference structure does not matter, so always we have the
other order of arguments available.

\begin{table}
	\centering
	\resizebox{\linewidth}{!}{%
  \def\arraystretch{2}
	\begin{NiceTabular}{CCC>{$}c<{$}}[hvlines]
		                             & S_{1}                                  & S_{2}                & PT(S_{1}, S_{2})              \\
		\Block{2-1}{\rm Base Types}  & l: \ta \to \tb                         & r: \ta               & \Japp{l}{\ta}{\tb}{r}         \\
		                             & l: \ta \to \t                          & r: \ta \to \t        & \Jconj{l}{r}{\ta}             \\
		\Block{1-1}{\rm Functors}    & \cont l: \f{F}\ta                      & \cont r: \ta \to \tb & \Jfmap{\f{F}}{l}{\ta}{r}{\tb} \\
		\Block{1-1}{\rm Applicative} & \cont l: \f{F}\left(\ta \to \tb\right) & \cont r: \ta         & \Junit{\f{F}}{l}{\ta}{r}{\tb}
	\end{NiceTabular}
}

	\caption{List of possible combinations for different presuppositions for inputs, as a definition of a function
		$PT$ from proof trees to proof trees.}
	\label{tab:proof-trees}
\end{table}

It is important to note that here the proof trees are done ``in the wrong direction'', as they are written
top-down instead of bottom-up in the sense that we start by the axioms which are the typing judgements of
constants in the lexicon (this could actually be said to be a part of the sentence, but non-determinism in the
meanings is fine as long as the syntactic category is provided in the parsing tree).
This leads to proof trees being a bit weird to the type theorist and a bit weird to the linguist as they are
not written as usual trees.
Moreover, while a proof tree only provides a type, the tree also provides the sequence of function applications
that is needed to compute the actual denotation.

This leads the induced algorithm (for computing the set of denotations) to be exponential in the input in the
worst case, when using the recursive scheme that naturally arises from the definition of $PT$.
Indeed, and while this may seem weird, in the case of a function $\f{F}\ta \to \tb$ where $\f{F}$ is applicative
and the other combinator is of type $\f{F}\ta$, one might apply the function directly or apply it under the $\f{F}$
of the argument, leading to two different proof trees:
\begin{align*}
	\prooffrac{\cont \f{F}: \mC \Rightarrow \mC \poulpe \Junit{\f{F}}{l}{\ta}{r}{\tb}}{\cont \f{F}lr: \f{F}\tb} \\
	\Japp{l}{\f{F}\ta}{\tb}{r}
\end{align*}
Those two different proof trees have different semantic interpretations that may be useful, as discussed by
\newcite{bumfordEffectdrivenInterpretationFunctors2025}, especially in their analysis of closure properties, which
is modified in the following section.

\subsubsection{Islands}
\newcite{bumfordEffectdrivenInterpretationFunctors2025} provide a formal analysis of islands\footnote{Syntactic
	structures which prevent some notion of moving outside of the island.} based on a islands being a different type
of branching nodes in the syntactic tree of a sentence, which asks to resolve all $\f{C}$ effects\footnote{Those
	represent continuations.} before that node or being resolved at that node.
The main example they propose is the one of existential closure inside conditioning.
To reconcile this inside our type system, we propose the following change to their formalism: once the syntactic
information of an island existing is added to the tree (or at semantic parsing time, this does not change the
time complexity), we \emph{mark} each node inside the island by adding an ``void effect'' to it, in the same
way as we did for our model of plural.
This translates into a functor which just maintains the \emph{island marker} on \texttt{fmap} and add to the
words which create an island node\footnote{Or \emph{post-compose} the proof trees with a rule on handling the
	$\f{C}$ effect.} a function which handles the $\f{C}$ effects, which could be seen as adding a node in the
semantics parse tree from the syntactic parse tree.
A way to do this would be the following: we pass all functors until finding a $\f{C}$, handle it inside the other
functors and keep going, on both sides, where $\mathrm{PT}$ is defined in Table \ref{tab:proof-trees}:
\begin{equation*}
	\mathrm{PT}\left(
	\suppfrac{\cont x_{1}: \f{F}\f{C}\f{F'}\tau}{\cont x_{1}: \f{F}\f{F'}\tau}\suchthat
	\suppfrac{\cont x_{2}: \f{F}\f{C}\f{F'}\tau}{\cont x_{2}: \f{F}\f{F'}\tau}
	\right)
\end{equation*}
Note that this preserves the linear size of the parse tree in the number of input words, as we at most double the
number of nodes, and note that this would be preserved with additional similar constructs.

\medskip

This idea amounts to seeing islands, whatever their type, as a form of grammatical/syntactic effect, which is
still part of the language but not of the lexicon, a bit like future, modality or plurals, without the semantic
component.
The idea of seeing everything as effects, while semantically void, allows us to translate into our theory of
type-driven semantics outer information from syntax, morphology or phonology that influences the semantics of the
sentence.
Other examples of this can be found in the modelisation of plural (for morphology, see Sections \ref{par:higherorder}
and \ref{subsec:modality}) and the emphasis of the words by the $\f{F}$ effect (for phonology), and show the
empirical well-foundedness of this point of view.
While we do not aim to provide a theory of everything linguistics through categories\footnote{``\emph{A theory is as
		large as its most well understood examples}''.}, the idea of expressing everything in our effect-driven type-driven
theory of semantics allows us to prepare for any theoretical or empirical observation that has an impact on the
semantics of a word/sentence, the allowed combinators and even the addition of rules.

\subsection{Rewriting Rules}
\label{subsec:rewrite}
Here we provide a rewriting system on derivation trees that allows us to improve our time complexity.
In the worst case, in big o notation there is no improvement, but there is no loss.
The goal is to reduce branching in the definition of $PT$.

\mesdkip

First, let's consider the reductions proposed in Section \ref{sec:nondet}.
Those define normal forms under Theorem \ref{thm:confluence} and thus we can use those to reduce the number
of trees.
Indeed, we usually want to reduce the number of effects by first using the rules \emph{inside} the language,
understand, the adjunction co-unit and monadic join.
Thus, when we have a presupposition of the form:
\begin{equation*}
	\suppfrac{\blank}{\cont x: \f{M}\f{M}\tau} \poulpe \text{or} \poulpe \suppfrac{\blank}{\cont x: RL\tau}
\end{equation*}
we always simplify it directly, as not always doing it would propagate multiple trees.
This in turn means we always verify the derivational rules we set up in the previous section for the join
equations of the monad.

\medskip

Secondly, while there is no way to reduce the branching proposed in the previous section since they end in
completely different reductions, there is another case in which two different reductions arise:
\begin{equation*}
	\PT{\blank}{\cont x: \f{F}\tau_{1}} \poulpe \text{and} \poulpe \PT{\blank}{\cont y: \f{G}\tau_{2}}
\end{equation*}
Indeed, in that case we could either get a result with effects $\f{F}\f{G}$ or with effects $\f{G}\f{F}$.
In general those effects are not the same, but if they happen to be, which trivially will be the case when one of
the effects is external, the plural or islands functors for example.
When the two functors commute, the order of application does not matter and thus we choose to get the outer
effect the one of the left side of the combinator.
Note that this is also true when we have an equation that

\medskip

The observant reader might have noticed that in the definition of $PT$, we actually do not cover all the possible
cases, as in particular, we have not included the option of using the unit of an applicative/monad alone, as it
actually would reduce to equivalent cases most of times.
Note moreover that this is simply of scheme to apply typing rules to a syntactic derivation, but that this will not
be enough to actually gain all reductions possible.
This will actually happen once a confluent reduction scheme is provided for the denotations.
When this is done, we can combine the reduction schemes for effects along with the one for denotation and the one
for combinations in one large reduction scheme.
