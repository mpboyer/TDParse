\section*{Introduction}
What is \emph{a chair}? How do I know that \emph{Jupiter, a planet}, is
\emph{a planet}?
To answer those questions, \cite{bumfordEffectdrivenInterpretationFunctors2025}
provides a \textsc{Haskell}-based view on the notion of typing in natural
language semantics.
Their main idea is to include a layer of effects which allows for improvements
in the expressiveness of the denotations used.
This allows us to model complex concepts such as anaphoras, or non-determinism
in an easy way, independent of the actual way the words are represented.
Indeed, when considering the usual denotations of words as typed lambda-terms,
this allows us to solve the issue of meaning getting lost through impossible
typing, while still being able to compose meanings properly.
When two expressions have the same syntactic distribution, they must also have
the same type, which forces quantificational noun phrases to have the same type
as proper nouns: the entity type $\e$.
However, there is no singular entity that is the referent of \emph{every
	planet}, and so, the type system gets in the way of meaning, instead of being
a tool at its service.

\smallskip

Our formalism is inscribed in the contemporary natural language semantic
theories based on three main elements: a \emph{lexicon}, a
\emph{syntactic description} of the language, and a theory of
\emph{composition}.
More specifically, we explain how to extend the domain of the lexicon and the
theory of composition to account for the phenomena described above.
We will not be discussing most of the linguistic foundations for the usage of
the formalism, nor its usefulness.
We refer to \cite{bumfordEffectdrivenInterpretationFunctors2025} to get an
overview of the linguistic considerations at the base of the theory.

\smallskip

In this report, we will provide a formal definition of an enhanced type and
effect system for natural language semantics, based on categorical tools.
This will increase the complexity (both in terms of algorithmic operations and
in comprehension of the model) of the parsing algorithms, but through the use
of string diagrams to model the effect of composition on potential effectful
denotations (or more generally computations), we will provide efficient
algorithms for computing the set of meanings of a sentence from the meaning of
its components.

\section{Related Work}
This is not the first time a categorical representation of compositional
semantics of natural language is proposed,
\cite{coeckeMathematicalFoundationsCompositional2010} already suggested an
approach based on monoidal categories using an external model of meaning.
Our approach gives additional latitude for the definition of denotations in the
lexicon, and a visual explanation of the difference between multiple possible
parsing trees.
We will go back later on the differences between their Lambek inspired grammars
and our more abstract way of looking at the semantic parsing of a sentence.

\smallskip

On a completely different approach,
\cite{marcollimatildeetchomskynoametberwickrobertc.MathematicalStructureSyntactic}
provide a categorical structure based on Hopf algebra and coloured operads
to explain their model of syntax.
Their approach allowed to reconcile the works on syntactic and
morphosyntactic trees in an upcoming paper by Isabella Senturia and Matilde
Marcolli.
Similarly, \cite{melliesCategoricalContoursChomskySchutzenberger2025} provides
a modeling of CFGs using coloured operads.
Our approach is based on the suggestion that merge in syntax can be done using
labels, independent on how it is mathematically modelled.

\smallskip

Of course, the works most related to this report are the ones on the notion of
monadic types, see for example \cite{asudehEnrichedMeaningsNatural2020} or
previous works by Simon Charlow.
