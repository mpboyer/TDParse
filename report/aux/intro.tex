\section*{Introduction}
\newcite{moggiComputationalLambdacalculusMonads1989} provided a way to view monads as effects
in functional programming. This allows for new modes of combination in a compositional
semantic formalism, and provides a way to model words which usually raise problems with the
traditional lambda-calculus representation of the words. In particular we consider words such
as \textsl{the} or \textsl{a} whose application to common nouns results in types that should
be used in similar situations but with largely different semantics, and model those as
functions whose application yields an effect. This allows us to develop typing judgements and
an extended typing system for compositional semantics of natural languages.
Type-driven compositional semantics acts under the premise that given a set of words and their
denotations, a set of grammar rules for composition and their associated typing judgements,
we are able to form an enhanced parser for natural language which provides a mathematical
representation of the meaning of sentences, as proposed by \newcite{heimSemanticsGenerativeGrammar1998}.

\medskip

This is not the first time a categorical representation of a compositional semantics of
natural language is proposed, \newcite{coeckeMathematicalFoundationsCompositional2010}
already suggested an approach based on monoidal categories using an outside model of meaning.
However, what we propose here is a representation of the different capabilities of words
as categorical constructs: we allow for a wider set of representations inside our
model of meaning, trading non-determinism for additional structures.
In this regard, we basically combine the grammatical type and the meaning of a word
by having our denotations be associated with a type: there is no need from an additional
category outside of our typing category and we limit ourselves to reducing non-determinism
by limiting our possibilities for combinations with a provided CFG\footnote{Or another model
	that could generate our language.} of the language.

The focus of our system is to allow more flexibility in denotations, leading to more
possibilities of combinations.
