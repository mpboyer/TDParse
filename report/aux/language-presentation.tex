\section{Presenting a Language}
\label{sec:language}
In this section we will give a list of words along with a way to express them
as either arrows or endo-functors of our typing category.
This will also give a set of functors and constructs in our language.
Apart from Section \ref{subsec:modality}, the contents from this section are a
rewriting from \cite{bumfordEffectdrivenInterpretationFunctors2025}.

\subsection{Syntax}
\label{subsec:syntax}
In Figure \ref{fig:english-cfg} we provide a toy Context-Free Grammar to
support the claims made in Section \ref{subsec:ssparsing}.
The discussion of whether this system accurately models the syntax of the
english language is far beyond the scope of this paper, but could replace
fully this section.
Our syntactic categories will be written in a more linguistically
appreciated style than the one proposed in Figure \ref{fig:sctypes}.
Our set of categories is the following:
\begin{center}
	\tt CP, Cmp, CBar, DBar, Cor, DP, Det, Gen, GenD, Dmp, NP, TN, VP, TV, DV, AV, AdjP, TAdj, Deg, AdvP, TAdv
\end{center}

\begin{figure}
	\centering
	\inputtikz{english-cfg}
	\caption{Partial CFG for the English Language}\label{fig:english-cfg}
\end{figure}

\subsection{Lambda-Calculus}\label{subsec:lambdacalc}
In this section we use the traditional lambda-calculus denotations, with
two types for truth values and entities freely generating our typing category.

\subsubsection{Types of Syntax}\label{subsec:typingsyntax}
We first need to setup some guidelines for our denotations, by asking ourselves
how the different components of sentences interact with each other.
For syntactic categories\footnote{Those are a subset of the ones proposed
	in Section \ref{subsec:syntax}.}, the types that are generally used are the
following, and we will see it matches with our lexicon, and simplifies our
functorial definitions\footnote{We don't consider effects in the given typings.}.
Those are based on \cite{parteeLecture2Lambda}. Here $\Upsilon$ is the
operator which retrieves the type from a certain syntactic category.
\begin{figure}
	\centering
	\inputtikz{syntactic-categories}
	\caption{Usual Typings for some Syntactic Categories}
	\label{fig:sctypes}
\end{figure}

\subsubsection{Lexicon: Semantic Denotations for Words}\label{subsec:lexicon}
Many words will have basically the same ``high-level'' denotation.
For example, the denotation for most common nouns will be of the form:
$\cont \lambda x. \w{planet} x: \e \to \t$.
In Figure \ref{fig:lexicon} we give a lexicon for a subset of the english
language, without the syntactic categories associated with each word, since
some are not actually words but modifiers on words\footnote{Emphasis,
	for example.}, and since most are obvious but only contain one meaning of a
word.
We describe the constructor for the functors used by our denotations in the
table, but all functors will be reminded and properly defined in Figure
\ref{fig:functors} along with their respective \fmap.
For denotations that might not be clear at first glance, $\cdot_{\tt F}$ is the
phonologic accentuation or emphasis of a word and ${\cdot, \bf a \cdot}$ is the
apposition of a \textbf{NP} to a \textbf{DP} or proper noun.
\begin{figure*}
	\centering
	\inputtikz{lexicon-table}
	\caption{$\lambda$-calculus representation of the english language $\mL$}
	\label{fig:lexicon}
\end{figure*}
Note that for terms with two lambdas, we also say that inverting the lambdas also provides a valid denotation.
This is important for our formalism as we want to be able to add effects in the order that we want.

\subsubsection{Effects of the Language}\label{subsec:effects}
For the applicatives/monads in Figure \ref{fig:functors} we do not specify the unit and multiplication functions, as they are quite usual examples.
We still provide the \fmap{} for good measure.

\begin{figure*}
	\centering
	\inputtikz{functors-table}
	\caption{Denotations for the functors used}
	\label{fig:functors}
\end{figure*}

Let us explain a few of those functors: $\f G$ designates reading from a certain environment of type $\r$ while $\f W$ encodes the possibility of logging a message of type $\t$ along with the expression.
The functor $\f M$ describes the possibility for a computation to fail, for example when retrieving a value that does not exist (see $\mathbf{the}$).
The $\f S$ functor represents the space of possibilities that arises from a non-deterministic computation (see $\mathbf{which}$).

We can then define an adjunction between $\f G$ and $\f W$ using our definitions:
\begin{equation*}
	\phi: \begin{array}{l}
		\left( \alpha \to \f{G}\beta \right)\to \f{W}\alpha \to \beta \\
		\lambda k. \lambda \left( a, g \right) . k a g
	\end{array}
	\text{ and }
	\psi: \begin{array}{l}
		\left( \f{W}\alpha \to \beta \right) \to \alpha \to \f{G} \beta \\
		\lambda c \lambda a \lambda g. c \left( a, g \right)
	\end{array}
\end{equation*}
where $\phi \circ \psi = \id$ and $\psi \circ \phi = \id$ on the properly defined sets.
Now it is easy to see that $\phi$ and $\psi$ do define a natural transformation and thus our adjunction $\f{W} \dashv \f{G}$ is well-defined, and that its unit $\eta$ is $\psi \circ \id$ and its co-unit $\epsilon$ is $\phi \circ \id$.

As a reminder, this means our language canonically defines a monad $\f{G}\circ \f{W}$.
Moreover, the co-unit of the adjunction provides a canonical way for us to deconstruct entirely the comonad $\f{W}\circ \f{G} $, that is we have a natural transformation from $\f{W} \circ \f{G} \Rightarrow \Id$.

\subsubsection{Modality, Plurality, Aspect, Tense}\label{subsec:modality}
We will now explain ways to formalise higher-order concepts as described in Section \ref{par:higherorder}.

First, let us consider in this example the plural concept.
Here we do not focus on whether our denotation of the plural is semantically correct, but simply on whether our modelisation makes sense.
As such, we give ourselves a way to measure if a certain $x$ is plural our not, which we will denote by $\abs{x} \geq 2$\footnote{We chose this representation as an example, it is not important for our formalism that this denotation is actually a good choice for the plural.}.
It is important to note that if $\tau$ is a type, and $\cont x: \tau$ then we should have $\abs{\Pi x} \geq 1 = \top\footnote{This might be seen as a typing judgement~!}$.
We then need to define the functor $\Pi$ which models the plural. We do not need to define it on types in any way other than $\Pi\left( \tau \right) = \Pi\tau$.
We only need to look at the transformation on arrows\footnote{\texttt{fmap}, basically.}.
This one depends on the \emph{syntactic}\footnote{Actually it depends on the word considered, but since the denotations (when considered effect-less) provided in Figure \ref{fig:lexicon} are more or less related to the syntactic category of the word, our approximation suffices.} type of the arrow, as seen in Figure \ref{fig:lexicon}.
\begin{figure*}
	\centering
	\inputtikz{plural-table}
	\caption{(Partial) Definition for the $\Pi$ Plural Functor}
	\label{fig:pluralfunctor}
\end{figure*}
There is actually a presupposition in our definition. Whenever we apply $\Pi$ to an arrow representing a predicate $p: \e \to \t$, we still apply $p$ to its argument $x$ even though we say that $\Pi p$ is the one that applies to a plural entity $x$.
This slight notation abuse results from our point of view on plural/its predicate representation: we assume the predicate $p$ (the common noun, the adjectve, the VP\ldots) applies to an object without regarding its cardinality.
The two point of views on the singular/plural distinction could be adapted to our formalism: whether we believe that singular is the \emph{natural} state of the objects or that it is to be always specified in the same way as plural does not change anything:
In the former, we do not change anything from the proposed functors.
In the latter we simply need to create another functor $\Sigma$ with basically the same rules that represents the singular and ask of our predicates $p$ to be defined on $\left(\Pi\star\right) \sqcup \left(\Sigma \star\right)$.

\medskip

See that our functor only acts on the words which return a boolean by adding a plurality condition, and is simply passed down on the other words.
Note however there is an issue with the \textbf{NP} category: in the case where a \textbf{NP} is constructed from a determiner and a common noun (phrase) the plural is not passed down.
This comes from the fact that in this case, either the determiner or the common noun (phrase) will be marked, and by functoriality the plural will go up the tree.

Now clearly our functor verifies the identity and composition laws and works as theorised in Section \ref{par:higherorder}.
To understand a bit more why this works as described in our natural transformation rules, consider $\Pi\left( \mathbf{which} \right)$.
The result will be of type $\Pi\left( \f{S}\left( e \right) \right) = \left(\Pi \circ \f{S}\right) \left( e \right)$.
However, when looking at our transformation rule (and our functorial rule) we see that the result will actually be of type $\f{S}\left( \Pi e \right)$ which is exactly the expected type when considering the phrase $\w{which} p$.
The natural transformation $\theta$ we set is easily inferable:
\begin{equation*}
	\theta_{A}:
	\begin{tikzcd}
		\left( \Pi \circ \f{S} \right) A \ar[r, "\theta_{A}"] & \left( \f{S} \circ \Pi \right) A\\[-.7cm]
		\Pi\left( \left\{ x \right\} \right) \ar[r, mapsto] &  \left\{ \Pi(x) \right\}
	\end{tikzcd}
\end{equation*}
The naturality follows from the definition of $\fmap$ for the $\f{S}$ functor.
We can easily define natural transformations for the other functors in a similar way: $\Pi \circ \f{F} \overset{\theta}{\Rightarrow} \f{F} \circ \Pi$ defined by $\theta_{\tau} = \fmap_{\f{F}}\left( \Pi \right)$.

\medskip

Now, in quite a similar way, we can create functors from any sort of judgement that can be seen as a function $\e \to \t$ in our language\footnote{An easy way to define those would be, similarly to the plural, to define a series of judgement from a property that could be infered.}.
Indeed, we simply need to replace the $\abs{p} \geq 1$ in most of the definitions by our function and the rest would stay the same.
Remember that our use of natural transformations is only there to allow for possible under-markings of the considered effects and propagating the value resulting from the computation of the high-order effect.

