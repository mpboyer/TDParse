\documentclass[math, english, info]{cours}

\makeatletter
\def\tikzimp@rt{0}
\makeatother

\usepackage{bigstrut}
\usepackage{makecell}
\usepackage{emoji}
\usepackage{tikz-dependency}
\usepackage{diag}

\def\cont{\Gamma\vdash}
\def\poulpe{\qquad}

\setlength\belowcaptionskip{0pt}
\setlength\abovecaptionskip{\baselineskip}

\DeclareMathOperator{\Var}{Var}

\def\ppl{\mathbin{+\mkern-12mu+}}

\makeatletter
\renewenvironment{thebibliography}[1]
     {\section{\bibname}
      \@mkboth{\MakeUppercase\bibname}{\MakeUppercase\bibname}%
      \list{\@biblabel{\@arabic\c@enumiv}}%
           {\settowidth\labelwidth{\@biblabel{#1}}%
            \leftmargin\labelwidth
            \advance\leftmargin\labelsep
            \@openbib@code
            \usecounter{enumiv}%
            \let\p@enumiv\@empty
            \renewcommand\theenumiv{\@arabic\c@enumiv}}%
      \sloppy
      \clubpenalty4000
      \@clubpenalty \clubpenalty
      \widowpenalty4000%
      \sfcode`\.\@m}
     {\def\@noitemerr
       {\@latex@warning{Empty `thebibliography' environment}}%
      \endlist}

\def\black@or@white#1#2{%
  \@tempdima#2 pt
  \ifdim\@tempdima>0.5 pt
    \definecolor{temp@c}{gray}{0}%
  \else
    \definecolor{temp@c}{gray}{1}%
  \fi}
\def\letterbox#1#{\protect\letterb@x{#1}}
\def\letterb@x#1#2#3{%
  \colorlet{temp@c}[gray]{#2}%
  \extractcolorspec{temp@c}{\color@spec}%
  \expandafter\black@or@white\color@spec
  {\color#1{temp@c}\tallcbox#1{#2}{#3}}}
\def\tallcbox#1#{\protect\color@box{#1}}
\def\color@box#1#2{\color@b@x\relax{\color#1{#2}}}
\long\def\color@b@x#1#2#3%
 {\leavevmode
  \setbox\z@\hbox{{\set@color#3}}%
  \ht\z@\ht\strutbox
  \dp\z@\dp\strutbox
  {#1{#2\color@block{\wd\z@}{\ht\z@}{\dp\z@}\box\z@}}}
\makeatother

\contourlength{0.005em}
\def\backbox#1{\letterbox{Lavender!40}{\contour{black}{#1}}}

\def\ty#1{\backbox{\tt\color{yulm!90!black}#1}}
\def\f#1{\backbox{\tt\color{vulm}#1}}
\def\w#1{\mathbf{#1}\,}

\def\e{\ty{e}}
\def\t{\ty{t}}
\def\r{\ty{r}}

\newcolumntype{C}{>{$}c<{$}}
\newcolumntype{L}{>{$}l<{$}}
\newcolumntype{R}{>{$}r<{$}}
\def\fmap{\texttt{fmap}}


\makeatletter
\newcommand{\@word}[4][]{%
	#2 & #3 & #4\\
\ifx&#1&%
	%
\else
	&\multicolumn{2}{l}{Generalizes to \textbf{#1}}\\%
\fi%
}
\def\word#1#2#3#4{\@word[#4]{#1}{#2}{#3}}
\makeatother

\usepackage{calc}

\makeatletter
\def\textSq#1{%
\begingroup% make boxes and lengths local
\setlength{\fboxsep}{0.4ex}% SET ANY DESIRED PADDING HERE
\setbox1=\hbox{#1}% save the contents
\setlength{\@tempdima}{\maxof{\wd1}{\ht1+\dp1}}% size of the box
\setlength{\@tempdimb}{(\@tempdima-\ht1+\dp1)/2}% vertical raise
\raise-\@tempdimb\hbox{\fbox{\vbox to \@tempdima{%
  \vfil\hbox to \@tempdima{\hfil\copy1\hfil}\vfil}}}%
\endgroup%
}
\def\Sq#1{\textSq{\ensuremath{#1}}}%

\def\c@lsep{2.3}
\def\r@wsep{.8}

\tikzset{
	uptree/.style={
			draw=green!80!black,
			thick,
		},
	typenode/.style={
			align=center,
			text width=24mm,
			%font={\large},
		},
	treenode/.style={
			align=center,
			text width=24mm,
		},
	wordnode/.style={
			inner sep=0pt,
			align=center,
			font={\large},
		},
	downtree/.style={
			draw=red!80!black,
			thick,
		},
}

\newcommand{\wnode}[3]{%
	\node (#2) at (#1*\c@lsep, 0) [wordnode] {#2};
	\node[anchor=north] (#2-) at ($(#1*\c@lsep, 0) + (0, -.142)$) [typenode] {\ensuremath{#3}};
}
\newcommand{\utnode}[3]{%
	\path let \p1 = (#2.north), \p2 = (#3.north) in coordinate (Q1) at (\x1, {max(\y1, \y2)});
	\path let \p1 = (#2.north), \p2 = (#3.north) in coordinate (Q2) at (\x2, {max(\y1, \y2)});
	\node (#2#3) at ($($(Q1)!0.5!(Q2)$) + (0, 1)$) [treenode] {\ensuremath{#1}};
	\draw[uptree] ($(#2.north) + (0, .142)$) -- (#2#3.south);
	\draw[uptree] ($(#3.north) + (0, .142)$) -- (#2#3.south);
}
\newcommand{\dtnode}[4][0.5]{%
	\path let \p1 = (#3.south), \p2 = (#4.south) in coordinate (Q1) at (\x1, {min(\y1, \y2)});
	\path let \p1 = (#3.south), \p2 = (#4.south) in coordinate (Q2) at (\x2, {min(\y1, \y2)});
	\node (#3#4) at ($($(Q1)!#1!(Q2)$) + (0, -1)$) [treenode] {\ensuremath{#2}};
	\draw[downtree] ($(#3.south) + (0, -.142)$) -- (#3#4.north);
	\draw[downtree] ($(#4.south) + (0, -.142)$) -- (#3#4.north);
}

\def\inputtikz#1{
	\ifnum\tikzimp@rt=1
		\input{figures/#1}
	\else
		\ensuremath{\text{\Huge\color{vulm}A TikZ PICTURE GOES HERE.}}
	\fi
}
\makeatother

\catstyle{catone}{gray!50}
\catstyle{catmc}{vulm!10!yulm}
\catstyle{catmca}{vulm!20!yulm}
\catstyle{catmcb}{vulm!30!yulm}
\catstyle{catmcc}{vulm!40!yulm}
\catstyle{catmcd}{vulm!50!yulm}
\catstyle{catmce}{vulm!60!yulm}
\catstyle{catmcf}{vulm!70!yulm}
\catstyle{catmcg}{vulm!80!yulm}
\catstyle{catmch}{vulm!90!yulm}

\def\din#1{#1\mathrm{.S}}
\def\dnb#1{#1\mathrm{.N}}
\def\dlb#1#2{#1\mathrm{.L}\left(#2\right)}
\def\dl#1{#1\mathrm{.L}}
\def\dnlg#1{#1\mathrm{.h}}
\def\dnin#1{#1\mathrm{.in}}
\def\dnout#1{#1\mathrm{.out}}

\newcounter{lingexcnt}
\newcounter{tmplingexcnt}
\renewcommand*{\thelingexcnt}{(\arabic{lingexcnt})}
\newenvironment{sentence}[1][]{
     \begin{list}{\thelingexcnt}{\refstepcounter{lingexcnt}}\item
     \ifnum\pdfstrcmp{#1}{}=0\else\label{#1}\fi
}{\end{list}}

\newenvironment{nsentence}{%
     \setcounter{tmplingexcnt}{\value{lingexcnt}}
     \addtocounter{tmplingexcnt}{-1}
     \begin{list}{\thelingexcnt}{
         \usecounter{lingexcnt}
         \setcounter{lingexcnt}{\value{tmplingexcnt}}
         \refstepcounter{lingexcnt}
     }
}{\end{list}}

\newcommand*{\oneSentence}[2][]{\begin{sentence}[#1]#2\end{sentence}}


\title{On a Categorical Type and Effect Inference Structure for Semantic Denotation Combinations in Natural Languages:\\ A Purely Functional Analysis of English}
\author{Matthieu Boyer}

\begin{document}
\bettertitle

\begin{abstract}
	The main idea is that you can consider some words to act as functors in a typing category, for example determiners:
	they don't change the way a word acts in a sentence,
	but more the setting in which the word works by adding an effect to the work.
	The linguistics is based mostly on work by \newcite{bumfordEffectdrivenInterpretationFunctors2025}.
\end{abstract}

\section*{Introduction}
\newcite{moggiComputationalLambdacalculusMonads1989} provided a way to view monads as effects
in functional programming. This allows for new modes of combination in a compositional
semantic formalism, and provides a way to model words which usually raise problems with the
traditional lambda-calculus representation of the words. In particular we consider words such
as \textsl{the} or \textsl{a} whose application to common nouns results in types that should
be used in similar situations but with largely different semantics, and model those as
functions whose application yields an effect. This allows us to develop typing judgements and
an extended typing system for compositional semantics of natural languages.
Type-driven compositional semantics acts under the premise that given a set of words and their
denotations, a set of grammar rules for composition and their associated typing judgements,
we are able to form an enhanced parser for natural language which provides a mathematical
representation of the meaning of sentences, as proposed by \newcite{heimSemanticsGenerativeGrammar1998}.

\medskip

This is not the first time a categorical representation of a compositional semantics of
natural language is proposed, \newcite{coeckeMathematicalFoundationsCompositional2010}
already suggested an approach based on monoidal categories using an outside model of meaning.
However, what we propose here is a representation of the different capabilities of words
as categorical constructs: we allow for a wider set of representations inside our
model of meaning, trading non-determinism for additional structures.
In this regard, we basically combine the grammatical type and the meaning of a word
by having our denotations be associated with a type: there is no need from an additional
category outside of our typing category and we limit ourselves to reducing non-determinism
by limiting our possibilities for combinations with a provided CFG\footnote{Or another model
	that could generate our language.} of the language.

The focus of our system is to allow more flexibility in denotations, leading to more
possibilities of combinations.

\section{Categorical Semantics of Effects: A Typing System}
\label{sec:typingsystem}
In this section, we will designate by $\mL$ our language, as a set of words (with their semantics) and syntactic rules to combine them semantically.
We denote by $\O\left( \mL \right)$ the set of words in the language whose semantic representation is a low-order function and $\mF\left( \mL \right)$ the set of words whose semantic representation is a functor or high-order function.
Our goals here are to describe more formally, using a categorical vocabulary, the environment in which the typing system for our language will exist, and how we connect words and other linguistic objects to the categorical formulation.

\subsection{Typing Category}\label{subsec:typingcategory}
\subsubsection{Types}\label{subsubsec:types}
Let $\mC$ be a closed cartesian category. This represents our main typing system, consisting of words $\O(\mL)$ that can be expressed without effects.
Remember that $\mC$ contains a terminal object $\bot$ representing the empty type or the lack thereof.
We can consider $\bar{\mC}$ the category closure of $\mF\left( \mL \right)\left(\O\left( \mL \right)\right)$, that is consisting of all the different type constructors (ergo, functors) that could be formed in the language.
What this means is that we consider for our category objects any object that can be attained in a finite number from a finite number of functorial applications from an object of $\mC$.
In that sense, $\bar{\mC}$ is still a closed cartesian category (since all our functors induce isomorphisms on their image)\footnote{Our definition does not yield a closed cartesian category as there are no exponential objects between results of two different functors, but this is not a real issue as we can just say we add those.}.
$\bar{\mC}$ will be our typing category (in a way).

We consider for our types the quotient set $\star = \mathrm{Obj}\left( \bar{\mC} \right)/\mF\left( \mL \right)$.
Since $\mF\left( \mL \right)$ does not induce an equivalence relation on $\Obj\left( \bar{\mC} \right)$ but a preorder, we consider the chains obtained by the closure of the relation $x\succeq y \Leftrightarrow \exists F, y = F(x)$ (which is seen as a subtyping relation as proposed in \newcite{melliesFunctorsAreType2015}).
We also define $\star_{0}$ to be the set obtained when considering types which have not yet been \emph{affected}, that is $\Obj(\mC)$.
In contexts of polymorphism, we identify $\star_{0}$ to the adequate subset of $\star$.
In this paradigm, constant objects (or results of fully done computations) are functions with type $\bot \to \tau$ which we will denote directly by $\tau \in \star_{0}$.

What this construct actually means, in categorical terms, is that a type is an object of $\bar{\mC}$ (as intended) but we add a subtyping relationship based on the procedure used to construct $\bar{\mC}$.
Note that we can translate that subtyping relationship on functions as $F\left( A \xrightarrow{\phi} B \right)$ has types $F\left( A\Rightarrow B \right)$ and $FA \Rightarrow FB$.

We will provide in Figure \ref{fig:sctypes} a list of the effect-less usual types associated to regular linguistic objects.

\subsubsection{Functors, Applicatives and Monads}\label{subsubsec:functors}
Our point of view leads us to consider \emph{language functors}\footnote{The elements of our language, not the categorical construct.} as polymorphic functions: for a (possibly restrained, though it seems to always be $\star$) set of base types $S$, a functor is a function
\begin{equation*}
	x: \tau\in S\subseteq \star \mapsto F x: F\tau
\end{equation*}
Remember that $\star$ is a fibration of the types in $\bar{\mC}$.
This means that if a functor can be applied to a type, it can also be applied to all \emph{affected} versions of that type, i.e. $\mF\left( L \right)(\tau\in \star)$.
More importantly, while it seems that $F$'s type is the identity on $\star$, the important part is that it changes the effects applied to $x$ (or $\tau$).
In that sense, $F$ has the following typing judgements:
\begin{equation*}
	\frac{\Gamma\vdash x: \tau \in \star_{0}}{\Gamma\vdash F x: F\tau \notin \star_{0}}\fracnotate{$\text{Func}_{0}$} \hspace{2cm} \frac{\Gamma\vdash x: \tau}{\Gamma\vdash Fx : F\tau\preceq \tau}\fracnotate{Func}
\end{equation*}
We use the same notation for the \emph{language functor} and the \emph{type functor} in the examples, but it is important to note those are two different objects, although connected.
More precisely, the \emph{language functor} is to be seen as a function whose computation yields an effect, while the \emph{type functor} is the endofunctor of $\bar{\mC}$ (so a functor from $\mC$) that represents the effect in our typing category.

In the same fashion, we can consider functions to have a type in $\star$ or more precisely of the form $\star \to \star$ which is a subset of $\star$.
This justifies how functors can act on functions in our typing system, thanks to the subtyping judgement introduced above, as this provides a way to ensure proper typing while just propagating the effects.
Because of propagation, this also means we can resolve the effects or keep on with the computation at any point during parsing, without any fear that the results may differ.

\medskip

In that sense, applicatives and monads only provide with more flexibility on the ways to combine functions:
they provide intermediate judgements to help with the combination of trees.
For example, the multiplication of the monad provides a new \emph{type conversion} judgement:
\begin{equation*}
	\frac{\Gamma\vdash x: MM\tau}{\Gamma\vdash x: M\tau \succeq MM\tau}\fracnotate{Monad}
\end{equation*}
This is actually a special case of the natural transformation rule that we define below, which means that, in a way, types $MM\star$ and $M\star$ are equivalent, as there is a canonical way to go from one type to another.
Remember however that $M\star$ is still a proper subtype of $MM\star$ and that the objects are not actually equal:
they are simply equivalent.

\subsubsection{Natural Transformations for Handlers and Higher-Order Constructs}\label{subsubsec:transnat}
We could also add judgements for adjunctions, but the most interesting thing is to add judgements for natural transformations, as adjunctions are particular examples of natural transformations which arise from \emph{natural} settings.
While in general we do not want to find natural transformations, we want to be able to express these in three situations:
\begin{enumerate}
	\item If we have an adjunction $L\dashv R$, we have natural transformations for $\Id_{\mC} \Rightarrow L \circ R$ and $R\circ L \Rightarrow \Id_{\mC}$.
	      In particular we get a monad and a comonad from a canonical setting.
	\item To deal with the resolution of effects, we can map handlers to natural transformations which go from some functor $F$ to the $\Id$ functor, allowing for a sequential\footnote{In particular, non-necessarily commutative} computation of the effects added to the meaning.
	      We will develop a bit more on this idea in Paragraph \ref{par:handlers} and in Section \ref{sec:nondet}.
	\item To create \emph{higher-order} constructs which transform words from our language into other words, while keeping the functorial aspect.
	      This idea is developed in \ref{par:higherorder}.
\end{enumerate}
To see why we want this rule, which is a larger version of the monad multiplication and the monad/applicative unit, it suffices to see that the diagram below provides a way to construct the ``correct function'' on the ``correct functor'' side of types. If we have a natural transformation \begin{tikzcd}
	F\ar[r, Rightarrow, "\theta"] & G
\end{tikzcd} then for all arrows $f: \tau_{1} \to \tau_{2}$ we have:
\begin{category}
	F\tau_{1}\ar[r, "Ff"]\ar[d, "\theta_{\tau_{1}}"'] & F\tau_{2}\ar[d, "\theta_{\tau_{2}}"]\\
	G\tau_{1}\ar[r, "Gf"] & G\tau_{2}
\end{category}
and this implies, from it being true for all arrows, that from $\cont x: F\tau$ we have an easy construct to show $\cont x: G\tau$.

Remember that in the Haskell programming language, any polymorphic function is a natural transformation from the first type constructor to the second type constructor, as proved by \newcite{wadlerTheoremsFree1989}.
This will guarantee for us that given a \emph{Haskell} construction for a polymorphic function, we will get the associated natural transformation.

\paragraph{Adjunctions}
\label{par:adjunctions}
We will not go in much details about adjunctions, as a full example and generalization process is provided in \ref{subsec:effects}.
First, we remind the definition: an adjunction $L \dashv R$ is a pair of functors $L: \A \to \B$ and $R: \B \to \A$, and a pair of natural transformations $\eta: \Id_{\A}  \Rightarrow R \circ L$ and $\epsilon: L\circ R \Rightarrow \Id_{\A}$ such that the two following equations are satisfied:
\begin{equation}
	\inputtikz{cd-zigzag}
	\tag{\emoji{cloud-with-lightning}}
	\label{eq:zigzag}
\end{equation}
\begin{equation}
	\inputtikz{cd-zagzig}
	\label{eq:zagzig}
	\tag{\reflectbox{\emoji{cloud-with-lightning}}}
\end{equation}

An adjunction defines two different structures over itself: a monad $L \circ R$ and a comonad $R\circ L$.
The fact these structures arise from the interaction between two effects renders them an intrinsic property of the language.
In this lies the usefulness of adjunction in a typing system which uses effects: adjunctions provide a way to combine effects and to handle effects, allowing to simplify the computations on the free monoid on the set of functors.

\paragraph{Handlers}
\label{par:handlers}
As introduce by \newcite{marsikAlgebraicEffectsHandlers}, the use of handlers as annotations to the syntactic tree of the sentence is an appropriate formalism.
This could also give us a way to construct handlers for our effects as per \newcite{bauerEffectSystemAlgebraic2014}, or \newcite{plotkinHandlingAlgebraicEffects2013}.
As considered by \newcite{wuEffectHandlersScope2014} and \newcite{vandenbergFrameworkHigherorderEffects2024}, handlers are to be seen as natural transformations describing the free monad on an algebraic effect.
Considering handlers as so, allows us to directly handle our computations inside our typing system, by ``transporting'' our functors one order higher up without loss of information or generality since all our functors undergo the same transformation.
Using the framework proposed in \cite{vandenbergFrameworkHigherorderEffects2024} we simply need to create handlers for our effects/functors and we will then have in our language the result needed.
The only thing we will require from an algebraic handler $h$ is that for any applicative functor of unit $\eta$, $h\circ \eta = \id$.

\medskip

What does this mean in our typing category ?
It means that either our language or our parser for the language \footnote{Depending whether we think handling effects is an intrinsic construct of the semantics of the language or whether it is associated with a speaker.} should contain natural transformations $F \Rightarrow \Id$ for $F \in \mF\left( \mL \right)$.
In this goal, we remember that from any polymorphic function in \emph{Haskell} we get a natural transformation \cite{wadlerTheoremsFree1989} meaning that it is enough to be able to define our handler in \emph{Haskell} to be ensured of its good definition.
Note that the choice of the handler being part of the lexicon or the parser over the other is a philosophical question more than a semantical one, as both options will result in semantically equivalent models, the only difference will be in the way we consider the resolution of effects.
Mathematically\footnote{Computer-wise, actually.}, this means the choice of either one of the options is purely of detail left during the implementation.

However, this choice does not arise in the case of the adjunction-induced handlers. Indeed here, the choice is caused by the non-uniqueness of the choices for the handlers.
For example, two different speakers may have different ways to resolve the $\f{S}$ (which will be introduced as the \emph{Powerset} monad in Section \ref{subsec:effects}) that arises from the phrase \textsl{A chair}.
This usual example of the differences between the cognitive representation of words is actually a specific example of the different possible handlers for the powerset representation of non-determinism/indefinites:
there are $\abs{S}$ arrows from the initial object to $S$ in $\mathit{Set}$, representing the different elements of $S$.
In that sense, while handlers may have a normal form or representation purely dependant on the effect, the actual handler does not necessarily have a canonical form.
This is the difference with the adjunctions: adjunctions are intrinsic properties of the coexistence of the effects\footnote{Which we identify to their functorial representations, which we may identify to their free monad in the framework \cite{vandenbergFrameworkHigherorderEffects2024}.}, while the handlers are user-defined.
As such, we choose to say that our handlers are implemented parser-side but again, this does not change our modelisation of handlers as natural transformations and most importantly, this does not add non-determinism to our model:
The non-determinism that arises from the variety of possible handlers does not add to the non-determinism in the parsing.

\paragraph{Higher-Order Constructs}
\label{par:higherorder}
We might want to add plurals, superlatives, tenses, aspects and other similar constructs which act as function modifiers.
For each of these, we give a functor $\Pi$ corresponding to a new class of types along with natural transformations for all other functors $F$ which allows to propagate down the high-order effect.
This transformation will need to be from $\Pi \circ F$ to $\Pi \circ F \circ \Pi$ or simply $\Pi \circ F \Rightarrow F \circ \Pi$ depending on the situation.
This allows us to add complexity not in the compositional aspects but in the lexicon aspects.
We do not want these functors to be applicatives, as we do not want a way to \emph{create} them in the parser if they are not marked in the sentence.

One of the main issues with this is the following:
In the English language, plural is marked on all words (except verbs, and even then case could be made for it to be marked),
while future is marked only on verbs (through the \textit{will + verb} construct which creates a ``\emph{new}'' verb in a sense) though it applies also to the arguments on the verb.
A way to solve this would be to include in the natural transformations rules to propagate the functor depending on the type of the object.
Consider the superlative effect \textbf{most}\footnote{We do not care about morphological markings here, we say that $\mathbf{largest} = \mathbf{most} \left(\mathbf{large}\right)$}.
As it can only be applied on adjectives, we can assume its argument is a function (but the definition would hold anyway taking $\tau_{1} = \bot$).
It is associated with the following function (which is a rewriting of the natural transformation rule):
\begin{equation*}
	\frac{\cont x: \tau_{1} \to \tau_{2}}{\cont \mathbf{most}\, x \coloneqq \Pi_{\tau_{2}} \circ x = x \circ \Pi_{\tau_{1}}}
\end{equation*}
What curryfication implies, is that higher-order constructs can be passed down to the arguments of the functions they are applied to, explaining how we can reconciliate the following semantic equation even if some of the words are not marked properly:
\begin{equation*}
	\bf future\left(be\left( I, a\ cat \right)\right) = will\ be\left( future\left( I \right), a\ cat \right) = be\left( future\left( I \right), a\ cat \right)
\end{equation*}
Indeed the above equation could be simply written by our natural transformation rule as:
\begin{equation*}
	\bf future\left( be \right)\left( arg_{1}, arg_{2} \right) = future\left( be \right)\left( arg_{2} \right)\left( future\left( arg_{1} \right) \right) = future \left( be \right) \left( future \left( arg_{2} \right) \right) \left( future \left( arg_{1} \right) \right)
\end{equation*}
This is not a definitive rule as we could want to stop at a step in the derivation, depending on our understanding of the notion of future in the language.

\subsection{Typing Judgements}\label{subsec:judgements}
To complete this section, Figure \ref{tab:judgements} gives a simple list of different typing composition judgements through which we also re-derzive the subtyping judgement to allow for its implementation.
\begin{figure}
	\inputtikz{typing-judgements}
	\caption{Typing and Subtyping Judgements}
	\label{tab:judgements}
\end{figure}
Note that here, the syntax is not taken into account: a function is always written left of its arguments, whether or not they are actually in that order in the sentence.
This issue will be resolved by giving the syntactic tree of the sentence (or infering it at runtime).
We could also add symmetry induced rules for application.

Furthermore, note that the App rule is the \texttt{fmap} rule for the identity functor and that the \texttt{pure/return} and \texttt{>>=} is the \texttt{nat} rule for the monad unit (or applicative unit) and multiplication.

\medskip

Using these typing rules for our combinators, it is important to see that our grammar will still be ambiguous and thus our reduction process will be non-deterministic.
As an example, we provide the typing reductions for the classic example: \textsl{The man sees the girl using a telescope} in Figure \ref{fig:ud}.

\begin{figure*}
	\centering
	\inputtikz{parse-tree-ex}
	\caption{Parsing trees for the typing of \textsl{The man sees the girl using a telescope}.}
	\label{fig:ud}
\end{figure*}

This non-determinism is a component of our language's grammar and semantics: a same sentence can have multiple interpretation without context.
By the same reasoning we applied on handlers, we choose not to resolve the ambiguity in the parser but we will try in Section \ref{sec:nondet} to reduce it to a minimum.
Even so some of our derivations will include ways to read from a context (for example the \textbf{Jupiter, a planet} construct), we forget about the definition of the context, or its updating after a sentence, in our first proposition.


\section{Handling Non-Determinism}
\label{sec:nondet}
The typing judgements proposed in Section \ref{subsec:judgements} lead to ambiguity.
In this section we propose ways to get our derivations to a certain normal form, by deriving an equivalence relation on our derivation and parsing trees, based on string diagrams.

\subsection{String Diagram Modelisation of Sentences}
\label{subsec:sd}
String diagrams are the Poincaré duals of the usual categorical diagrams when
considered in the $2$-category of categories and functors.
This means that we represent categories as regions of the plane, functors as
lines separating regions and natural transformations as the intersection points
between two lines.
We will always consider application as applying to the right of the line so
that composition is written in the same way as in equations.
This gives us a new graphical formalism to represent our effects using a few
equality rules between diagrams.
The commutative aspect of functional diagrams is now replaced by an equality of
string diagrams, which will be detailed in the following section.

The important aspect of string diagrams that we will use is that two diagrams that are planarily homotopic are equal \cite{joyalGeometryTensorCalculus1991} and that we can map a string diagram to a sequence of computations on computation: the vertical composition of natural transformations (bottom-up) represents the reductions that we can do on our set of effects.
This means that even when adding handlers, we get a way to visually see the meaning get reduced from effectful composition to propositional values, without the need to specify what the handler does.
Indeed we only look at \emph{when} we apply handlers and natural transformations reducing the number of effects.
This delimits our usage of string diagrams as ways to look at computations and a tool to provide equality rules to reduce non-determinism by constructing an equivalence relationship (which we denote by $\eqcirc$) and yielding a quotient set of \emph{normal forms} for our computations.

Let us define the category $\mathds{1}$ with exactly one object and one arrow: the identity on that object. It will be shown in grey in the string diagrams below.
A functor of type $\mathds{1} \to \mC$ is equivalent to choosing an object in $\mC$, and a natural between two such functors $\tau_{1}, \tau_{2}$ is exactly an arrow in $\mC$ of type $\tau_{1} \to \tau_{2}$.
Knowing that allows us to represent the type resulting from a sequence of computations as a sequence of strings whose farthest right represents an object in $\mC$, that is, a base type.
\begin{wrapfigure}{l}{.45\textwidth}
	\begin{center}
			\begin{tikzpicture}
		\path coordinate[dot, label=right:$\w{the}$] (the) + (0, 1) coordinate[dot, label=left:$\w{sleeps}$] (sleeps) + (0, 2) coordinate[label=above:$\t$] (bool)
		++(-2, 1) coordinate (ctlthe) + (0, 1) coordinate[label=above:$\f{M}$] (effthe)
		++(2, -2) coordinate[dot, label=left:$\w{cat}$] (cat) + (0, -2) coordinate[label=below:$\bot$] (bot);
		\draw (cat) -- (the) -- (sleeps) -- (bool);
		\draw[name path=effect] (the) to[out=180, in=-90] (ctlthe) -- (effthe);
		\draw[dashed] (bot) -- (cat);
		\begin{pgfonlayer}{background}
			\fill[catone] (bot) rectangle ($(bool) + (1, 0)$);
			\fill[catmca] (bot) rectangle ($(effthe) + (-1, 0)$);
			\fill[catmc] (the) to [out=180, in=-90] (ctlthe) -- (effthe) -- (bool) -- (the);
		\end{pgfonlayer}
	\end{tikzpicture}

	\end{center}
\end{wrapfigure}

For simplicity reasons, and because the effects that are buried in our typing system not only give rise to functors but also have types that are not purely currifiable, we will write our string diagrams on the fully parsed sentence, with its most simplified/composed expression.
Indeed, the question of providing rules to compose the string diagram for \textbf{the} and the one for \textbf{cat} to give the one for \textbf{the cat} is a difficult question, that natural solutions to are not obvious, and will be discussed in the next section.

\medskip

To justify our proposition to only consider fully reduced expressions note that
in this formalism we don't consider the expressions for our functors and
natural transformations but simply the sequence in which they are applied.
This works since the following diagrams commute for any $F, G$ functors and
$\theta$ natural transformation:
\begin{center}
	\begin{tikzcd}
		G\ar[r, "F"]\ar[d, "\theta"'] & F\circ G\ar[d, "F\circ \theta"] &[.5cm] G\ar[r, "F"]\ar[d, "\theta"'] & G\circ F \ar[d, "\theta\circ F"] \\
		\theta G\ar[r, "F"'] & F\circ \theta G & \theta G\ar[r, "F"'] & \theta G \circ F
	\end{tikzcd}
\end{center}
The property of natural transformations to be applied before or after any arrow in the category justifies that even when composing before the handling we get the same result.
These properties justifies the fact that we only need to prove the equality of
two reductions at the farthest step of the reductions, even though in practice
the handling might be done at earlier points in the parsing.

In the end, we will have the need to go from a certain set of strings (the effects that applied) to a single one, through a sequence of handlers, monadic and comonadic rules and so on.
Notice that we never reference the zero-cells and that in particular their colors are purely an artistical touch.

\medskip

There is however one thing that may seem to be an issue: existential quantifiers inside \textbf{if then} sentences and other functions looking like $\f{M}\ty{a} \to \ty{b}$.
Indeed, \cite{bumfordEffectdrivenInterpretationFunctors2025} suggests that those could be of type $\f{S}\t \to \t$ and that we might want to apply that function to something of type $\f{S}\t$ by first using the unit of the monad $\f{S}$ then invoking \texttt{fmap}.
There is no issue with that in our typing system, as this combination mode is
the composition of two typing judgements (in a sense).
To graphically reconciliate this and our string diagrams, especially since
units and handlers cancel each other, we suggest to see the effects as being
dragged along the parsing trees (or parsing string diagrams) until they're
never needed again.

\subsection{Achieving Normal Forms}
\label{subsec:normalforms}
We will now provide a set of rewriting rules on string diagrams (written as
equations) which define the set of different possible reductions.

First, Theorem \ref{thm:isotopy} reminds the main result by \cite{joyalGeometryTensorCalculus1991} about string diagrams which shows that our artistic representation of diagrams is correct and does not modify the equation or the rule we are presenting.
\begin{theorem}[Theorem 1.2 \cite{joyalGeometryTensorCalculus1991}]
	\label{thm:isotopy}
	A well-formed equation between morphism terms in the language of monoidal categories follows from the axioms of monoidal categories if and only if it holds, up to planar isotopy, in the graphical language.
\end{theorem}

Secondly, let us now look at a few of the equations that arise from the
commutation of certain class of diagrams:
\begin{description}
	\item[The \emph{Elevator} Equations] are a consequence of Theorem
	      \ref{thm:isotopy} but also highlight one of the most important properties
	      of string diagrams in their modelisation of multi-threaded computations:
	      what happens on one string does not influence what happens on another in
	      the same time.
	\item[The \emph{Snake} Equations] are a rewriting of the categorical diagrams
	      which are the defining properties of an adjunction.
	\item[The \emph{(co-)Monadic} Equations] are the string diagrammatic
	      translation of the properties of unitality and associativity of the
	      monad.
	      Similarly, there are co-monadic equations which are the categorical dual
	      of the previous equations.
\end{description}
This set of equations, when added to our reduction rules from the Section
\ref{sec:parsing} explain all the different reductions that can be made to
limit non-determinism in our parsing strategies.
Indeed, considering the equivalence relation $\mathcal{R}$ freely generated
from the equations defined above and the equivalence relationship
$\mathcal{R}'$ of planar isotopy from Theorem \ref{thm:isotopy}, we get a set
of normal forms $\mathcal{N}$ from the set of all well-formed parsing diagrams
$\mD$ defined by:
$\mathcal{N} = \left( \mD / \mathcal{R} \right) / \mathcal{R}'$


\subsection{Computing Normal Forms}
Now that we have a set of rules telling us what we can and cannot do in our model while preserving the equality of the diagrams, we provide a combinatorial description of our diagrams to help compute the possible equalities between multiple reductions of a sentence meaning.

\cite{delpeuchNormalizationPlanarString2022} proposed a combinatorial description to check
in linear time for equality under Theorem \ref{thm:isotopy}.
However, this model does not suffice to account for all of our equations, especially as
labelling will influence the equations for monads, comonads and adjunctions.
To provide with more flexibility (in exchange for a bit of time complexity) we use the
description provided and change the description of inputs and outputs of each $2$-cell by
adding types and enforcing types.
In this section we formally describe the data structure we propose, as well as algorithms for
validity of diagrams and a system of rewriting that allows us to compute the normal forms
for our system of effects.

\subsubsection{Representing String Diagrams}
We follow \cite{delpeuchNormalizationPlanarString2022} in their combinatorial description
of string diagrams. We describe a diagram by an ordered set of its $2$-cells (the natural
transformations) along with the number of input strings, for each $2$-cell the following
information:
\begin{itemize}
	\item Its horizontal position: the number of strings that are right of it (we adopt this
	      convention to match our graphical representation of effects: the number of strings
	      is the distance to the base type).
	\item Its type: an array of effects (read from left to right) that are the inputs to the
	      natural transformation and an array of effects that are the outputs to the natural
	      transformation. The empty array represents the identity functor.
	      Of course, we will not actually copy arrays and arrays inside our datastructure but
	      simply copy labels which are keys to a dictionary containing such arrays to limit
	      the size of our structure, allowing for $\O(1)$ access to the associated properties.
\end{itemize}
We will then write a diagram $D$ as a tuple of $3$ elements\footnote{We could write it
	as a tuple of $5$ elements by replacing $\dl{D}$ by three functions that lower level}:
$\left( D.N, D.S, D.L \right)$ where $D.N$ is a positive integers representing the
height (or number of nodes) of $D$, $D.S$ is an array for the input strings of $D$ and
where $D.L$ is a function which takes a natural number smaller than $D.N - 1$ and
returns its type as a tuple of arrays
$nat = \left( \dnlg{nat}, \dnin{nat}, \dnout{nat} \right)$.
From this, we can derive a naive algorithm to check if a string diagram is
valid or not.

\medskip

Since our representation contains strictly more information (without slowing
access by a non-constant factor) than the one it is based on, our
datastructure supports the linear and polynomial time algorithms proposed with
the structure by \cite{delpeuchNormalizationPlanarString2022}.
This in particular means that our structure can be normalized in polynomial
time to check for equality under Theorem \ref{thm:isotopy}.
More precisely, the complexity of our algorithm is in $\O\left( n \times
	\sup_{i} \abs{\dnin{\dlb{D}{i}}} + \abs{\dnout{\dlb{D}{i}}} \right)$, which
depends on our lexicon but most of the times will be linear time.

\subsubsection{Equational Reductions}
We are faced a problem when computing reductions using the equations for our diagrams
which is that by definition, an equation is symmetric.
To solve this issue, we only use equations from left to right to reduce as much as
possible our result instead.
This also means that trivially our reduction system computes normal forms: it suffices
to re-apply the algorithm for recumbent equivalence after the rest of equational
reduction is done.
Moreover, note that all our reductions are either incompatible or commutative, which
leads to a confluent reduction system, and the well definition of our normal forms:
\begin{theorem}[Confluence]\label{thm:confluence}
	Our reduction system is confluent and therefore defines normal forms:
	\begin{enumerate}
		\item Right reductions are confluent and therefore define \emph{right} normal forms for
		      diagrams under the equivalence relation induced by exchange.
		\item Equational reductions are confluent and therefore define \emph{equational}
		      normal forms for diagrams under the equivalence relation induced by exchange.
	\end{enumerate}
\end{theorem}

Before proving the theorem, let us first provide the reductions for the different
equations for our description of string diagrams.
\begin{description}
	\item[The Snake Equations]
	      First, let's see when we can apply the equation for $\id_{L}$ to a
	      diagram $D$ which is in \emph{right} normal form, meaning it's been
	      right reduced as much as possible.
	      Suppose we have an adjunction $L \dashv R$.
	      Then we can reduce $D$ along the equation at $i$ if, and only if:
	      \begin{itemize}
		      \item $\dnlg{\dlb{D}{i}} = \dnlg{\dlb{D}{i + 1}} - 1$
		      \item $\dlb{D}{i} = \eta_{L, R}$
		      \item $\dlb{D}{i + 1} = \epsilon_{L, R}$
	      \end{itemize}
	      This comes from the fact that we can't send either $\epsilon$
	      above $\eta$ (or the other way around) using right reductions and
	      that there cannot be any natural transformations between the two.
	      Obviously the equation for $\id_{R}$ works the same.
	      Then, the reduction is easy: we simply delete both strings,
	      removing $i$ and $i + 1$ from $D$ and reindexing the other nodes.
	\item[The Monadic Equations] For the monadic equations, we only use
	      the unitality equation as a way to reduce the number of natural
	      transformations, since the goal here is to attain normal forms
	      and not find all possible reductions.
	      We ask that associativity is always used in the direct
	      sense $\mu\left( \mu\left( TT \right),T \right) \to \mu\left(
		      T\mu\left( TT \right) \right)$ so that the algorithm terminates.
	      We use the same convention for the comonadic equations.
	      The validity conditions are as easy to define for the monadic
	      equations as for the \emph{snake} equations when considering
	      diagrams in \emph{right} normal forms.
	      Then, for unitality we simply delete the nodes
	      and for associativity we switch the horizontal
	      positions for $i$ and $i + 1$.
\end{description}

\begin{proof}[Proof of the Confluence Theorem]
	The first point of this theorem is exactly Theorem 4.2
	in \cite{delpeuchNormalizationPlanarString2022}.
	To prove the second part, note that the reduction process terminates as
	we strictly decrease the number of $2$-cells with each reduction.
	Moreover, our claim that the reduction process is confluent is obvious
	from the associativity equation and the fact the other
	equations delete nodes.
	Since right reductions do not modify the equational reductions, and thus
	right reducing an equational normal form yields an equational normal form,
	combining the two systems is done without issue, completing our proof of
	Theorem \ref{thm:confluence}.
\end{proof}


\begin{theorem}[Normalization Complexity]
	\label{thm:normalize}
	Reducing a diagram to its normal form is done in polynomial time in
	the number of natural transformations in it.
\end{theorem}
\begin{proof}
	Let's now give an upper bound on the number of reductions.
	Since each reductions either reduces the number of $2$-cells or applies the
	associativity of a monad, we know the number of reductions is linear in the
	number of natural transformations.
	Moreover, since checking if a reduction is possible at height $i$ is done in
	constant time, checking if a reduction is possible at a step is done in
	linear time, rendering the reduction algorithm quadratic in the number of
	natural transformations.
	Since we need to \emph{right} normalize before and after this method, and
	that this is done in linear time, our theorem is proved.
\end{proof}



\section{Efficient Semantic Parsing}
\label{sec:parsing}
In this section we explain our algorithms and heuristics for efficient semantic
parsing with as little ambiguity as possible, and reducing time complexity
of our parsing strategies.

\subsection{Syntactic-Semantic Parsing}
Using a naïve strategy of type checking on syntax trees yields an exponential
algorithm.
To avoid that, we extend the grammar system used to do the syntactic part of
the parsing to include semantic combination of words.
In this section, we will take the example of a CFG since it suffices to create
our typing combinators,
In Figure \ref{fig:combination-cfg}, we explicit a grammar of combination
modes, based on \cite{bumfordEffectdrivenInterpretationFunctors2025} as it
simplifies the rewriting of our typing judgements in a CFG.

\begin{figure}
	\centering
\newdimen\thecolsep
\thecolsep=-1cm
\advance\thecolsep by -50pt
\setlength{\columnsep}{\thecolsep}
\begin{multicols}{2}
	\begin{mgrammar}
		\firstrule{r}{\tau}{}
		\grule{T}{}
		\gskip
		\firstrule{\tau}{\pi}{}
		\grule{\phi}{}
		\grule{\bar{\tau}}{}
		\gskip
		\firstrule{\pi}{\scalar{r, r}}{}
		\gskip
		\firstrule{\phi}{r \to r}{}
		\gskip
		\firstrule{\bar{\tau}}{\e}{}
		\grule{\t}{}
		\grule{\vdots}{}
		\gskip
		\firstrule{T}{F \tau}{}
		\gskip
		\firstrule{F}{\mathcal{F}\left(\mL\right)^{*}}{}
		\gskip
	\end{mgrammar}
	\begin{mgrammar}
		\firstrule{B}{\mathrm{>} \enspace|\enspace \mathrm{<} \enspace|\enspace \wedge \enspace|\enspace \vee}{}
		\gskip
		\firstrule{M}{B}{}
		\grule{\combML, M \enspace|\enspace \combMR, M}{\fmap}
		\grule{\combUL, M \enspace|\enspace \combUR, M}{\tt pure}
		\grule{\combA, M}{\tt Struct}
		\grule{\combJ, M}{\tt join}
		\grule{\combC, M}{\tt counit}
		\grule{ER, M \enspace|\enspace EL, M}{\tt eject}
		\grule{DN, M}{\tt closure}
	\end{mgrammar}

	\def\arraystretch{1.2}
	\begin{mgrammar}
		\firstrule{\sobj}{M, r}{}
		\grule{>, (\alpha\to \beta) \to \alpha \to \beta}{}
		\grule{<, \alpha \to (\alpha \to \beta) \to \beta}{}
		\grule{\wedge, (\alpha \to \t) \to (\alpha \to \t) \to (\alpha \to \t)}{}
		\grule{\vee, (\alpha \to \t) \to (\alpha \to \t) \to (\alpha \to \t)}{}
		& \vdots & \\
		\firstrule{\combMR_{\f{F}} M, \f{F}\alpha \to \beta \to \f{F}\gamma}{M, \alpha \to \beta \to \gamma}{}
		\firstrule{\combML_{\f{F}} M, \alpha \to \f{F}\beta \to \f{F}\gamma}{M, \alpha \to \beta \to \gamma}{}
		\firstrule{\combA_{\f{F}} M, \f{F}\alpha \to \f{F}\beta \to \f{F}\gamma}{M, \alpha \to \beta \to \gamma}{}
		\firstrule{\combUR_{\f{F}} M, \alpha \to (\f{F}\beta \to \beta') \to \gamma}{\alpha \to (\beta \to \beta') \to \gamma}{}
	\end{mgrammar}
\end{multicols}

	\caption{Possible type combinations in the form of a near CFG. Here, $a, b\in \star_{0}$, $\alpha, \beta, \tau \in \star$ and $\f{F}, \f{L}, \f{R} \in \mF(\mL)$ with $\f{L} \dashv \f{R}$.}
	\label{fig:combination-cfg}
\end{figure}

This grammar works in five major sections:
\begin{enumerate}
	\item We reintroduce the grammar defining the type and effect system.
	\item We introduce a structure for the semantic parse trees and their labels,
	      based on the combination modes from
	      \cite{bumfordEffectdrivenInterpretationFunctors2025}.
	\item We introduce rules for basic type combinations.
	\item We introduce rules for higher-order unary type combinators.
	\item We introduce rules for higher-order binary type combinators.
\end{enumerate}

Each of these combinators can be, up to order, associated with a inference
rule, and, as such, with a higher-order denotation, which explains the actual
effect of the combinator, and are described in
Figure \ref{fig:combinator-denotations}.

\begin{figure}
	\centering
\begin{multicols}{2}
	\small
	$	>                    = \lambda \phi. \lambda x. \phi x $\\[1.5ex]
	$ <                    = \lambda x. \lambda \phi. \phi x $\\[1.5ex]
	$ \combML_{\f{F}}      = \lambda M. \lambda x. \lambda y. (\fmap_{\f{F}} \lambda a. M(a, y)) x $\\[1.5ex]
	$ \combMR_{\f{F}}      = \lambda M. \lambda x. \lambda y. (\fmap_{\f{F}} \lambda b. M(x, b)) y $\\[1.5ex]
	$	\combA_{\f{F}}       = \lambda M. \lambda x. \lambda y. (\fmap_{\f{F}}\lambda a. \lambda b. M(a, b))(x) \texttt{<*>} y $\\[1.5ex]
	$	\combUL_{\f{F}}      = \lambda M. \lambda x. \lambda \phi. M(x, \lambda b. \phi(\eta_{\f{F}} b))$\\[1.5ex]
	$	\combUR_{\f{F}}      = \lambda M. \lambda \phi. \lambda y. M(\lambda a. \phi(\eta_{\f{F}} a),y) $\\[1.5ex]
	$ \combJ_{\f{F}}       = \lambda M. \lambda x. \lambda y. \mu_{\f{F}} M(x, y) $\\[1.5ex]
	$\combC_{\f{L}\f{R}}  = \lambda M. \lambda x. \lambda y. \epsilon_{\f{L}\f{R}}(\fmap_{\f{L}}(\lambda l. \fmap_{\f{R}}(\lambda r. M(l, r))(y)) (x))$\\[1.5ex]
	$	\combEL_{\f{R}}      = \lambda M. \lambda \phi. \lambda y. M(\Upsilon_{\f{R}} \phi, y)$\\[1.5ex]
	$	\combER_{\f{R}}      = \lambda M. \lambda x. \lambda \phi. M(x, \Upsilon_{\f{R}} \phi)$\\[1.5ex]
	$	\combDN_{\Downarrow} = \lambda M. \lambda x. \lambda y. \Downarrow M(x, y)$
\end{multicols}

	\caption{Denotations describing the effect of the combinators used in the
		grammar describing our combination modes presented in
		Figure \ref{fig:combination-cfg}}
	\label{fig:combinator-denotations}
\end{figure}

The main reason why denotations associated to combinators are needed, is to
properly define how they actually do the combination of denotations.
Those denotations are a direct translation of the judgements defining the
notions of functors, applicatives, monads and thus are not specific to any
denotation system, even though we use lambda-calculus to describe them.
Some are duplicated for a left and right version to account for the fact CFGs
are not actually symmetric in their "input" unlike intuitionistic inference
rules.

This makes us able to compute the actual denotations associated to a sentence
using our formalism, as presented in Figure \ref{fig:parsing-trees}.
Note that the order of combination modes is not actually the same as the one
that would come from the grammar.

\begin{wrapfigure}[39]{l}{.45\textwidth}
	\centering
	\begin{subfigure}{.45\textwidth}
		\centering
		\resizebox{\textwidth}{!}{\begin{tikzpicture}[every tree node/.style={align=center, anchor=north}, level distance=2.5cm]
				\Tree [
				.{$\f{M}\f{D}\t$ \\ $\left\{\mathbf{eats}(\texttt{obj=}m, \texttt{subj=}c) \middle| \w{mouse}(m)\right\}$ if $\mathbf{cat}^{-1}(\top) = \{c\}$ \\ $\combMR_{\f{M}}\combML_{\f{D}}>$}
				[
				.{$\f{M}(\e)$ \\ $c$ if $\mathbf{cat}^{-1}(\top) = \{c\}$} \edge[roof]; {the cat}
				]
				[
				.{$\f{D}(\e \to \t)$ \\ $\left\{\lambda s. \mathbf{eats}(\texttt{obj=}m, \texttt{subj=}s) \middle| \w{mouse}(m)\right\}$} \edge[roof]; {eats a mouse}
				]
				]
			\end{tikzpicture}}
		\caption{Labelled tree representing the equivalent parsing diagram to
			\ref{fig:parsing-diagram}}
		\label{fig:tree-eats}
	\end{subfigure}

	\begin{subfigure}{.45\textwidth}
		\centering
		\begin{tikzpicture}[every tree node/.style={align=center, anchor=north}, level distance=2cm]
			\Tree [
			.{$\f{D}\e$ \\ $\{x \mid \w{cat} x \land \w{in\ a\ box} x \} $ \\ $\combJ_{\f{D}}\combML_{\f{D}} >$}
			{$(\e \to \t) \to \f{D}\e$ \\ a}
			[ .{$\f{D}(\e \to \t)$ \\ $\lambda x. \w{cat} x \land \w{in\ a\ box} x$} \edge[roof]; {cat in a box} ] ]
		\end{tikzpicture}
		\caption{Labelled tree representing the equivalent parsing diagram to
			\ref{fig:parsing-diagram2}}
		\label{fig:tree-box}
	\end{subfigure}

	\begin{subfigure}{.45\textwidth}
		\centering
		\resizebox{\textwidth}{!}{\begin{tikzpicture}[every tree node/.style={align=center, anchor=north}, level distance=1.75cm]
				\Tree [
				.\node{\t \\ $\mathbf{if}(\forall x. \w{past}\w{pass} x)(\w{past}\mathbf{rain})$ \\ $>$};
				[ .\node{$\t \to \t$ \\ $\mathbf{if}(\forall x.\w{past}\w{pass} x)$ \\ $>$};
				{$\t \to \t \to \t$ \\ if}
				[
				.\node{$\t$\\ $\forall x. \w{pass} x$ \\ $\combDN_{\Downarrow_{\f{C}}}$};
				[ .\node{$\f{C}\t$ \\ $\lambda c.\forall x. c(\w{past}\w{pass} x)$ \\ $\combMR_{\f{C}}<$};
				{$\f{C}\e$ \\ $\lambda c. \forall x. c\, x$ \\ everyone}
				{$\e \to \t$ \\ $\w{past}\mathbf{pass}$ \\ passed}
				]
				]
				]
				[
				.{\t \\ $\w{past}\mathbf{rain}$} \edge[roof]; {it was raining}
				]
				]
			\end{tikzpicture}}
		\caption{Labelled tree representing the equivalent parsing diagram to
			\ref{fig:3dparsing-diagram}}
		\label{fig:tree-rain}
	\end{subfigure}
	\caption{Examples of labelled parse trees for a few sentences.}
	\label{fig:parsing-trees}
\end{wrapfigure}

The reason why will become more apparent when string diagrams for parsing are
introduced in the next section, but simply, this comes from the fact that while
we think of $\combML$ and $\combMR$ as reducing the number of effects on each
side (and this is the correct way to think about those), this is not actually
how its denotation works, they are actually modifying a combination mode via
their denotation.
This formalism gives us the following theorems:

\begin{theorem}
	\label{thm:ptime-parse}
	Parsing of a sentence with combination modes is polynomial in the length of
	the	sentence and the size of the type system and syntax system.
\end{theorem}

\begin{proof}
	Suppose we are given a syntactic generating structure $G_{s}$ along with our
	type combination grammar $G_{\tau}$.
	The syntactico-semantic system $G$ constructed from the product of $G_{s}$ and
	$G_{\tau}$ has size $\abs{G_{s}}\times \abs{G_{\tau}}$.
	Computing membership of a sentence to the language generated by $G$, is then
	in polynomial time if, and only if, finding membership to the language
	generated by $G_{s}$ is done in polynomial time.
	Parsing the sentence is then done in polynomial time in the size of the
	input, $\abs{G_{s}}$ and
	$\abs{G_{\tau}} = \O\left(\abs{\mF\left(\mL\right)} + \abs{\Obj\left(\mC\right)}\right)$.
\end{proof}

\begin{theorem}
	\label{thm:ptime-denot}
	Retrieving a pure denotation for a sentence is polynomial in
	the length of the sentence, given a polynomial time syntactic parsing
	structure and polynomial combinator denotations.
\end{theorem}

\noindent To prove this theorem we need a short lemma on the size of the trees generated
through our structure:
\begin{lemma}
	\label{lem:quad-tree}
	Semantic parsing trees are quadratic in the length of the sentence.
\end{lemma}

\begin{proof}
	Let $m_{\mL}$ be the maximum number of effects created by a word in $\mL$.
	Since at any step $i$ in the parsing, there can never be more than
	$m_{\mL}\times i$ effects borne by the considered inputs, there is no need
	for more than $(2 + c) \times m_{\mL}\times (i + 1) + 1$ combinators where
	$c$ is a constant dependent only on the language.
	Indeed, we will have at most one combinator among
	$\{\combML, \combMR, \combA, \combUR, \combUL\}$ per input effect, at most one
	of $\combJ$ and $\combDN$ per output effects
	($m_{\mL} \times (i + 1)$ at most),	at most a fixed number $c$ of modes
	between $\{\combC, \combEL, \combER\}$ which depends only on the number of
	adjunctions in the language.
	We get the wanted upper bound when adding the \emph{base combinator}.
	Summing the steps for $i$, we get a quadratic upper bound on the number of
	combinators and thus on the tree size.
\end{proof}

\begin{figure}
	\centering
	\begin{equation*}
	\begin{tikzpicture}[baseline={([yshift=-.5ex]current bounding box.center)}]
		\path coordinate[dot, label=below:$>$] (m)
		+ (0, 1) coordinate[label=above:$\beta$] (result)
		+ (-1, -1) coordinate[label=below:$\alpha \to \beta$] (phi)
		+ (1, -1) coordinate[label=below:$\alpha$] (x);
		\draw (m) -- (result);
		\draw (m) to[out=180, in=90] (phi);
		\draw (m) to[out=0, in=90] (x);
		\begin{pgfonlayer}{background}
			\fill[catmcb] (result) -- (m) -- ($(m) + (-1, 0)$) |- (result);
			\fill[catmcb] (result) -- (m) to[out=180, in=90] (phi) -- ($(phi) + (-1, 0)$) |- (result);
			\fill[catmc] (result) -- (m) -- ($(m) + (1, 0)$) |- (result);
			\fill[catmc] (result) -- (m) to[out=0, in=90] (x) -- ($(x) + (1, 0)$) |- (result);
			\fill[catmca] (m) to[out=0, in=90] (x) -- (phi) to[out=90, in=180] (m);
		\end{pgfonlayer}
	\end{tikzpicture}
	\hspace{2.5cm}
	\begin{tikzpicture}[baseline={([yshift=-.5ex]current bounding box.center)}]
		\path coordinate[dot, label=below:$\combML_{\f{F}}$] (m)
		+ (1, 1) coordinate[label=above:$\beta$] (out2)
		+ (-1, 1) coordinate[label=above:$\alpha$] (out)
		+ (-1.5, 1) coordinate[label=above:$\f{F}$] (out1)
		+ (-1, -1) coordinate[label=below:$\alpha$] (phi)
		+ (1, -1) coordinate[label=below:$\beta$] (x);
		\draw (m) to[out=180, in=90] (phi);
		\draw (m) to[out=0, in=90] (x);
		\draw (m) to[out=180, in=-90] (out1);
		\draw (m) to[out=180, in=-90] (out);
		\draw (m) to[out=0, in=-90] (out2);
		\begin{pgfonlayer}{background}
			\fill[catmcc] (m) to[out=180, in=90] (phi) -- ($(phi) + (-1, 0)$) |- (out1) to[out=-90, in=180] (m);
			\fill[catmc] (m) to[out=0, in=-90] (x) -- ($(x) + (1, 0)$) |- (out2) to[out=-90, in=0] (m);
			\fill[catmca] (m) to[out=180, in=90] (phi) -- (x) to[out=90, in=0] (m);
			\fill[catmcb] (m) to[out=180, in=-90] (out1) -- (out) to[out=-90, in=180] (m);
			\fill[catmca] (m) to[out=180, in=-90] (out) -- (out2) to[out=-90, in=0] (m);
		\end{pgfonlayer}
	\end{tikzpicture}
	\hspace{2.5cm}
	\begin{tikzpicture}[baseline={([yshift=-.5ex]current bounding box.center)}]
		\path coordinate[dot, label=below:$\combJ_{\f{F}}$] (m)
		+ (0, 1) coordinate[label=above:$\f{F}$] (result)
		+ (-1, -1) coordinate[label=below:$\f{F}$] (phi)
		+ (1, -1) coordinate[label=below:$\f{F}$] (x);
		\draw (m) -- (result);
		\draw (m) to[out=180, in=90] (phi);
		\draw (m) to[out=0, in=90] (x);
		\begin{pgfonlayer}{background}
			\fill[catmcb] (result) -- (m) -- ($(m) + (-1, 0)$) |- (result);
			\fill[catmcb] (result) -- (m) to[out=180, in=90] (phi) -- ($(phi) + (-1, 0)$) |- (result);
			\fill[catmc] (result) -- (m) -- ($(m) + (1, 0)$) |- (result);
			\fill[catmc] (result) -- (m) to[out=0, in=90] (x) -- ($(x) + (1, 0)$) |- (result);
			\fill[catmca] (m) to[out=0, in=90] (x) -- (phi) to[out=90, in=180] (m);
		\end{pgfonlayer}
	\end{tikzpicture}
\end{equation*}

	\caption{String Diagrammatic Representation of Combinator Modes $>, \combML$ and $\combJ$}
	\label{fig:combinator-sd}
\end{figure}

\begin{wrapfigure}[29]{r}{.45\textwidth}
	\centering
	\includegraphics[width=.45\textwidth]{parsing-diagram}
	\caption{Representation of a parsing diagram for the sentence
		\emph{the cat eats a mouse}.
		See Figure \ref{fig:tree-box} for translation in a parse tree.}
	\label{fig:parsing-diagram}
\end{wrapfigure}

We can now return to the proof of the main result of this section:
\begin{proof}[Proof of Theorem \ref{thm:ptime-denot}]
	From Theorem \ref{thm:ptime-parse} we can retrieve a
	semantic parse tree from a sentence in polynomial time in the input.
	Lemma \ref{lem:quad-tree} states that we have a polynomial number of
	combinator denotations to apply, all done in polynomial time by hypothesis.
	We have already seen that given a denotation, handling all effects and
	reducing effect handling to normal forms can be done in polynomial time.
	The sequencing of these steps yields a polynomial-time algorithm in the
	length of the input sentence.
\end{proof}

While we have gone the assumption that we have a CFG for our language,
any type of polynomial-time structure could work, as long as it is at least
as expressive as a CFG.

The \emph{polynomial time combinators} assumption in Theorem
\ref{thm:ptime-denot} is not a complex assumption, this is for example true for
denotations based on lambda-calculus, with function application being linear in
the number of uses of variables in the function term, which in turn is linear
in the number of terms used to construct the function term and thus of words,
and the different \fmap{} being in polynomial time for the same reason.
This would also be true for denotations inspired by machine learning for
example.

\subsection{Diagrammatical Parsing}
When considering \cite{coeckeMathematicalFoundationsCompositional2010}
way of using string diagrams for syntactic parsing/reductions, we can see
string diagrams as (yet) another way of writing our parsing rules.
They are an expanded rewriting of labelled parsing trees\footnote{Point of view
	which connects this formalism nicely to the one of
	\cite{senturiaAlgebraicStructureMorphosyntax2025}, preserving all their
	results inside our theory.} presented in
\cite{bumfordEffectdrivenInterpretationFunctors2025}, .
In our typed category, we can see our combinators as natural transformations
($2$-cells): then we can see the different sets of combinators as different
arity natural transformations.
Combinators $>$, $\combML_{\f{F}}$ and $\combJ_{\f{F}}$ are represented in
Figure \ref{fig:combinator-sd}.
The coloring of the regions is purely for artistic rendition and will not be
used for larger diagrams.

\begin{wrapfigure}{r}{.45\textwidth}
	\centering
	\includegraphics[width=.45\textwidth]{parsing-diagram2.pdf}
	\caption{Example of a parsing diagram for the phrase
		\emph{a cat in a box}, presenting the integration of unary combinators
		inside the connector line. See Figure \ref{fig:tree-box} for translation in
		a parse tree.}
	\label{fig:parsing-diagram2}
\end{wrapfigure}
Understanding the diagrams could be thinking of them on an orthogonal plane to
the ones of Section \ref{sec:nondet}: we could use the syntactic version of the
diagrams to model our parsing, according to the rules in Figure
\ref{fig:combination-cfg}, and then combine the diagrams as shown in Figure
\ref{fig:parsing-diagram}, which highlights the \emph{orthogonal} components.
In this diagram we exactly see the sequence of combinations play out on the
types of the words, and thus we also see what exact \emph{stitch} would
be needed to construct the effect diagram.
Here we talk about \emph{stitches} because, in a sense, we use $2$-cells
to do braiding-like operations on the strings, and don't actually allow for
braiding inside the diagrammatic computation, leading to the intervention of
outside tools (combinators) which serve as \emph{knitting needles}.
To better understand what happens in those parsing diagrams, Figure
\ref{fig:parsing-trees} provides the translations in labelled trees of the
parsing diagrams of Figures \ref{fig:parsing-diagram},
\ref{fig:parsing-diagram2} and \ref{fig:3dparsing-diagram}.


For the combinators $\combJ$, $\combDN$ and $\combC$, which are applied to
reduce the number of effects inside a denotation, it might seem less obvious
how to include them.
Applying them to the actual \emph{parsing} part of the diagram is done
in the exact same way as in the CFG: we just add them where needed, and they
will appear in the resulting denotation as a form of forced handling, in a
sense, as shown in the result of Figure \ref{fig:parsing-diagram2}.
It is interesting to note that the resulting diagram representing
the sentence can visually be found in the connection strings that arise from
the combinators.

\smallskip

Categorically, we start from a meaning category $\mC$, our typing category, and
take it as our grammatical category.
This is a form of extension on the monoidal version by
\cite{coeckeMathematicalFoundationsCompositional2010} and
\cite{toumiHigherOrderDisCoCatPeirceLambekMontague2023}, as it is seemingly a
typed version, where we change the pregroup category for the typing category,
taken with a product for representation of the English grammar representation,
to accommodate for syntactic typing on top of semantic typing if it does not
already encompass it.
We have a first plane of string diagrams in the category
$\mC$ - our string diagrams for effect handling, as in Section
\ref{sec:nondet} - and the second \emph{orthogonal} plane of string diagrams
on a larger category, with formal endofunctors labelled by the types in our
typing category $\bar{\mC}$ and formal natural transformations for the
combinators defined in Figures \ref{fig:combination-cfg} and
\ref{fig:combinator-denotations}.
\begin{wrapfigure}{r}{.45\textwidth}
	\centering
	\begin{tikzpicture}
		\node (fig) at (0, 0) {\includegraphics[width=.4\textwidth]{knitting-example}};
		\draw[->] ($(fig.south west) + (-.1, -.1)$) -- node[anchor=east] {\rotatebox{90}{Direction of Reduction}} ($(fig.north west) + (-.1, .1)$);
	\end{tikzpicture}
	\caption{Example of a \emph{Jacquard} knitwork. Photography and work courtesy
		of the author's mother.}
	\label{fig:knitting-example}
\end{wrapfigure}

The category in which we consider the second-axis string diagrams does not have
a meaning in our compositional semantics theory, and to be more precise, we
should talk about $1$-cells and $2$-cells instead of endofunctors and natural
transformations, to keep in the idea that this is really just a diagrammatic
way of computing and presenting the operations that are put to work during
semantic parsing.

The main theoretical reason why this point of view of diagrammatic parsing is
useful will be clear when looking at the rewriting rules and the normal forms
they induce, because, as stated in Theorem \ref{thm:norm}, string
diagrams make it easy to compute normal forms when provided with a confluent
reduction system.
However, the just as useful graphical interpretation of string diagrams as
easy to read expanded labelled parsing trees.
Using orthogonal planes to visualise this interpretation cannot be well
presented in a 3D space, and even less so on a page, so we suggest an
interpretation based on actual strings:
Suppose you're knitting a rainbow scarf.

\begin{wrapfigure}[22]{l}{.5\textwidth}
	\centering
	\includegraphics[width=.5\textwidth]{3d-parsing-diagram}
	\caption{Knitting-like representation of the diagrammatic parsing of a sentence. See Figure \ref{fig:tree-rain} for the translation in a parse tree}
	\label{fig:3dparsing-diagram}
\end{wrapfigure}

You have multiple threads (the different words) of the different colours (their
types and effects) you're using to knit the scarf.
When you decide to change the color, you take the different threads you have
been using, and mix them up.
You can create a new colour\footnote{This is not how wool works, but
	one can also imagine a pointillist-like way of drawing using multiple
	coloured lines that superimpose on each other, or a marching band's multiple
	instruments playing either in harmony or in disharmony and changing that
	during a score.} thread from two (that's the base combinators).
Creating a thicker one from two of the same colour is the result of the
applicative mode and the monadic join.
$\fmap$ puts aside a thread until a later step, the monadic unit adds a new
thread to the pattern, and the co-unit and closure operators cut a thread which
will no longer be used.
Changing a thread by cutting it and making a knot at another point is what the
eject combinators do.

This more tangible representation can be seen in Figure
\ref{fig:3dparsing-diagram}.
The sections in the rectangle represent what happens when considering our
combination step as implementing patterns inside a knitwork, as seen in
Figure~\ref{fig:knitting-example}.
The different patterns provide, in order, a visual representation of the
different ways one can combine two strings, i.e., two types and thus two
denotations.
The sections outside of the rectangle are the strings of yarn not currently
being used to make a pattern.

\subsection{Rewriting Rules}
\label{subsec:rewrite}
In this section we study reductions for our diagrams that allows us
to improve our time complexity by reducing the size of the grammar.
This is done by looking at equations on sequences of combinators.
In the worst case, there is no improvement in big o notation in the size of the
sentence, but there is no loss.

\noindent Consider the case where we have the two arguments of our parsing step of
type $\f{F}\tau$ and $\f{G}\tau'$.
In that case we could either get a result with effects $\f{F}\f{G}$ or
with effects $\f{G}\f{F}$.
If those effects happen to be equal, which trivially will be the case when one
of the effects is external (the plural or islands functors for example), the
order of application does not matter and we choose to get the effect on the
left side of the combinator first: $\combML_{\f{F}}\combMR_{\f{G}}$ over
$\combMR_{\f{F}}\combML_{\f{F}}$.

\noindent There are sequence of modes that clearly encompass other ones
the grammar notation for ease of explanation.
One should not use the unit of a functor after using $\combML$ or $\combMR$, as
that adds void semantics.
Same things can be said for certain other derivations containing the lowering
and co-unit combinators since they could in theory be applied at many points
inside the derivation.

\noindent We use $\combDN$ when we have not used any of the following, in all
derivations:
\let\mcolsep=\multicolsep
\setlength{\multicolsep}{.4\mcolsep}
\begin{multicols}{2}
	\begin{itemize}
		\item $m_{\f{F}}, \combDN, m_{\f{F}}$ where
		      $m \in \{\combMR, \combML\}$
		\item $\combML_{\f{F}}, \combDN, \combMR_{\f{F}}$
		\item $\combA_{\f{F}}, \combDN, \combMR_{\f{F}}$
		\item $\combML_{\f{F}}, \combDN, \combA_{\f{F}}$
		\item $\combC$
	\end{itemize}
\end{multicols}
\noindent We use $\combJ$ if we have not used any of the following,
for $j \in \{\epsilon, \combJ_{\f{F}}\}$
\begin{multicols}{2}
	\begin{itemize}
		\item $\left\{m_{\f{F}}, j, m_{\f{F}}\right\}$ where
		      $m \in \{\combMR, \combML\}$
		\item $\combML_{\f{F}}, j, \combMR_{\f{f}}$
		\item $\combA_{\f{F}}, j, \combMR_{\f{F}}$,
		\item $\combML_{\f{F}}, j, \combA_{\f{F}}$
		\item $k, \combC$ for $k \in \{\epsilon, \combA_{\f{F}}\}$
		\item If $\f{F}$ is commutative as a monad:
		      \begin{itemize}
			      \item $\combMR_{\f{F}}, \combA_{\f{F}}$
			      \item $\combA_{\f{F}}, \combML_{\f{F}}$
			      \item $\combMR_{\f{F}}, j, \combML_{\f{F}}$
			      \item $\combA_{\f{F}}, j, \combA_{\f{F}}$
		      \end{itemize}
	\end{itemize}
\end{multicols}

\begin{theorem}
	The rules proposed above yield equivalent results.
\end{theorem}

\begin{proof}
	The rules about not using combinators $\combUL$ and $\combUR$ come from the
	notion of handling and granting termination and decidability to our system.
	The rules about adding $\combJ$ and $\combDN$ after moving two of the same
	effect from the same side (i.e. $\combML \combML$ or $\combMR\combMR$) are
	normalization along Theorem \ref{thm:isotopy}: the only reason to keep two of
	the same effects and not join them is to at some point	have something get in
	between the two.
	Joining and closure should then be done at earliest point in parsing where it
	can be done, and that is equivalent to later points because of Theorem
	\ref{thm:isotopy}.
	The last set of rules follows from the following: we should not use $\combJ
		\combML \combMR$ instead of $\combA$, as those are equivalent because of the
	equation defining them.
	The same thing goes for the other two, as we should use the units of monads
	over applicative rules and \fmap.
\end{proof}

The reductions described above amount to equational reductions for the string
diagrams, as they are equivalent to specific sequences of $2$-cells.
This leads to the same algorithms developed in Section \ref{sec:nondet} being
usable here: we just have a new improved version of Theorem
\ref{thm:confluence}: computing two different normal forms along the
tensor product of our reduction schemes, which amounts to computing a larger
normal form.
Theorem \ref{thm:norm} still stands with the improved system and thus, proving
two parses are equal can be done in polynomial time.
Moreover, considering the possible normal forms of syntactic reductions
or denotational reductions adds ways to reduce our diagrams to normal forms.


\section{Language Translation}
\label{sec:language}
In this section we will give a list of words along with a way to express them as either arrows or endo-functors of our typing category.
This will also give a set of functors and constructs in our language.

\subsection{Syntax}
We don't talk about syntactic production rules or any system of the sort: it exists, we take it for granted.
The only discussion of translating syntax will be found in Appendix \ref{app:prodgram}.

\subsection{Lambda-Calculus}\label{subsec:lambdacalc}
In this section we use the traditional lambda-calculus denotations.

\subsubsection{Types of Syntax}\label{subsec:syntax}
We first need to setup some guidelines for our denotations, by asking ourselves how the different components of sentences interact with each other.
For syntactic categories, the types that are generally used are the following, and we will see it matches with our lexicon, and simplifies our functorial definitions\footnote{We don't consider effects in the given typings.}.
Those are based on \newcite{parteeLecture2Lambda}. Here $\Upsilon$ is the operator which retrieves the type from a certain syntactic category.
\begin{figure}
	\centering
	\inputtikz{syntactic-categories}
	\caption{Usual Typings for some Syntactic Categories}
	\label{fig:sctypes}
\end{figure}
\subsubsection{Lexicon: Semantic Denotations for Words}\label{subsec:lexicon}
Many words will have basically the same ``high-level'' denotation.
For example, the denotation for most common nouns will be of the form: $\cont \lambda x. \w{planet} x: \e \to \t$.
In Table \ref{fig:lexicon} we give a lexicon for a subset of the english language.
We describe the constructor for the functors used by our denotations in the table, but all functors will be reminded and properly defined in Table \ref{fig:functors} along with their respective \fmap.
\begin{figure}
	\centering
	\inputtikz{lexicon-table}
	\caption{$\lambda$-calculus representation of the english language $\mL$}
	\label{fig:lexicon}
\end{figure}
Note that for terms with two lambdas, we also say that inverting the lambdas also provides a valid denotation.
This is important for our formalism as we want to be able to add effects in the order that we want.

\subsubsection{Effects of the Language}\label{subsec:effects}
For the applicatives/monads in Table \ref{fig:functors} we do not specify the unit and multiplication functions, as they are quite usual examples.
We still provide the \fmap{} for good measure.

\begin{figure}
	\centering
	\inputtikz{functors-table}
	\caption{Denotations for the functors used}
	\label{fig:functors}
\end{figure}

Let us explain a few of those functors: $\f G$ designates reading from a certain environment of type $\r$ while $\f W$ encodes the possibility of logging a message of type $\t$ along with the expression.
The functor $\f M$ describes the possibility for a computation to fail, for example when retrieving a value that does not exist (see $\mathbf{the}$).
The $\f S$ functor represents the space of possibilities that arises from a non-deterministic computation (see $\mathbf{which}$).

We can then define an adjunction between $\f G$ and $\f W$ using our definitions:
\begin{equation*}
	\phi: \begin{array}{l}
		\left( \alpha \to \f{G}\beta \right)\to \f{W}\alpha \to \beta \\
		\lambda k. \lambda \left( a, g \right) . k a g
	\end{array}
	\text{ and }
	\psi: \begin{array}{l}
		\left( \f{W}\alpha \to \beta \right) \to \alpha \to \f{G} \beta \\
		\lambda c \lambda a \lambda g. c \left( a, g \right)
	\end{array}
\end{equation*}
where $\phi \circ \psi = \id$ and $\psi \circ \phi = \id$ on the properly defined sets.
Now it is easy to see that $\phi$ and $\psi$ do define a natural transformation and thus our adjunction $\f{W} \dashv \f{G}$ is well-defined, and that its unit $\eta$ is $\psi \circ \id$ and its co-unit $\epsilon$ is $\phi \circ \id$.

As a reminder, this means our language canonically defines a monad $\f{G}\circ \f{W}$.
Moreover, the co-unit of the adjunction provides a canonical way for us to deconstruct entirely the comonad $\f{W}\circ \f{G} $, that is we have a natural transformation from $\f{W} \circ \f{G} \Rightarrow \Id$.

\subsubsection{Modality, Plurality, Aspect, Tense}\label{subsec:modality}
We will now explain ways to formalise higher-order concepts as described in Section \ref{par:higherorder}.

First, let us consider in this example the plural concept.
Here we do not focus on whether our denotation of the plural is semantically correct, but simply on whether our modelisation makes sense.
As such, we give ourselves a way to measure if a certain $x$ is plural our not, which we will denote by $\abs{x} \geq 2$\footnote{We chose this representation as an example, it is not important for our formalism that this denotation is actually a good choice for the plural.}.
It is important to note that if $\tau$ is a type, and $\cont x: \tau$ then we should have $\abs{\Pi x} \geq 1 = \top\footnote{This might be seen as a typing judgement~!}$.
We then need to define the functor $\Pi$ which models the plural. We do not need to define it on types in any way other than $\Pi\left( \tau \right) = \Pi\tau$.
We only need to look at the transformation on arrows\footnote{\texttt{fmap}, basically.}.
This one depends on the \emph{syntactic}\footnote{Actually it depends on the word considered, but since the denotations (when considered effect-less) provided in Table \ref{fig:lexicon} are more or less related to the syntactic category of the word, our approximation suffices.} type of the arrow, as seen in Table \ref{fig:lexicon}.
\begin{figure}
	\centering
	\inputtikz{plural-table}
	\caption{(Partial) Definition for the $\Pi$ Plural Functor}
	\label{fig:pluralfunctor}
\end{figure}
There is actually a presupposition in our definition. Whenever we apply $\Pi$ to an arrow representing a predicate $p: \e \to \t$, we still apply $p$ to its argument $x$ even though we say that $\Pi p$ is the one that applies to a plural entity $x$.
This slight notation abuse results from our point of view on plural/its predicate representation: we assume the predicate $p$ (the common noun, the adjectve, the VP\ldots) applies to an object without regarding its cardinality.
The two point of views on the singular/plural distinction could be adapted to our formalism: whether we believe that singular is the \emph{natural} state of the objects or that it is to be always specified in the same way as plural does not change anything:
In the former, we do not change anything from the proposed functors.
In the latter we simply need to create another functor $\Sigma$ with basically the same rules that represents the singular and ask of our predicates $p$ to be defined on $\left(\Pi\star\right) \sqcup \left(\Sigma \star\right)$.

\medskip

See that our functor only acts on the words which return a boolean by adding a plurality condition, and is simply passed down on the other words.
Note however there is an issue with the \textbf{NP} category: in the case where a \textbf{NP} is constructed from a determiner and a common noun (phrase) the plural is not passed down.
This comes from the fact that in this case, either the determiner or the common noun (phrase) will be marked, and by functoriality the plural will go up the tree.

Now clearly our functor verifies the identity and composition laws and works as theorised in Section \ref{par:higherorder}.
To understand a bit more why this works as described in our natural transformation rules, consider $\Pi\left( \mathbf{which} \right)$.
The result will be of type $\Pi\left( \f{S}\left( e \right) \right) = \left(\Pi \circ \f{S}\right) \left( e \right)$.
However, when looking at our transformation rule (and our functorial rule) we see that the result will actually be of type $\f{S}\left( \Pi e \right)$ which is exactly the expected type when considering the phrase $\w{which} p$.
The natural transformation $\theta$ we set is easily inferable:
\begin{equation*}
	\theta_{A}:
	\begin{tikzcd}
		\left( \Pi \circ \f{S} \right) A \ar[r, "\theta_{A}"] & \left( \f{S} \circ \Pi \right) A\\[-.7cm]
		\Pi\left( \left\{ x \right\} \right) \ar[r, mapsto] &  \left\{ \Pi(x) \right\}
	\end{tikzcd}
\end{equation*}
The naturality follows from the definition of $\fmap$ for the $\f{S}$ functor.
We can easily define natural transformations for the other functors in a similar way: $\Pi \circ \f{F} \overset{\theta}{\Rightarrow} \f{F} \circ \Pi$ defined by $\theta_{\tau} = \fmap_{\f{F}}\left( \Pi \right)$.

\medskip

Now, in quite a similar way, we can create functors from any sort of judgement that can be seen as a function $\e \to \t$ in our language\footnote{An easy way to define those would be, similarly to the plural, to define a series of judgement from a property that could be infered.}.
Indeed, we simply need to replace the $\abs{p} \geq 1$ in most of the definitions by our function and the rest would stay the same.
Remember that our use of natural transformations is only there to allow for possible under-markings of the considered effects and propagating the value resulting from the computation of the high-order effect.

\subsection{Other Representation of Concepts}
In this section we provide explanations on how to translate other semantic descriptions of the lexicon inside our extended type system.

\subsubsection{Manifold Representation}
\newcite{goodaleManifoldsConceptualRepresentations2022} proposes a way to describe the words of the lexicon (concepts, actually) as manifold.

\subsubsection{Vector Representation}
\newcite{sadrzadehStaticDynamicVector2018} provides a way to translate vector embeddings to a typed-lambda-calculus system, and as such to a cartesian closed category, meaning we could easily apply the formalism described in Section \ref{subsec:lambdacalc}.

\subsubsection{Probabilistic Monad}
\newcite{groveProbabilisticCompositionalSemantics2023}


\section*{Acknowledgements}
Thanks to Antoine Groudiev for his precious insights on the direction the snakes for \eqref{eq:snek1} and \eqref{eq:snek2} should face.
\clearpage
\appendix
\bibliographystyle{main_natbib}
\bibliography{tdparse.bib}

\section{Other Considered Things}
\subsection{Typing with a products Category (and a bit of polymorphism)}
\label{app:prodcat}
Another way to start would be to consider product categories: one for the main
type system and one for the effects.
Let $\mC_{0}$ be a closed cartesian category representing our main type system.
Here we again consider constants and full computations as functions
$\bot \to \tau$ or $\tau \in \mathrm{Obj}\left( \mC_{0} \right)$.
Now, to type functions and functors, we need to consider a second category:
We consider $\mC_{1}$ the category representing the free monoid on
$\mF\left( \mL \right)$.
Monads and Applicatives will generate relations in that monoid.
To ease notation we will denote \emph{functor types} in $\mC_{1}$ as lists
written with head on the left.

Finally, let $\mC = \mC_{0} \times \mC_{1}$ be the product category.
This will be our typing category.
This means that the real type of objects will be
$\left( \bot \to \tau, [] \right)$, which we will still denote by $\tau$.
We will denote by $F_{n} \cdots F_{0} \tau$ the type of an object, as if it
were a composition of functions.

In that paradigm, functors simply append to the head of the \emph{functor type}
while functions will take a polymorphic form:
$x: L\tau_{1} \mapsto \phi x: L\tau_{2}$ and $\phi$'s type can be written as
$\star\tau_{1} \to \star\tau_{2}$.

\subsection{Multiple Ways}
\label{app:arities-and-denots}
In this section we will discuss what can happen when the denotation system
we consider is not actually based on a compositional model, or is independent
of the syntactic model.
In that case, our model of using typing to retrieve parsing trees will need
modifications to be integrated.

In the case where the compositional aspects of the sentence are blurred by a
non-binary arity, whether it's variable (leading to non-binary trees) or simply
non-compositional and makes use of all the words of the sentence (or the ones
prior to the last one read, like in a LLM), we can make use of higher arity
base combinators.
We replace the function application by the combinator adapted to our model,
then a theoretical (or learned) analysis of the concepts that can be modeled
by adding effects can still be done, and our work can be completed in the way
described above.

Of course in the case where no syntactic structure is used for semantics,
using semantics to derive syntax is useless, but the formalism of string
diagrams used to present the actual parsing can still be useful:
it could also be used to model the attention patterns (as side-effects of the
reading of a new token and adding its meaning to the current representation
of the sentence).
This is an insight of how to implement our enhancement onto a non-compositional
system, although that of course removes a bit of the interest of the system, as
typing will not matter in that case.

\subsection{Coproduct Integration}
In this section based on the work by
\cite{marcollimatildeetchomskynoametberwickrobertc.MathematicalStructureSyntactic}
and \cite{senturiaAlgebraicStructureMorphosyntax2025},
we explore the integration of our notion inside the structure of syntactic
merge, and why the two formalisms are compatible.
The idea behind that structure is to see the union of trees as a product and
the merging of trees as based on a coproduct in a well-defined Hopf algebra.
Now of course, with our insights on syntactico-semantic parsing, we can either
think of our parsing structure labeled by multiple modes, as in
\cite{bumfordEffectdrivenInterpretationFunctors2025}, or think of it as string
diagrams, as presented above.
In both cases, we make a more or less implicit use of the merge operation.

\medskip

Labeled trees can be seen as computed from a sequence of labeled merge.
As such, it is easy to see how one can adapt the construction of a purely
syntactic merge onto a semantic merge, using a similar approach.
There is just the need for a coloured operad to create a series of labeled
merges, like proposed in
\cite{melliesCategoricalContoursChomskySchutzenberger2025} for example.
Since there is a one-to-one mapping from our string diagrams to parse trees,
mapping explained when comparing the tables for denotations of combinators
in Figure \ref{fig:combinator-denotations} and Figure 12 in
\cite{bumfordEffectdrivenInterpretationFunctors2025}, or by looking at the
parse trees equivalent to string diagrams in Figures \ref{fig:parsing-diagram},
\ref{fig:parsing-diagram2} and \ref{fig:3dparsing-diagram}, it is easy to see
that indeed, there is a notion of merge inside our diagrams that can in a way
be expressed through a Hopf algebra, by transporting the diagrams to and from
their equivalent parsing diagram.
More generally, what this means is that a merge, in our string diagrams, is
the addition of a combinator to one, two\footnote{Or more, see
	\ref{app:arities-and-denots}} without conditions.
This is useful as a definition, because it allows integration of our system in
a broader framework, including for example the notion of morphosyntactic trees.

\medskip

However, this is not fully satisfying: like for syntax trees, applying a random
sequence of merges to a set of input strings will not always yield a properly
typed denotation.
Since type soundness is the main feature of our system, it seems weird to have
a definition of merge which cannot take that into account.



\end{document}
